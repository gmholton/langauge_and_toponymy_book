\chapter[Direct borrowings and loan-translations of Navajo toponyms into New Mexican Spanish]{\vspace{-1.5in}
 {\normalfont\normalsize
\hspace{0.2\textwidth}\parbox{0.6\textwidth}{
\textit{``The human species...is composed of two distinct races, the men who borrow, and the men who lend.''}\\[2pt] \strut\hfill Charles Lamb, 1823\\\bigskip}}
\\ Direct borrowings and loan-translations of Navajo toponyms into New Mexican Spanish: Examples and explanations}


\sethandle{10125/24846}

\def\authorlast{Jett}
\renewcommand{\beginchapter}{\pageref{jett-ch-begin}}
\renewcommand{\finishchapter}{\pageref{jett-ch-end}}
\label{jett-ch-begin}



\thispagestyle{firststyle}

\chapauth{Stephen C. Jett}
\affiliation{University of California, Davis}
\authortoc{Stephen C. Jett}

\begin{Abstract}
Although Navajo culture reflects fusion with pre-existing Native cultures in the U.S. Southwest, the Navajo retained the language of the Athabaskan-speaking component that migrated southward from western Canada well over half a millennium ago.  Like other Athabaskan languages, Navajo resists linguistic borrowing and contains a minimum of placenames originating by either direct loan or loan-translation.  New Mexican Traditional Spanish, on the other hand, incorporated a fair number of toponyms from Navajo, occasionally by direct borrowing (of which seven probable examples are provided here) but much more often in the form of calques and quasi-calques (of which nearly three dozen likely instances are given).  This asymmetry reflects not only the intrinsic borrowing propensities of the two languages but also 1) the priority of Navajo in the region; 2) Hispanos’ making more, larger, and better-organized trading, slaving, and punitive intrusions into Navajo Country than did Navajos into Hispano territory; 3) post-1846 Anglo-Americans’ undertaking official exploratory and military expeditions into Navajo Country; and 4) Euroamericans’ use not only of Puebloan and Hispano guides and support personnel but also of guides and warriors from the functionally bilingual Cebolleta Navajo band, which cooperated against Navajos elsewhere.  Slaves who escaped or were released brought topographic and toponymic information back home as well, and were subsequently sometimes employed as guides. In addition, to an extent Navajo served as a regional lingua franca.
\end{Abstract}


%\vspace{1em}
\clearpage
\noindent
Placenames often lie on the land in metaphorical layers, and may “provide vital evidence for dating and, indeed, for estimating the mixture of races” over time (Gelling 1988:dust jacket).  They frequently reflect “the struggles for place that underlie the actual socio-cultural processes involved in naming[,] itself....  Recently there has been a recognition that place names ‘mark the spatiality of power relationships’ (Myers 1996: 237)” (Kearns \& Berg 2002: 284).  These things are well manifested in the American Southwest.

\section{The Navajo and the Navajo Language}

The Navajo (\textit{Dine’é} ‘People’) are a very populous American Indian nation of the southwestern United States, whose extensive reservations, allotted lands, and other occupied places (\textit{Diné Bikeyah} ‘Navajoland’)—which exceed West Virginia in total area—lie in the Four Corners Region of New Mexico, Arizona, and Utah (Kluckhohn \& Leighton 1946; Underhill 1956; Young 1961; Downs 1972; Iverson 2002).  The Navajo language is a Southern Athabaskan (Apachean) one, closely related to Northern Athabaskan (Dene) tongues of western Canada such as Chipewyan (\textit{Dënes\k{u}łiné}), Carrier (\textit{Dakelh),} Sarci (\textit{Tsuu T'ina}), and Beaver (\textit{Dane-zaa}), and less-closely akin to Alaskan Athabaskan idioms (Hoijer 1956; Dyen \& Aberle 1974; Young 1983).  Athabaskan is (with recently extinct Eyak) a branch of North America’s Na-Dené linguistic family; the latter appears to relate to western Siberia’s Yeniseian languages (Kari \& Potter 2010).

By at least A.D. 1400 and very possibly up to a century or more earlier, ancestral Athabaskan-speaking pre-Navajo hunter/fisher/gatherer peoples had moved southward along the Rocky Mountains from Canada, to the upper-middle San Juan River drainage of northern New Mexico (\textit{Dinétah} ‘Navajo Country’; Jett 1964; Seymour 2012; Matson \& Magne 2013; Haskell 1987; Perry 1991; Ives et al. 2014), merging to some extent with already-present non-Athabaskan farming indigenes (Brugge 2006) and adopting crop-raising, pottery-making (Brugge 1973), and Plains-Southwest fish-avoidance (Simoons 1994: 254).  The merger yielded a culture whose early form may today be echoed more in certain non-Navajo Apachean societies than among the \textit{Dine’é}.  Proto-Navajo culture was a hybrid, then, one which experienced subsequent cultural fusion with horticultural Puebloan culture, particularly from Tanoan-speaking Jémez but also from Zuni (whose language may be an isolate) and Uto-Aztecan- (UA-) speaking Hopi (on whose toponyms, see Ferguson et al. 1985 and Hedquist et al. 2014, respectively) and possibly Western Keres-speaking Ácoma and Laguna, occurring particularly during the latest seventeenth and early eighteenth centuries and yielding fully Navajo culture, which became increasingly dependent on farming and on livestock-raising based on Spanish-introduced sheep, goats, horses, and donkeys but which remained more mobile than that of Puebloans (Jett 1978).  This was accompanied by westward and southward territorial expansion (Hester 1962; Towner 1996) into Anasazi- (Ancestral Puebloan-) abandoned lands, a movement eventually accelerated by Ute raids after the latter had acquired firearms, which triggered evacuation of most of \textit{Dinétah}.  But despite the aforementioned fusion plus the absorption of a scattering of Utes, Southern Paiutes, Apaches, Hispanos, and so on, and despite a renowned general Navajo (and pan-Athabaskan) pragmatic tendency toward selective non-linguistic cultural incorporation (Farmer 1953; Vogt 1961: 327–328), the language (\textit{Diné Bizaad}) “has entirely preserved its original form” (Mirkowich 1941: 313).  As Jack Ives et al. (2014: 619) asserted, “Dene communities ... have a strong proclivity to protect language identities, yet they readily accept material culture and ceremonial life from neighboring societies.”

\section{Navajo, Spanish, and Toponymic Borrowing}

It is somewhat puzzling to read the following in the estimable \textit{The Spanish Language of New Mexico and Southern Colorado: A Linguistic Atlas}:

\begin{quote}
	Cobos (2003) \ldots identifies only two words as possibly from Navajo (\textit{chihuil} ‘small valley’ and \textit{josquere} in the phrase \textit{andar en el josquere} ‘to sow one’s wild oats’)....  However, the reverse direction of lexical influence was more substantial.  In Navajo, for example, the terms for ‘money (\textit{béeso} {\textless} \textit{peso}) and ‘Anglo’ (\textit{bilagáana}) were borrowed from Spanish \textit{peso} and \textit{Americano}, to mention only two everyday words.  (Bills \& Vigil 2008: 154)
\end{quote}

I have not identified plausible Navajo sources for either \textit{chihuil} (possibly, \textit{chíí’} ‘red’?) or \textit{josquere} (Studerus 2001: 73 deemed the former to be ``Of uncertain Indian origin”).  Neither am I, offhand, aware of Spanish non-toponymic lexical loans into Navajo other than the two that Bills and Vigil cite, plus \textit{siláo} {\textless} \textit{soldado} ‘soldier’, \textit{mandagíiya} {\textless} \textit{mantequilla} ‘butter’, \textit{bááh} \textless \textit{pan} `loaf bread', \textit{gowééh} \textless \textit{café} `coffee', \textit{bilasáana} {\textless} \textit{manzana} ‘apple’, \textit{tres} ‘three-spot playing-card’ {\textless} \textit{tres }‘three’, \textit{gíinse }‘fifteeen cents’ {\textless} \textit{quinze }‘fifteen’, and \textit{béégashi }‘cow’ {\textless} \textit{vaca}.  All these represent entities alien to the pre-Conquest \textit{Diné’é}.

These authors go on to observe that while Puebloan languages have absorbed numbers of words from Spanish, “it is most certainly the case that [New Mexican] Traditional Spanish has seen exceedingly little linguistic influence from the local Pueblo languages after four centuries of contact,” displaying a mere 16 such borrowings, several of which remain tentative.  They attribute this picture to there being several Puebloan tongues, to the existence of a proprietary Puebloan attitude respecting their languages, and to the Spaniards’ having been conquerors imbued with a sense of linguistic superiority.  The fact that the settlers’ were technologically more advanced and organizationally more elaborated presumably also played a notable role.

Adoption of pre-existing foreign placenames is one of the most common kinds of linguistic borrowing in contact situations, although, unsurprisingly, the degree of acquisition appears to be significantly conditioned by the hostile or pacific nature of the encounters involved (Callaghan \& Gamble 1996: 111–112) as well as by the level of intensity of the interactions.  Too, languages differ in terms of their degrees of intrinsic openness to the taking-in of foreign toponyms.  As has been mentioned, Athabaskan languages in general (including Navajo; Mirkowich 1941; La Farge 1975: 243; Jett 2001: 207; Kari 2011) are noted for their conservatism and their resistance to borrowing from other idioms (Sapir 1921: 209)---although the percentages of loan words vary dramatically among the many contemporary Athabaskan tongues (Brown 1994: 105).  This resistance applies to both general vocabulary and placenames but particularly to the latter (e.g., Kari 1989: 129).  Kari (2010) attributed this to both the complex Athabaskan verb morphology and speakers’ territorial ethos.

Toponymic conservatism may be especially strong in Navajo because adoption and adaptation of aspects of Puebloan mythology and ceremonialism have contributed to the Navajo perception that their topographic names were, for the most part, assigned by the supernaturals in mythic pre-\textit{Dine’é }times and are, therefore, divinely sanctioned and immutable (Jett 2014: 107; on Navajo religion, see Reichard 1990; on sacred places, see Watson 1964; Van Valkenburg 1974; Jett 1992; Kelley \& Francis 1994).

Nearly all Navajo toponyms are analyzable; typically, they are descriptive, sometimes possessive, and are neither corruptions of names borrowed from other languages nor currently-unsemantic archaic artifacts or otherwise-unetymological entities.  Some Puebloan-derived onamastic calques (names borrowed by translation) may exist in Navajo (most likely, of certain sacred places), but this is, so far, an entirely uninvestigated topic and my own limited observations suggest a dearth of such calques.  The sole example that has come to my attention is that of Ganado, AZ’s, Pueblo III- (P-III-) period Wide Reed Ruin, on Pueblo Colorado Wash, Navajo \textit{Lok’aah} (\textit{taa}) (\textit{Kin} [\textit{dah}])\textit{ Nteel} ‘Wide (Elevated) (House amidst) Reeds’, attested in 1902 (Ostermann 2004b:173; also, Franciscan Fathers 1910: 131) {\textless} Hopi \textit{Wukopakabi} ‘Wide Reed Place’, plausibly at the \textit{Ojo de Carrizo} ‘Reed Spring’ shown on Miera’s 1759 map (Kessell 2013: 36) and the \textit{Ojo del Carrizo} ‘Spring of the Reed’ mentioned in 1836 (Correll 1979: 159).

On the other hand, toponymic borrowing by translation from Navajo into Traditional Spanish has demonstrably occurred—if not massively, at least in multiple instances.  As the anthropologist for the Navajo Richard F. Van Valkenburg put it, “For some reason, with very few exceptions, Navaho names were shifted into Spanish use” (Van Valkenburg \& Walker 1945: 89; cf. Reichard 1990: 393), and he provided several examples (although his transcriptions and translations are in some cases a bit iffy); otherwise, the topic has never been systematically addressed.  The present paper aims to document, contextualize, and explain this phenomenon.

\section{New Mexican Spanish and Direct Toponymic Loans from Navajo}\label{jett:sec:3}

New Mexican Traditional Spanish evolved to dialect status during the centuries following the Colonial introduction of its ancestor, the then-current Mexican-Spanish “macro-dialect.”  Although it does not differ dramatically from other forms of New World Spanish, New Mexican does contain numbers of distinct lexemes, pronunciations, syntaxes, and usages and it preserves some archaisms.  In northern New Mexico/southern Colorado, it further differentiated into the Rio Arriba and Rio Abajo subdialects (Cobos 1983; Julyan 1998; Bills \& Vigil 2008; see below).

In contrast to large-scale acceptance of native placenames in much of colonized Ibero-America, the direct incorporation into New Mexican Spanish of Navajo names was uncommon—although the native names of a number of Indian \textit{pueblos} (e.g., Jémez, NM, and Ácoma, NM) and certain other locations carrying Puebloan designations (e.g., Abiquiú, NM) were retained.  Still, over a half dozen examples of borrowing from Navajo may be forwarded, most involving major geographic areas.

Black Creek Canyon, AZ, a gorge of moderate length cut into the Cutler Formation redbeds of the Defiance Plateau.  Van Valkenburgh (1945: 90) suggested that the Spanish names for this canyon---\textit{Cañón Chenelle} and \textit{Cañón de Calites}---were likely to be corruptions of the Navajo\textit{ Teeł Ch'ínít'i' Tsékooh} ‘Reeds-/Cattails-Extend-out-Horizontally-in-a-Narrow-Line Rock Canyon’.  This seems likely for the first of the Spanish names (attested 1864, Madsen 2010:map), but the \textit{Calites} of the second is probably a reflex of the Mexican (and New Mexican) Spanish Nahuatlism \textit{quelites} ‘edible greens, wild spinach (such as wild lamb’s-quarters, a chenopod)’ {\textless} Nahua (Aztec) \textit{quilitl} ‘edible greens’ (Santamaría 1959: 903; Cobos 1983: 141; Asociación de Academias de la Lengua Española 2010: 1799;  Julyan 1998: 281; Bye 1981; Bills \& Vigil 2008: 15; David M. Brugge, personal communication).  \textit{Teeł Ch'ínít'i'} ‘Reeds/Cattails-Extend-out-Horizontally-in-a-Narrow-Line ‘/\textit{Los Quelites} ‘The Edible Greens’ is nearby Oak Springs, AZ, currently a tiny dispersed Navajo settlement in a valley heading the canyon; this basin appears as Calites Cañon on von Egloffstein’s 1864 map (Madsen 2010: map). The name (as ``the Quelites,'' ``the Callitas Valley,'', ``the Calites,'' and ``Quilities'') appears in government reports dating to 1857, 1864, and 1866 (Correll 1979: 88, 168, 308, 327).

Canyon de Chelly, AZ, a long, branched, sheer-walled gorge in northeastern Arizona.  In the eighteenth and nineteenth centuries, this canyon system (along with Chinle Valley, or Valley of the Chelly, downstream of the canyon mouth) was probably the most agriculturally-productive and populous area within the Navajo Country as well as affording “fortresses,” defensible difficult-of-access cliff shelters and promontories used as hideouts and strongholds.  This area was, unsurprisingly, a major target of slavers and military expeditions from the capital, Santa Fé, and from the greater Rio Abajo country to the city’s south (see below).  The name \textit{Chelly} (also, as \textit{Chegui} and in multiple other spelling variants over the years) represents the Spanish attempt to pronounce and spell the Navajo name \textit{Tséyi’} ‘Within Rock, i.e., [Sheer-Walled] Rock Canyon’, a name attested as early as 1777 as well as in 1778, 1786, 1796, and repeatedly thereafter (Correll 1979: 82ff; Kessell 2013: 38; Reeve 1971a:105); cf. \textit{Arroyo}/\textit{Río de Chelly} ‘Chelly River’ (Chinle/Navajo Wash/Creek), \textit{Valle de Chelly} ‘Chelly Valley’ (Chinle Valley), and \textit{Sierra del Cañón de Chelly} ‘Range of the Chelly Canyon’ (the [Fort] Defiance Plateau, into which the canyon system is cut; also, \textit{Sierra María} ‘Mary Range’ or, perhaps, \textit{Sierra Amarilla} ‘Yellow Range’, after Fort Defiance, AZ, described below (Jett 2001: 43–45; Rice 1970: 72; Eidenbach 2012: 97).\footnote{On the 1778 Bernardo Miera y Pacheco map of the 1776 Atanasio Domínguez and Silvestre Vélez de Escalante expedition as well as on his related 1778 map, the Carrizo/Lukachukai chain is labeled \textit{Sierra de Chegui} ‘Mountain Range of Chelly’ (McNitt 1972:endpapers; Bailey 1964a:53; Van Valkenburgh 1999: 18; Kessell 2013: 38; Eidenbach 2012: 53), and the name was also used on the 1810 Humboldt map (based on 1804 work; Eidenbach 2012: 63).  A (non-existent) east-west range shown on the 1828 John Disturnell map is called \textit{Monte Chegui} ‘Chelly Mountain’ (Tyler 1985:frontis).  This designation appears later to have been transferred westward to the Defiance Plateau (which today’s local Navajos refer to as `the mountain').}

Chaco Canyon, NM, a broad, fairly-shallow, sandstone-walled east-west-trending canyon crossing the San Juan Basin:  \textit{Cañón de Chaco} ‘Chaco Canyon’ {\textless} Navajo \textit{Tsékooh}/\textit{Chékooh} ‘Rock Drainage or Canyon’ ({\textless} a generic); another suggested (but less-plausible) derivation of Chaco is {\textless} Navajo \textit{Tségai} ‘White Rock’ (Brugge 1986: 1).  An additional if unconvincing hypothesis is that the name derives from \textit{Tséyaa Chahałheeł Nlíní }‘The Darkness-Rock-Shelter Stream’ (Pearce 1965: 30; Linford 2005: 43).  The name \textit{Chaca} appears on Miera y Pacheco maps of 1778; Kessell 2013: 39; Van Valkenburgh 1999: 19; Briggs 1976) and on the 1828 Disturnell map (Tyler 1985:frontis).\footnote{Van Valkenburgh (1999: 19) proposed that Navajo \textit{tsékooh} was inspired by the Spanish designation \textit{Chaco} rather than the reverse, \textit{Chaco} being a name of mysterious derivation.  I think this unlikely.  Although the shift from \textit{é} (/eh/) in Navajo to (stressed) \textit{a} (/ah/) in Spanish is puzzling, there \textit{is} the (not-entirely-comparable) example in Spanish of \textit{San Diego}/\textit{Santiago}); note, also, \textit{ciénega}/\textit{ciénaga} and \textit{quelites}/\textit{calites}, discussed in the text.  Too, in New Mexican conjugated forms of the verb \textit{haber}, what is \textit{e} in Castillian may become \textit{a} (Bills \& Vigil 2008: 146).  There is an 1890 rendering, referring to the mid-nineteenth century, \textit{Río Chueco} (Conrad 1890: 228).  Perhaps the presence, in Colonial New Mexico, of the Spanish surname Chacón had an influence.  There is also the Spanish term \textit{charco} ‘puddle, pool’, which can refer to a \textit{tinaja} ‘jar’ (but, in the Southwest, also to a rock pool in the bed of an ephemeral stream) or to a stock tank (Cobos 1983: 224); the name \textit{Cañon del Charco} actually appears on an 1864 map (Eidenbach 2012: 120).  Note that although \textit{tsé} is the standard Navajo word for ‘rock’, it is modified to \textit{ché-} in \textit{chékoh}, in \textit{chézhin} ‘black [volcanic] rock’, and in several other compounds.  Note that the Spanish \textit{vaca} ‘cow’ entered Navajo as \textit{béégashii}.} The (redundant) \textit{Cañón de Chaco} is attested for 1819, 1821, the 1840s (Correll 1979: 120, 122, 170, 178, 185, 369, 401), 1858 (Brugge 1980: 14), and later; a \textit{Mesa de Chaco}/\textit{Chaca }‘Chaco Tableland’—presumably, adjacent Chacra Mesa (also known in Spanish as \textit{Mesa Azul} ‘Blue Tableland’)—was referred to in 1768, 1774, 1812, 1845 (Brugge 1980: 9–10, 13), and 1853 (Bailey 1964b:52; Brugge 1965: 17; Linford 2000: 183–184, 190–191), and a \textit{Río de Chaco} ‘Chaco River’, i.e., Chaco Wash, in 1864 (Madsen 2010:map).  \textit{Chacra} appears on a 1774 Miera map (Eidenbach 2012: 50, 53).  T. M. Pearce (1965: 30) asserted that that name was regional dialect for ‘desert’ (the word does not appear in Cobos 1983 or Bills \& Vigil 2008).  Note that Google Translate defines \textit{chacra} as ‘ranch’ and \textit{chaco} as ‘territory crossed by rivers and streams that form lakes and swamps’—not an apt description of normally-dry Chaco Canyon, whose floodplain seems not to have included significant swales.

“Chacolí,” NM.  Documented earlier than \textit{Chaco} (but not than \textit{Chaca}) is, to the southeast of Chaco Canyon, NM, and to the west of Jémez, NM, the \textit{Río de Chacoli} (or \textit{Chacalina} or \textit{Chaculín}) ‘Chacoli (or Chacalina) River’, in the formerly Navajo-occupied \textit{Valle de Chacoli} ‘Chacoli Valley’.  Díaz (2014: 61), writing of the 1779 Miera map, considered \textit{Chacolina} to be “possibly a reference to Chaco Wash,” perhaps thinking \textit{{}-ina} to be a diminuative suffix.  However, the places are in fact distinct.  This map of Miera’s displays the earliest appearance of \textit{Chacoli} that I have found, indicated to be a right-bank tributary of the \textit{Río Salado} ‘Salty River’, which flows into the \textit{Río Jémez} (Kessell 2013: 81; Eidenbach 2012: 54).  The \textit{Salado }does indeed have a right-bank tributary mapped today as \textit{Arroyo Cachulie }‘Cachuli Drainage’.  At the head of the \textit{Chacoli} is, according to Miera, the \textit{Ojo del Espititu Santo} ‘Spring of the Holy Spirit/Ghost’, conceivably today’s Soda Spring.  This is on the large historic Ojo del Espiritu Santo grant, established in 1815 (or possibly two decades earlier) and said to lie some six leagues from Jémez, NM (see Bowden 2004; USGS topographic map \textit{Holy Ghost Spring, NM}; if the Spanish geographical league of the era is meant, the distance would be about 24 miles).  It is conceivable that the topo map’s Cachulie is a misprint and that Chaculie was meant; this kind of error, though rare, does occasionally occur on such maps.  In Spanish, a \textit{cachuli} is a meal prepared to accompany a festive \textit{matanza} ‘livestock slaughtering/butchering (literally, killing)’ (El Bienhablao n.d.), which is a New Mexico tradition (The Santa Fe Site n.d.).  The arroyo could have been named for a \textit{matanza} meal that took place there long ago.  However, considering the apt location, it seems more likely that, instead, the name here represents modification of the original \textit{Chacolí} by metathesis, under the influence of the lexeme \textit{cachuli}.  Against this is the fact that today’s hot Holy Ghost Spring lies in the valley of Lopez Arroyo, a \textit{left}{}-bank tributary of the \textit{Salado} (Julyan 1998: 247).  This could be explained by positing that the latter spring takes its name from that of the land grant and is not the spring that originally lent its name to the grant.  In 1796, the toponym \textit{Chacoli }was listed with Chelly and Sevolleta (Cebolleta; see below), among other places (Correll 1979: 92; Reeve 1971a:105), suggesting that it possessed some importance.  In 1823, the explorer Antonio Vizcarra placed his \textit{El Chacoli} camp nine leagues (35.5 miles) from Jémez and two leagues (circa 8 miles) from the \textit{Río Puerco} (Linford 2005: 44)—which seems somewhat distant from both places but otherwise consistent (\textit{Río Puerco} ‘Dirty, Muddy [literally, Pig] River’ of the East; see Julyan 1998: 278; Bills \& Vigil 2008: 126; Navajo \textit{Naasisé }[\textit{Tééh}] ‘Enemy-Belt [Water Channel]’).  The designation\textit{ Chaculín }also appeared in\textit{ }1823 (Brugge 1980: 11).  In 1849, Lt. James H. Simpson stated that the \textit{Río de Chacoli} was “a small affluent of the Puerco.... a running stream four feet in breadth and a few inches in depth,” not part of the \textit{Río Jémez} watershed (McNitt 1964: 26, 28-29).  Richard H. Kern’s diary refers to this area as the “Chacalina valley.”  His map of the expedition shows the route crossing the \textit{Valle de Chacoli}, which runs southwestward to the \textit{Puerco} and which could be interpreted as the \textit{Rito Olguín} ‘Olguín Little River’ (Olguín is a surname).  However, the map shows the (unlabeled) \textit{Río Salado} also joining the \textit{Puerco} nearby (Fort Huachuca, Arizona n.d.), which is contrary to reality.  In light of all this confusion and considering Simpson’s description of his route (pp.\ 25-28), my surmise is that what Simpson thought to be a Puerco tributary and the \textit{Chacolí} was, in reality, the \textit{Salado} and that the party crossed the latter and then ascended the Cachulie (“a very shallow valley”; p.\ 28) and traversed the low  divide into the Puerco Valley.  \textit{Chacolí} and \textit{Chaculín}/\textit{Chacalín}[\textit{a}] would seem certainly to be Spanish versions of the (here, undocumented) Navajo \textit{Chékoohl\'{\k{i}}}[\textit{ní}]\textit{ }‘[The] Rock-Drainage Flow’.\footnote{The fact that a red wine of Viscayan Spain is called \textit{chacolí} (Williams 1955: 2:168) is doubtless fortuitous.  The word there is Basque, \textit{txakolin}, with \textit{txacolina} meaning ‘the chacolí’ (Wikipedia, “Txacoli”).  Similar, Kichwa-derived terms in South American Spanish also seem irrelevant.} A difficulty is that the Cachulie watershed is not rocky.  The \textit{Rito Olguín} does qualify as a rock drainage, but otherwise seems unlikely. Possibly relevant is a left-bank \textit{Salado }tributary downstream of the \textit{Arroyo Cachulie }called \textit{Arroyo Peñasco} ‘Boulder Waterflow’.  I speculate that the whole \textit{Salado} drainage—much of which is cliff-girt—constituted \textit{Chacolí} and that these tributary names are onamastic survivals in different forms.  Today’s Holy Ghost Spring\textit{ }could then be the same as Miera’s of 1779.  Although there exists the Mexican Nahuatlism \textit{chacalín} ‘freshwater prawn’ {\textless} Nahua \textit{chaculin} ‘\textit{Macrobrachium}’ (Asociación de Academias de la Lengua Española 2010: 464), in the U.S. Southwest neither that genus nor crayfish occurs to the west of the Great Plains (Bowles, Aziz \& Knight 2000; Wikipedia “Crayfish”).


Chuska Mountains, AZ/NM, a long, elevated plateau, which separates the Arizona and the New Mexico sections of the Navajo Country.  Also included with the aforementioned places in a 1796 list (as well as, frequently, in later years) are (\textit{Sierras de}) \textit{Chusca}, \textit{Tunicha}, and \textit{Carrizo}.  These (from south to north) are the three divisions of the forested north-south-trending Chuska (formerly, Tunicha) Mountains.  The first two names derive, respectively, from Navajo \textit{Ch'óshgai} ‘White Spruce’ (for Chuska Peak; Van Valkenburgh 1974: 171; Spanish version attested for 1776 and 1778; Julyan 1998: 83; Kessell 2013: 38; Eidenbach 2012: 53; also seen as Choiskai Plateau; Dutton 1886: 130) and from \textit{Tó Ntsaa} ‘Big Water/Spring’ (cf. Spanish \textit{Río Tunicha} ‘Tunicha River’, Chaco Wash); \textit{Carrizo} is a calque, discussed below (Correll 1979: 93, 114, 122, 143, 146, 147, 174; see also, Barnes \& Granger 1960: 8).

Penistaja, NM, a Hispano ghost village some ten miles to the west of Cuba, NM, in the \textit{Río Puerco} of the East drainage (see “Chacolí,” above; also, below).  According to Julyan (1998: 263), while the name Penistaja “has been said to be derived from the Spanish \textit{peña}, ‘boulder,’ more likely it’s a corruption of the Navajo name of the place, transliterated as \textit{binishdaahi” }(\textit{Bíniishdáhí }‘The One I Sit Down Against’).  On the other hand, Young and Morgan (1980: 243; followed by Kelley \& Francis 2018: 292) asserted that “The Navajo name mimics the Spanish.”  Thus, in this instance the direction of borrowing is ambiguous.


\section{Examples of Navajo Toponymic Calques in Southwestern Spanish}\label{jett:sec:4}

Spanish placenaming in Colonial New Mexico heavily stressed Roman Catholic Christian saints’ names; names of biblical concepts, miracles, and places; and the like.  However, in the North American Southwest many Spanish descriptive toponyms were coined, as well (e.g., \textit{Agua Azul} ‘Blue Water’, NM), plus a few metaphorical ones (e.g., NM’s \textit{La Alesna} {\textless} Iberian \textit{La Lezna} ‘The Awl’ for a particular pointed volcanic neck; variants of \textit{Corona Gigante}/\textit{Corona de Geganta }‘Giant Crown/Giantess’s Crown’ for flat-topped, cliff-girt Beautiful Mountain, NM, Navajo \textit{Dzilk'i Hózhónii} ‘Beautiful-on-Top Mountain’; Van Valkenburgh \& Walker 1945: 90).  Family and full names of local Hispano settlers were also often applied to canyons and to other landscape features that those settlers utilized—e.g., \textit{Cañon Juan Tafoya}, NM\textit{—}as well as to small settlements (e.g., Marquez, NM, Los Lunas, NM).  In addition, a not-negligible number of descriptive Navajo placenames (and at least a few others in additional local Native languages) were adopted into Spanish via loan-translation.  Indian-fighting U.S. Army commanders often employed local Hispano support personnel as well as New Mexican irregular troops,\footnote{Readily available print and electronic reproductions of relevant historic maps are often not of sufficient resolution to be fully legible.} and historical attestations of such Navajo-derived calques come as much from post-Mexican-American War (i.e., 1846 and later) U.S. military and exploratory reports, maps, and diaries as from Spanish and Mexican documentary sources.\footnote{For synopses of, and excerpts from, primary documents relating to the Navajo, see Correll 1979; on 1821–1848 regional documents, see Tyler 1984; also, more-comprehensive Twitchell 1976; Beers 1979.}  These translated names generally give the impression of having been well established by the times of their initial documentary attestations.

  Without undertaking an exhaustive search of the literature, I have identified the entries in the following list of 34 examples (depending on how one counts them) of likely toponymic calques or quasi-calques from Navajo in eighteenth- and nineteenth-century New Mexican Spanish, and have included early dates of demonstration of their presences; without doubt, other such loan translations remain unidentified.\footnote{Sources for the Navajo names include Jett 1970, 2001, and personal field work; Van Valkenburgh 1974, 1999; Franciscan Fathers 1910; Haile 1950, 1951; Pearce 1965; Franstead ca. 1979; Young \& Morgan 1980; Downer ca. 1988, ca. 1989; Wilson \& Dennison 1995; Linford 2000, 2005; Bright 2013; see also, Jett 1997, 2006, 2011, 2014.  Orthography has been standardized according to the Young and Morgan system.  On Navajo Country geography, see Goodman 1982.  For an excellent map reference, see Automobile Club of Southern California n.d.}

Agathla Peak (Agathla Needle; \textit{El Capitán}, Spanish ‘The Captain’), AZ, a tall, striking, dark-colored volcanic neck in Monument Valley, sacred to the Navajo as a sky-support and as the breast of a huge conceptual female ground figure, and a landmark visible from great distances:  \textit{Lana Mucho} (sic) ‘Much Wool’, \textit{Lana Negra} ‘Black Wool’ (1859, McNitt 1972: 371; Van Valkenburgh 1945: 92; Bailey 1964a:67, given as \textit{Sana Negra} on p.\ 85; \textit{Lana Negra} on an 1860 map and on Frederick W. von Egloffstein’s 1864 synthetical map; Madsen 2010:map) {\textless} Navajo \textit{’Aghaał\'{ą}} ‘Much Wool’ (Gregory 1916: 48), from the traditional tale of members of the migrating original four (water) clans’ scraping the hair off deer (some say pronghorn or “yellow-legged-animal”) hides on the rough igneous rock, a practice known ethnographically (Matthews 1897: 154–157; Van Valkenburgh 1974: 77–79; 1999: 35; Downer ca. 1988: 35; Linford 2000: 33; Begay 2015a:3).

Bears Ears, UT, a pair of distantly-visible rounded-edged mesas atop Elk Ridge, on the northern edge of historical Navajo land-use:  \textit{Las Orejas} ‘The Ears’ (1823, Correll 1979: 136; Brugge 1965: 16), \textit{Orejas del Oso} ‘Ears of the Bear’ (1860, 1864, Madsen 2010: 39, map; van Cott 1990: 25) {\textless} Navajo \textit{Shashjaa’} ‘Bear Ears’ (today, also applied to the nearby town of Blanding, UT).

Bekidahatso Lakes, AZ, a sizable tight group of water bodies of fluctuating levels lying to the south of Chinle, AZ:  \textit{Laguna Grande }‘Large Lake’ (1859, Bailey 1964a:67) {\textless} Navajo \textit{Be’ek’id Hatsoh} ‘Large-Area Lake’ (Gregory 1916: 36, 118).  The lake cluster is also known as \textit{Tooh Dish’níd} ‘Water-Body Moans’, for vocalizations of a mythological once-resident water-monster (Van Valkenburgh 1999: 6; on \textit{tooh}, see Hotevilla, AZ, below).  Studerus (2001: 76) defined \textit{laguna} as ‘small lake’, so \textit{Laguna Grande }would seem a contradiction in terms; however, \textit{lago} ‘lake’ seems not normally to be used in the vernacular Southwest (Bills \& Vigil 2008: 271, 278; Cobos 1983: 136; Julyan 1996: 193).

Black Lake, NM; not today’s Black Lake, which lies on a tributary of Crystal Wash (Navajo name unascertained; probably the Spaniards’ \textit{El Salitre Negro} ‘The Black Saltpeter‘, likely reflecting alkaline deposits on the bed of a dry lake—and \textit{Cieneguilla Chiquita} ‘Tiny Little Swale’), in Black Salt Valley (Van Valkenburgh \& Walker 1945: 90).\footnote{\textit{Cieneguilla Chiquita} was probably named in contrast to \textit{Ciénega Grande} ‘Big Swale’, “a beautiful stream” with small ponds (1851; Rice 1970: 71–72)---likely, nearby Wheatfields Creek, to which Whiskey Creek is tributary.}  The historical Black Lake is, instead, dragoon James A. Bennett’s 1855 “Laguna Nigra, or Black Lake, which is 25 miles [northward] from the fort [Defiance]” (Brooks \& Reeve 1948: 68; 1853, Correll 1979: 404, 407).  What is now upper Black Creek’s left-bank tributary Tohdildonih Creek (Navajo \textit{Tó Dildo’ní} ‘At Water Pops/Explodes’) but was once considered part of Black Creek proper (today’s Navajo \textit{Be’ek’i}[\textit{d}] \textit{Halchííd’ęę’nlíní} ‘Stream from Red Lake’; see Red Lake, below), drains a former pond in Todilto Park, NM.  According to the Big Starway myth, this lake was the home of a water monster (Wheelwright \& McAllister 1988: 54).  Like proximate Red Lake, Black Lake was a historic meeting place for peace parlays.  Today, this defunct water body is evidenced by the flat floor of red-cliff-girt Todilto Park (Navajo \textit{Tó Dildǫ’} ‘Water Pops’) and by lacustrine sediments exposed in the sides of the incised channel:  \textit{Laguna Negra} (1859, Bailey 1964a: back endsheet; also on some other maps of the era) {\textless} Navajo \textit{Be’ek’id  Halzhin} ‘Black-Area Lake’ (Haile 1950: 63; 1951: 24).  One might suspect that \textit{Tó Dildǫ’} ‘Water Pops’ is a reference to the bursting of the old natural earthen barrier that dammed the lake, during the 1880s (see Van Valkenburgh 1999: 106; also, Julyan 1998: 355).  However, Haile (1950: 110) wrote that the place “is also called \textbf{t}\textbf{\textsuperscript{x}}\textbf{ó dildon cé dadešžahí }[\textit{Tó Dildon Tsé Dadeshzhahí}] \textit{the place of the jutting rock} (contour) \textit{popping spring}.”  Although identified by historians with Red Lake, the \textit{Ciénega del Peñasco Colorado} ‘Swale of the Red (literally, Colored) Crag’ visited in 1823 by the Vizcarra expedition (Acrey 1988: 136) is more likely Black Lake, near which rise the striking red-rock pinnacles known as Cleopatras Needle and Venus Needle (as, admittedly, do red cliffs near Red Lake; on \textit{ciénega}, see St. Michaels, AZ, below).

Borrego Pass, NM, a trading-post community in the Dutton Plateau area to the northeast of Gallup, NM:  According to Julyan (1998: 46), \textit{“}In NM, the Spanish \textit{borrego}, ‘yearling, lamb,’ became the general term for sheep....  The Navajo name means the same as the Spanish-English name”; it is \textit{Dibé Yázhí} \textit{Habitiin} ‘Lamb’s (literally, Small/Young Sheep’s) Trail up Out’.  Studerus (2001: 7) defined \textit{borrega} as ‘sheep’.

Bowl Canyon, NM, a substantial hollow on the west side of the Chuska Mountains, to the south of Crystal, NM, and adjacent to Todilto Park (see Black Lake, NM, above):  \textit{La Tinaja }‘The [large earthenware] Jar, Rock Tank’ (Van Valkenburgh \& Walker 1945: 91) {\textless} Navajo \textit{{}'Ásaahí} ‘The Jar’.  Today, the valley is holds Lake Asááyi in Bowl Canyon Recreation Area.

Buell Park, AZ, a circular caldera with mythic associations, including with Salt Woman and in the myth of Big Starway (Wheelwright \& McAllester 1988: 53), located not far to the west-southwest of Red Lake, NM (see below).  It was named for nineteenth-century U.S. Army officer Don Carlos Buell and was seemingly used by Army troops stationed at Fort Defiance, AZ, for occasional pasturage and perhaps for haying:  \textit{La Joya} ‘The Hollow/Large Earth Cavity’ (1859, Bailey 1964a:11, 37; New Mexican \textit{joyo/a} ‘hole, valley’ {\textless} Iberian Spanish \textit{hoyo}) {\textless} Navajo \textit{Ni' Haldzis} ‘Earth Hollow’ (Franciscan Fathers 1910: 130; Gregory 1917: 94).  In standard Spanish, \textit{La Joya} (\textit{La Jolla}) glosses as ‘The Jewel’; conceivably, this could reflect locally-occurring green olivines and red pyrope garnets, on the Park’s Peridot Ridge (Gregory 1917: 146; in 1870, Agent W. F. M. Arny mentioned “rubies” at Red Lake, AZ; in 1881, Bourke wrote that garnets were common near Fort Defiance, NM [Bloom 1936: 219; Murphy 1967: 30]), and one of the historic English-language variant names for the place is Jewell (along with Bule and Yule) Park (Barnes \& Granger 1960: 6).  However, in New Mexico \textit{joya} (or \textit{hoya}) “can also be translated as `valley, basin, hole’” (Julyan 1998: 180; also, 190; also, Cobos 1983: 93); cf. the Rio Grande Valley Hispano settlement La Joya, NM.  Another possibility is \textit{La Olla} ‘The Pot/[large earthenware] Water-Jar’; in Taos County lies “a huge volcanic feature that resembles a pot, \textit{Cerro de la Olla} ‘Hill of the Water-Jar’” (Julyan 1998: 73, 180, 190).  Note that some speakers of New Mexican Spanish add a \textit{j} (/h/) before an initial vowel (e.g., \textit{oso} {\textgreater} \textit{joso} ‘bear’; Bills \& Vigil 2008: 124; cf. Cockney English).

Cabezon Peak, NM, a dark, round-topped landmark volcanic neck in the drainage of the \textit{Río Puerco} of the East, sacred to the Navajo: El Cabezón \textless Navajo \textit{Yéé'itsoh Bitsiits'iin} `Big Monster's Head' (a sacred name; Williams \& Blackhorse n.d.: 4).  In 1860, Capt. John S. Newberry stated that the eminence “resembles in its outline a Spanish sombrero” (Madsen 2010: 91), which is more or less accurate.  It appears on the 1779 Miera map as \textit{El Cabezon de las Montoias} ‘The Big Head of the Hills and Valleys’ (if Montoya, a surname prominent in New Mexico, had been meant, \textit{Los Montoyas }would have been the spelling):  \textit{El Cabezón} ‘The Big Head’ (1778, 1778, Kessell 2013: 39, 92; McNitt 1972:endpapers; Eidenbach 2012: 53); \textit{Cerro Cabezón }‘Big-Head Hill’ was referred to in 1796, 1819, and 1829 (Correll 1979: 92, 119, 147; Reeve 1971a:105), and the eminence was noted in 1849 by Lt. James Simpson, who called it \textit{Cerro de la Cabeza} ‘Hill of the Head’ (McNitt 1964: 25, 28, 31, 34; repeated in 1864 and 1867, Eidenbach 2012: 120, 127); the same name appears on Capt. Allen Anderson’s 1864 map (Kelly 1970:map), and \textit{Cabezon} on the von Egloffstein map of 1864 (Masden 2010:map).  These names are not calques of the mundane Navajo name (\textit{Tsé Naajiin} ‘Black-Downward Rock’) but refer to the Upward-Movingway/Emergenceway, Monsterway, Enemyway, Big Godway, Shootingway, and Blessingway myths, in which the Hero Twins, \textit{Naayéé’ Neizghání }‘Monster-Slayer’ and \textit{Tóbájíshchíní} ‘Born for Water’, dispatch the evil giant Big Monster (\textit{Yé’iitsoh}) at present-day San Rafael, NM’s, \textit{Ojo del Gallo}/\textit{de la}(\textit{s})\textit{ Gallina}(\textit{s}) ‘Spring of the Cock/Hen Chicken(s)’ (in this case, probably after \textit{gallo}/\textit{gallina de la tierra}/\textit{sierra} ‘ Wild Turkey’; Julyan 1998: 143–44; Bills \& Vigil 2008: 32–33; Studerus 2001: 8; 1778 and 1779, Kessell 2013: 39,92; Eidenbach 2012: 53, 54; 1851, Sitgreaves 1962:map), site of the first Fort Wingate, NM (also, \textit{Tó Sido} ‘Hot Water/Spring’; see also, Tocito Wash, NM, below).  The Twins toss Big Monster’s scalped and severed head northeastward (Haile 1938; Fishler 1953: 53-56; Haile 1981: 191; Zolbrod 1984: 221).  The head—now, petrified—is today’s Cabezon Peak (Matthews 1897: 116, 128, 234; Reichard 1990: 22, 71, 392–393; Watson 1964: 11; Van Valkenburgh 1999: 10; Holmes 1989: 27, 34–34; see also, Dead Mans Wash, NM, and Los Gigantes Buttes, AZ, below); \textit{Cabezón} has been absorbed back into Navajo as \textit{Gaawasóón}; Van Valkenburgh 1999: 10; Young \& Morgan 1980: 838).  Cabezon, NM, is a nearby Hispano ghost village (Linford 2000: 180).

Carrizo Mountains, AZ, a prominent and sacred, more-or-less circular laccolithic range:  \textit{Sierra Carrizo} ‘Reed Mountain Range’, a name that was erroneously transferred from the somewhat-more-southerly Lukachukai Mountains {\textless} Navajo \textit{Lók’a’ch’égai} ‘White Streak of Reeds Extends Out Horizontally’.  “The patch of reeds is in a cove at the western base of Lukachukai [Buffalo] Pass.  Nearby is Lukachukai settlement” (Barnes \& Granger 1960: 14; also, Van Valknburgh 1999: 63).  A range called \textit{Carrizo} was mentioned in 1796, 1819, 1821, and 1823, as well as later (Correll 1979: 92, 114, 133, 166, 359, 380; Reeve 1971a:105; 1971b:227).  In 1851, a tributary of Chinle Wash, running in a \textit{cañada} ‘ravine’ (usually, in New Mexico, a large canyon or valley, according to Julyan 1998: 59; Cobos 1983: 27 said a small, dry, mountain canyon; ‘small canyon or dry river’, according to Studerus 2001: 74), was referred to as \textit{Carriso} (Correll 1979: 285), and \textit{Arroyo de Carrizo} ‘Reed Drainage’ was mentioned in 1853 (McNitt 1972: 348–349); Agent Walter G. Marmon (1894: 156) confirmed that these references were to “Lu-ka-chu-kai or Carrizo creek,” today’s Lukachukai Wash, flowing from the Lukachukai Mountains section of the Chuska Mountains to Chinle Wash, passing the modern community of Lukachukai, AZ (on \textit{ojo}, see Tocito Wash, NM, below).  A creek labeled \textit{Cariso} on Anderson’s 1864 map (Kelly 1970) and “Carriso Creek” on von Egloffstein’s map of the same date (Madsen 2010:map) corresponds to Lukachukai Wash (a \textit{Cañón de Carrizo} between today’s Lukachukais and Carrizos is also shown, perhaps reflecting a confusion consequent upon the transfer of the name \textit{Sierra Carrizo}).  In 1859, Capt. J. G. Walker mentioned “the Carizo range,” apparently meaning the Lukachukais.  What are today called the Carrizos were designated Carizo Mt. on Lt. Amiel W. Whipple’s 1855 map (Bailey 1978:endpapers), \textit{Sierra de Carriso} on John N. McComb’s 1860 map, Carrizo Mts. on Anderson’s 1864 map, and \textit{Sierra Carriso} on von Egloffstein’s 1864 map (Bailey 1964a:53; Madsen 2010: 39, map; Eidenbach 2012: 119; Barnes \& Granger 1960: 7).  (Cf. Southwestern Range and Sheep Breeding Laboratory, NM, below.)  The Navajo name for the Carrizos is \textit{Dził Náhooziłii} ‘The Mountain Gropes Around’.

Cerro Colorado, NM, a sacred prominence that the Navajo consider to be the ninth mountain created, located some five miles to the northeast of San Luis, NM, in the watershed of the \textit{Río Puerco} of the East: \textit{Cerro(s) Colorado(s)} ‘Red (literally, Colored) Hill(s)’ {\textless} Navajo \textit{Dził Łchíí’} ‘Is-Red Mountain’ (Fishler 1953: 14–16; Van Valkenburgh 1974: 47, 49–50, 65; 1999: 15).  It has Windway and Mountaintopway ceremonial associations (Wyman 1975: 177, 185, 197; 1962: 61–62, 69).  (Note that in Spanish, \textit{cerro} ‘hill’ can refer not only to a modest eminence but also to a towering mountain; in New Mexico, a \textit{cerro} is usually larger than a \textit{loma} ‘hill’; Julyan 1998: 72, 206–207.)

Dead Mans Wash, NM, a stream that descends from the eastern side of the Lukachukai section of the Chuska Mountains and crosses the lower Chuska Valley to the San Juan River:  \textit{Río Pajarito/a} ‘Little-Bird River’ (Van Valkenburgh \& Walker 1945: 91; 1864, 1866, 1867, Eidenbach 2012: 120, 123, 127; Madsen 2010: map), perhaps {\textless} speculated Navajo \textit{’Ayáásh Bikooh } ‘Small-Bird Drainage’.  Nearby is Ship Rock Pinnacle (\textit{Tsé Bit'ą'í} ‘The Rock’s Wings’; Franciscan Fathers 1910: 130), a dramatic and sacred dark volcanic neck a few miles to the northwest, whose form, with its attendant pair of radiating dikes, is interpreted as being that of a giant fallen bird (O'Bryan 1956: 123–124; Julyan 1998: 333; Begay 2015b; also formerly known as The Needles and Pinnacle).  Here, according to myth, the Hero Twins (see Cabezon Peak, above, Los Gigantes Buttes, below) slay, atop the peak, the Rock (Eagle) Monsters—\textit{Tsé Nenáhaalii} ‘Drops onto Rocks’—whose nestlings, flung from the summit, become the initial major raptor birds, whose pinions became raptors and game birds, and whose downy feathers become the initial passerine and other smaller birds, brought to life by pollen from wild sunflowers growing on a dry lake bed (Zolbrod 1984: 232–241)---although to the west of the butte, not in the Dead Mans Wash drainage but in that of adjacent Ship Rock Wash (Newcomb 1970), speculated \textit{Tsé Bit'ą'í Bikooh}  ‘The-Rock’s-Wings Drainage’; in a second version, the birds obtain the feathers to the \textit{east} of Ship Rock.  Dead Mans Wash is the first one to the south and to the east of the diatreme, which has Enemyway, Mountaintopway, and Eagle Catchingway cremonial-myth associations (Haile 1938; Kluckhohn \& Leighton 1946: 127–128; Fishler 1953: 60).

(New) Fort Wingate, NM (established 1860), the main mid-nineteenth-century U.S. military post for control of, and services to, the Navajo (see also, Cabezon Peak, above):  \textit{Ojo del }(\textit{H})\textit{Oso }‘Spring of the Bear’ (1774, Brugge 1980: 9; 1778, 1779, 1843, Kessell 2013: 38, 92; McNitt 1972:endpapers; Eidenbach 2012: 53, 54; 1851, 1853, 1859, 1863, Correll 1979: 87, 174, 285; Foreman 1941: 128; Sitgreaves 1962:map; Bailey 1964a:98; 1966: 94; Kelly 1970: 27) {\textless} Navajo \textit{Shash Bitoo} ‘Bear's Spring’ (on \textit{ojo}, see Tocito Wash, NM, below).  Successive other names for this place are Bear Spring, Fort Fauntleroy, and Fort Lyon (Van Valkenburgh \& Walker 1945: 92; Linford 2000: 208).

Hotevilla, AZ, one of the Third Mesa Hopi villages, founded in 1906:  \textit{Ojos de las Cebollas} ‘Springs of the Onions’ (1864, Madsen 2010:map) {\textless} Navajo \textit{Tł’ohchintó} ‘(Wild) Onion Spring’ (Haile 1950: 339).  Note that in Navajo geographic names, \textit{Tó}/\textit{{}-tó}/\textit{{}-too} ‘water’ ordinarily means ‘spring’, and \textit{tooh} ‘water-body (usually, river)’; in some compounds, water is \textit{tá-}.  The Hopi name \textit{Hotvela} means ‘Scraped Back,” referring to the spring, access to which requires crawling through a tunnel (Gregory 1916: 242; Van Valkenburgh 1999: 55).

Kinlichee, AZ, one of a cluster of P-IV Anasazi (Ancestral Puebloan) ruins and a close-by modern Navajo community, not far to the east-northeast of Ganado, AZ (Barnes \& Granger 1960: 13–14, 19):  \textit{Pueblo Colorado }‘Red Village’ (1839, 1846, 1847, Correll 1979: 166, 209, 221; 1863, Barnes \& Granger 1960: 19) {\textless} Navajo \textit{Kin }(\textit{Dah Łi})\textit{chíí' }‘(Elevated Is-)Red House’.  The ruin was also known as Rincon\footnote{”The Spanish term ‘\textit{rincón}’ (the inside corner of a house or box. . .) is applied by the Mexicans of Arizona and New Mexico to the square-cut recesses between buttresses of cliffs and canyon walls.  They are distinguished from the semicircular recesses that resemble parts of an amphitheater and are differentiated from box canyons by the absence of defined drainage” (Gregory 1917: 132).  In the Southwest, while a topographic rincon “is usually defined as an angular reentrant in higher land with rimming cliffs, in New Mexico it is also a ridge and a tributary valley” (Burrill 1956: 230); a “box canyon” (Studerus 2001: 74).  This ruin is on the edge of the upland overlooking the mouth of the shallow canyon of Kinlichee Creek, the upper reach of Pueblo Colorado Wash.} Red House.  The structure is not, in fact, red, but dull yellowish; ‘red’ may be a mythological/ceremonial designation, as seems to be the case with \textit{Kin Dootł’izh} ‘Blue House’, applied to certain other ruins.  From the late eighteenth century through the mid-nineteenth century, Kinlichee (Camp Florilla), AZ, was an important camping point and remount outpost for military and other expeditions en route between Santa Fé, NM, or Fort Wingate, NM (see above), and the mouth of Canyon de Chelly or the Hopi Mesas in Arizona (Van Valkenburgh 1999: 18, 59).  In 1863, the U.S. army wrote of \textit{Pueblo Colorado} on the \textit{Río de Pueblo Colorado} ‘Red-Village River’ as a potential field headquarters site (Kelly 1970: 22-23, map; Trafzer 1982: 75, 80); the ruined pueblo is adjacent to today’s Pueblo Colorado Wash (also known historically as \textit{Río Pueblitos }‘Little-Pueblos River’, in the “Cañon of the Pueblitos”; 1859, Bailey 1964a:60, 64, 67, 79; 1864, Madsen 2010:map).  In northern New Mexico and Arizona (all in New Mexico Territory from 1850 to 1863), \textit{pueblo} ‘small population/settlement’ ({\textless} Latin \textit{populus}) may be applied to any village but normally refers to indigenous single- or multiple-story masonry “condominium” apartment buildings, including archaeological ones; the New Mexico Hispano village was usually termed \textit{plaza }or\textit{ placita, }whose standard Iberian meanings are ‘plaza’ and ’courtyard’ (\textit{villa} and \textit{villeta} ‘villa’ and ‘small villa’, here, ‘village’ and ’hamlet’, were little used, because the terms implied a royal sanction\footnote{E.g., \textit{La Villa Real de la Santa Fé de San Francisco de Asís} ‘The Royal Town of the Holy Faith of Saint \href{https://en.wikipedia.org/wiki/Francis_of_Assisi}{Francis of Assisi}’.}; Julyan 1998: 193, 276).  There is also a Red House to the southwest of Kayenta, AZ, and \textit{Kin Łichíí' }is the Navajo name of contemporary San Juan Pueblo, NM; the Hopi speak of a legendary \textit{Palátkwapi} ‘Red House of the south’ (Waters 1963: 67–71), perhaps Casa Grande, AZ, or Tumacoc Hill in Tucson, AZ.  (See also, Pueblo del Arroyo, NM, and Pueblo Pintado, NM, below.)

Los Gigantes Buttes, AZ, a pair of prominent square-shouldered, Wingate Sandstone buttes to the northeast of Round Rock, AZ:  \textit{Los Gigantes} ‘The Giants’ (1864, Madsen 2010:map; 1892, Eidenbach 2012: 149).  The Spanish term is not derived from the Navajo secular name for these outliers, \textit{Tsé Ch’ídeelzhah} ‘Rocks Jut Out’.  However, the eminences may possibly be perceived as the petrified forms of the Hero Twins of legend (see Cabezon Peak, NM, Dead Mans Wash, NM, above; Jett 1965), who are among the Holy People (\textit{Diné Diyinii}).  Individuals of one (ill-defined) class of malevolent supernaturals are commonly referred to as yeis: \textit{yé’ii}, literally ‘dread one’ but usually glossed as ‘monster’ or ‘giant’, especially in the instance of \textit{Yé’iitsoh} ‘Big Dread One’.  In the same general vicinity is a sacred cave containing clay-sculpture heads, collectively called \textit{Yé’ii Shiijéé’í  }‘The Yé’iis Lying Down’ and representing the dozen persuadable supernaturals known as \textit{haashch’ééh} or as \textit{yé’ii} \textit{bichaii} ‘yeis’ maternal great uncles’ (Jett 1982), and the buttes are not far from Dancing Rock near Rock Point, AZ, site of an annual Nightway (Yeibichei) ceremonial, which includes masked impersonators of the 12 \textit{haashch’ééh} (Matthews 1902).  So, while not, strictly speaking, a calque (unless of an undocumented ceremonial name), the Spanish toponym possibly refers to the Twins who may be represented by these buttes—if not them, then a pair of the monsters that they slew or a pair of dancing Yeibicheis—and thus may be what could be termed a “quasi-calque.”  (The Taos Puebloans associated the two pinnacles of Colorado’s Chimney Rock with the Twin War Gods, and the Navajo consider a pair of standing rocks near Blanco, NM, to be manifestations of the Twins that guard the Blanco Canyon entrance to Dinétah.)  There also exist Los Gigantes, NM, a set of standing rocks off a Zuni Sandstone cliff point to the east of Ramah, NM, attested on the 1760 and 1778 Miera maps (Kessell 2013: 34, 39; Hodge 1937: 106; Eidenbach 2012: 53), known in Navajo as \textit{Tsé ’Ałkéé’ T’ąą’} ‘Backwards-behind-One-Another Rock’.  They are not terribly far from the mythic scene described under Cabezon Peak, NM, above, and are said to be a family (of Holy People?) who were petrified as they walked along.  The twin Cuba Mesa Sandstone towers of Angel Peak, near Bloomfield, NM, were also known as \textit{Los Gigantes} (Julyan 1998: 16), so perhaps this is simply a New Mexican Spanish generic for co-occurring standing rocks (such pairs of rocks were also sometimes called \textit{Los Capitanes} ‘The Captains’, single ones \textit{El Capitán} ‘The Captain’).

Moenkopi Wash, AZ, a right-bank tributary of the lower Little Colorado River: \textit{Los Algodones} ‘The Cottons’ {\textless} Navajo \textit{Naak’a K’éédilyéhé} Where Cotton Is Raised’, after a one-time Hopi annual crop at Moenkopi, AZ.

Pueblo del Arroyo, NM, a P-III Chacoan Anasazi (Ancestral Puebloan) great-house ruin in Chaco Canyon, NM (1849, McNitt 1964: 49):  ‘Village of the Waterflow/Drainage’ {\textless} Navajo \textit{Tábąąhkíní} ‘The Riverbank House’ (on \textit{pueblo}, see Kinlichee, AZ, above).  Although in the Southwest \textit{río}/\textit{rito} (the latter {\textless} standard \textit{riíto}) ‘river/brook’ is commonly applied to ordinary permanent or seasonal streams, \textit{arroyo} ‘waterflow’ ({\textless} Latin \textit{arrugia}) usually refers to “a normally dry stream or creek bed” or “wash” (Wozniak, Kemrer \& Carillo 1992: 175; also, Julyan 1998: 22, 291; Studerus 2001: 76).  Today, \textit{arroyo} most often refers to sheer-sided, usually ephemeral or intermittent drainages (\textit{arroyos secos}) incised into their former floodplains (Pearce 1932: 221).  However, the \textit{Río Chaco}’s channel was probably not entrenched until the 1860s, and Lt. James Simpson collected the placename pre-incision, in 1849, from the Hispano guide Caravajal (McNitt 1964: 42–43, 49).  (See also, St. Michaels, AZ, below.)

Pueblo Pintado (Spanish ‘Painted Village’), NM, a P-III Chacoan Anasazi (Ancestral Puebloan) great-house ruin overlooking upper Chaco Wash, and, in modern times, a nearby Navajo settlement:  \textit{Pueblo Grande} ‘Large Pueblo’, according to “Sandoval, the friendly Navajo chief with us” (1849, McNitt 1964: 35; see below) {\textless} Navajo \textit{Kinteel }(\textit{Ch’ínílíní}) ‘(The-Outflow) Wide House’; \textit{kinteel} ‘wide house’ is a generic (Franstead ca. 1979: 17, 46–47; cf. Wide Ruins, AZ, below).  The ruin was also referred to in Spanish as \textit{Pueblo de Ratón} ‘Mouse Pueblo’ (Brugge 1980: 10; Julyan 1998: 277) and \textit{Pueblo Colorado }‘Red  (literally, Colored) Village’ (Van Valkenburgh 1999: 83); \textit{Pueblo Pintado} appears on the 1864 Anderson map and on an 1867 map (Eidenbach 2012: 120, 127).  (On \textit{pueblo} and for another ‘Red House’, see Kinlichee, AZ, above.)

Quartzite Canyon, AZ (former Blue Canyon, a name still applied to a tiny, scattered local Navajo community and to a public road), a short gorge where light-purplish-gray to charcoal-gray Precambrian quartzite has been exposed by a left-bank tributary of Bonito Creek, located not far to the north-northwest of Fort Defiance, AZ, and to the north of Bonito Creek’s \textit{Cañon}[\textit{cito}]\textit{ Bonito} ‘Beautiful [Little] Canyon’ (Hell’s Gate, Bonito Canyon), on the western side of the Defiance Monocline:  \textit{Cañón Azul} ‘Blue Canyon’ (1863, Trafzer 1982: 141) {\textless} \textit{Chézhin Dootł’izhí }‘At Blue Basalt (literally, ‘Black Rock’)'.  Barnes and Granger (1960: 20) considered the toponym Blue Canyon here to be “inappropriate”; however, some shades that would be classed as gray in English are classed as blue in Navajo, and there is no specific term for ‘purple’ in \textit{Diné Bizaad}.  (See also, Gregory 1916: 89.)

Ramah, NM, a Mormon village to the south of Gallup, NM, on the southeastern edge of the Zuni Mountains, in 1878 moved to its present site, near Saboyeta Canyon, from an immediately earlier site three miles away, Savoia, NM (pronounced Say Voya; name attested by Beale 1858: 86; Dutton 1886:map; from 1886 to 1889, the Cebolla Cattle Company operated here; Pine Hill High School Students 1982: 7):  \textit{Cebolla}/\textit{Cebolleta} ‘Onion/Little (i.e., Wild) Onion’ {\textless} Navajo \textit{Tł{}'ohchiní} ‘The (Wild) Onion (literally, The Smell Grass)’ (Franciscan Fathers 1910: 133; Julyan 1998: 284).  The Navajo name moved with the settlement.  (See also, Seboyeta, NM, below.)

Red Lake, NM, a modest-sized water body in monoclinal Red Valley (continuous with north-south-running Black Salt Valley and Black Creek Valley) on the AZ/NM border, which drains into Black Creek (Julyan 1998: 288) just above the latter’s junction with Todildonih Creek (see Black Lake, NM, above):  \textit{Laguna Colorado} (sic) ‘Red Lake’ (1847, Hughes 1848: 63; 1864, Trafzer 1982: 146), \textit{Ciénega Colorada }‘Red Swale’ (1846, 1847, 1851, Correll 1979: 209, 221, 294; 1859, Bailey 1964a:front endsheet; 1966: 95), ‘Red Lake’ (1823, Correll 1979: 133) {\textless} Navajo \textit{Be'ek'}i\textit{(d})\textit{ Halchíí'} ‘Area-Is-Red Lake’ (concerning \textit{ciénega}, see St. Michaels, AZ, below); although in this instance the name is a specific toponym (Van Valkenburgh 1974: 171), the term \textit{be'ek'id halchíí'} is a generic for a characteristically muddy lake (Brugge 1993a:40).  In the nineteenth century, there seems to have been a natural predecessor of today’s dammed Red Lake, into which waters from Black Lake, NM (see above), were diverted in the latter nineteenth century (Haile 1951: 24).  Hughes (1848: 63) put “\textit{Laguna Colorado}” “in sight” of Todilto Park, to the west of which it does lie.  Red Lake is mentioned in the Big Starway myth (collected 1933, Wheelwright \& McAllester 1988: 53; Linford 2000: 122).

Round Rock, AZ, a substantial Wingate Sandstone mesa and a nearby eponymous modern trading-post community:  the eminence was “called by the Mexicans \textit{Piedra Rodia}” (1859, Bailey 1964a:front endsheet, 48-49; also, Linford 2000: 124–125), perhaps \textit{Piedra Rodillo} ‘Roller Rock’ {\textless} Navajo \textit{Tsé Nikání} ‘The Flat-topped Round Rock’ (per Franciscan Fathers 1910: 130; Gregory 1916: 34).  This proposed Spanish version is not an exact translation from the Navajo name, but both these terms do imply cylindricality.  The mesa is, in fact, oblong, not circular in plan. Contrary to the Franciscan Fathers, I detect no `Round' in the \textit{Nikání}, only `sit in a line' (`flat-topped' for Kelley \& Francis 2018: 332). (Oddly, Barnes and Granger [1960: 20] glossed the Navajo name as ‘desk-like rock’, perhaps meaning `disk’.)  The landform does sit atop a flat stripped sandstone surface and can easily be circumambulated.  Although in New Mexican Spanish, final \textit{o}’s and \textit{a}’s are sometimes substituted for one another (Bills \& Vigil 2008), and although certain American accents in English substitute final a’s for final o’s, a possibility other than \textit{Rodillo} remains: \textit{rodilla} means ‘knee’ and could be a reference to a natural arch on the southwestern corner of Round Rock; this feature does somewhat resemble a (the) lower leg(s) and bent knee(s) of a seated human figure.  If this and not \textit{rodillo} is the meaning of the Spanish name, we are not dealing here with a calque.  (\textit{Ventana} ‘window’ appears to be the \textit{nuevomexicano} generic for a natural rock opening, and the pinnacled mesa was also known as both \textit{Las Ventanas} and Walker’s Church.)

St. Michaels, AZ, former La Cienega Ranch and, since 1898, a Franciscan Roman Catholic mission and school:  \textit{Ciénaga Amarilla }‘Yellow Marshy Meadow’ (sometimes, erroneously, as \textit{Cieneguilla}/\textit{Sienneguilla }(\textit{de})\textit{ María} ‘Maria(‘s) Little Marshy Meadow’ [cf. Tierra Amarilla, NM, below]; 1849, McNitt 1964: 107; 1863, Kelly 1970: 53; Trafzer 1982: 80ff) {\textless} Navajo \textit{Ts'í'hootso }‘Meadow Extends Out Horizontally’ (attested 1900, Ostermann 2004a:54).  \textit{Hootso} ‘meadow’ glosses literally as ‘yellow area’, perhaps for the light-straw color of dormant grasses there or perhaps—as McNitt (1964: 107) suggested—for yellow blooms such as wild sunflowers, which sometimes grow in annular stands around the beds of fluctuating or ephemeral lakes and swales.  The grass “dries to a golden shade in the meadows.  There are also a multitude of yellow flowers here following rains....  Flowers at the end of summer here [at St. Michaels] are yellow, hence the name.  The grass also turns yellow in the summer heat” (Barnes \& Granger 1960: 8, 21).  Wilken (1955: 24) also logged a 1908 speculation that the name arose “Probably because of the heavy autumn bloom of golden \textit{Gynnolonia }[sic; \textit{Gymnolomia}] \textit{Multiflora},” the sunflower of meadows \textit{Heliomeris multiflora}.  There is a Cienega Creek here.  \textit{Ciénaga} (Castellano)/\textit{ciénega} (Estremeño) ‘marsh’ refers to often-marshy floodplain and spring-fed meadows (Julyan 1998: 84, 188)—something wetter than a \textit{vega} ‘meadow’ (\textit{prado} ‘meadow’ appears to be rare in Southwestern placenames).  \textit{Ciénega} is sometimes said to be an Americanism derived from \textit{cien aguas} ‘hundred springs’ or from \textit{cieno} ‘mud, slime’; however, García de Diego (1955: 164) traced it to the Latin \textit{coenicum} (cf. \textit{caenum}/\textit{coenum }‘filth, mud, mire’) and defined it as \textit{lugar pantanoso} ‘quag (literally, boggy place)’, and the lexeme is encountered widely in Latin America (Asociación de la Lengua Española 2010: 593).  In New Mexico, it can also mean ‘farm’ (Cobos 1983: 31): in the dry Southwest, floodplains, with their fertile soil and concentrated moisture, were by far the most productive zones for farming.  Before nineteenth-century natural (if overgrazing-induced) channel-trenching of floodplains occurred, many Southwestern drainages consisted of series of grassy meadows, reedy swales, and the occasional small pond (\textit{laguna}/\textit{lagunita}), all connected by ill-defined or braided channels (Bryan 1925: 338–344; Leopold 1951), resulting in the term \textit{ciénega}/\textit{ciénaga} being applied to some watercourses or reaches thereof (for explicit examples, see Black Lake, NM, and Red Lake, AZ, above, and Whiskey Creek, AZ, below).

Seboyeta, NM, a Hispano settlement established circa 1744; a mission to the Navajo was founded in 1748 but was abandoned within a couple of years (Beers 1979: 68; Acrey 1988: 99–100); a village mentioned in many documents over the decades (Correll 1979: passim; Eidenbach 2012: passim):  (\textit{La}) \textit{Cebolleta} ({\textless} standard \textit{Cebollita}) ‘(The) Wild (literally, Little) Onion’ {\textless} Navajo \textit{Tł'ohchin} ‘(Wild) Onion (literally, Smell Grass)’.  Down Cebolleta Creek from Cebolleta there is a hamlet called \textit{Cebolletita} ‘Small Cebolleta’.  Cf. the valleys in the vicinity known as \textit{Cebolla Grande} ‘Big Onion’ and \textit{Cebolla Chiquita} ‘Tiny Onion’, plus the \textit{Sierra }(\textit{de})\textit{ }(\textit{la}) \textit{Zebolleta}/\textit{Cebolleta} ‘Wild-Onion Range’—\textit{Sierra San Mateo}/Mount Taylor—the Cebolleta Mountains, source of the village’s name (Julyan 1998: 70, 329–30).  The toponym is also seen as Ciboletta, Sevoyetta, and other variants (1846, 1853, 1859, Abert 1848: 49; Foreman 1941: 125; Bailey 1964a:101–102; 1966: 75); Robinson (1848: 27) wrote of “Sabriote, or Onion-town.”  These names come from wild onions (\textit{Allium palmeri}) that were once gathered in the mountains (McNitt 1972: 27).  The first mention known to me of the placename \textit{Cebolleta }dates to 1730 (Reeve 1956: 307), earlier than the founding of the village, making this the oldest recognized placename calque from Navajo.  (Cf. Ramah, NM, above.)

Southwestern Range and Sheep Breeding Laboratory (Milk Ranch), NM, of the twentieth century, located four miles to the west of Ft. Wingate, NM:  \textit{Los Carrizos Blancos} ‘The White Reeds’ {\textless} Navajo \textit{Lók’a’ch’égai} ‘White Streak of Reeds Extends Out Horizontally’ (Van Valkenburgh \& Walker 1945: 93; cf. Carrizo Mountains, NM, above).

Teec Nos Pos, AZ, now a Navajo settlement near the Four Corners:  \textit{Bosque Redondo} ‘Round Grove’ (1859, Bailey 1964a:front endsheet) {\textless} Navajo \textit{T'iis Názbąs} ‘Cottonwood Circle’; attested (but mistranslated) in 1910 (Franciscan Fathers 1910: 132).  “\textit{Bosque }is Spanish for ‘forest, woods,’ but in NM the term has been used for dense thickets of trees and underbrush—cottonwoods, [exotic] tamarisk, willows, alders, and others—fringing lakes, rivers, streams, or marshes” (Julyan 1998: 46; confirmed by Studerus 2001: 76).  Cobos (1983: 20) defined \textit{bosque} as “cottonwood grove” fringing a water body.  Downer (ca. 1989: 25) noted that “There was once a lone cottonwood tree here, and it was sacred.”

Tierra Amarilla, NM, a Hispano village in the Chama Valley, near which Puebloans and others gathered a yellow micaceous clay for wall-plastering, etc. (Wozniak, Kemrer \& Carillo 1992: 175–76; Van Valkenburgh 1999: 105): \textit{ }‘Yellow Earth/Land/Ground’ (1845, Correll 1979: 186; 1864, Kelley 1970:map; Bailey 1964b:189; 1870, Murphy 1967: 14, 19; Van Valkenburgh 1999: 105) {\textless} Navajo \textit{Łitsooí}  ‘At Is-Yellow’.  The Tewa Puebloan word “has the same meaning” (Linford 2000: 271; also, Julyan 1998: 352–353), so the Spanish name is more likely a calque from Tewa rather than one from Navajo and the Navajo name a calque from Tewa as well (Harrington 1916: 112-13 gives San Juan \textit{Nąnc’eyiwe }‘At the Yellow Earth’ and Taos \textit{Nąmc’úlito} ‘Yellow Earth’).  The toponym is erroneously rendered \textit{Tierra María} ‘Mary Earth’ on the von Egloffstein map of 1864 (Madsen 2010: map; in Spanish, a final vowel is suppressed and the word elides with the [retained] initial vowel of an immediately-following vowel-initial word; cf. St. Michaels, AZ, above).

Tocito Wash, NM, springs.  In the bed of a Chuska Valley drainage tributary to the Chaco River’s left bank emerge hot-water seeps (Van Valkenburgh 1945: 93), said to have been started by the Holy People with their digging stick (Downer ca. 1989: 31), located to the south of the small Navajo trading-post community of Little Water, NM.  Julyan (1998: 205) asserted that “Little Water” refers to the waterless character of the place, but this is doubtful.  Another idea is that this name derives from that of the place’s Tocito Spring, “Tocito”, in turn, being a semi-calque combining the Navajo noun \textit{tó }‘water’ with the Spanish diminutive suffix \textit{{}-cito}.  However, the Navajo name \textit{Tó ’Álts’íísí} ‘Tiny Water’ pertains to  one of the nearby  warm springs (Downer ca. 1989: 32; Kelley \& Francis 2018:231), so “Little Water” might appear to be a Navajo-derived calque in English.  Nevertheless, note: \textit{Ojo}(\textit{s})\textit{ Caliente}(\textit{s}) ‘Hot/Warm Spring(s)’ {\textless} a generic (1853, Correll 1979: 369; Bailey 1964b:51; 1854, McNitt 1964: 189; 1864, 1866, Kelley 1970: map; Madsen 2010:map; Eidenbach 2012: 120, 123) {\textless} Navajo \textit{Tó Sido} ‘Hot Spring’ (Haile 1950: 112) {\textless} a generic (Julyan 1998: 355).  In New Mexican Spanish, \textit{ojo} ‘eye’ (as opposed to the Iberian \textit{fuente}) also means ‘spring’ (on \textit{tó} and \textit{tooh}, see Hotevilla, AZ, above).  According to Studerus (2001: 76), New Mexican \textit{ojo} comes from Mexican \textit{ojo de agua} ‘water eye’; cf. Arabic \textit{‘ain} ‘eye’, \textit{‘ain mâ} ‘spring (literally, water eye)’, perhaps an allusion to weeping (Steingass 1987: 136, 375, 455).  Note, also, the prominent diatremes Bennett Peak and Ford Butte, the “peaks of \textit{Ojos Calientes},” recorded in 1849 by Lt. James H. Simpson (McNitt 1964: 62–64) {\textless} \textit{Los Cerritos de los Ojos Calientes} ‘The Small Hills of the Hot Springs’ (Van Valkenburgh 1945: 90); Brugge (1965: 17) offered \textit{Cerros de los Ojo}(\textit{s})\textit{ Calientes} ‘Hills of the Hot Spring(s)’ (these may be the hills for which the renowned rug-producing trading-post settlement of Two Gray Hills, NM, was originally named; Winter 2011: 21–23; but see Kelley \& Francis 2018: 419).  (See also, Cabezon Peak, above.)

Tsaile Creek, AZ, which arises in the Chuska Mountains and flows westward into rock-walled Canyon del Muerto, the main branch of Canyon de Chelly:  \textit{Río Cascada} ‘Waterfall River’, perhaps {\textless} Navajo \textit{Tsééhílí} ‘Flows Into Rock [Canyon]’ (Van Valkenburgh \& Walker 1945: 94; there are no real cascades here; 1885, Jett 2001: 119–121, 171).  According to Ostermann (2004a:57; orig., 1901), “This [Tsaile] creek is so called because it flows into the Canyon del Muerto, ... at the entrance to which there is a huge perforated rock, through which the water of the creek flows” (there is no natural bridge here; the very head is currently covered by a 1963 earthen dam).  Note that as an adjective, \textit{cascada} means ‘broken, weak’, so either a feeble or broken-water stream or a cleft rock could conceivably be meant.  Another possibility is that \textit{Río Cascada} was originally applied to Whiskey Creek (below), whose upper reaches contain many cascades.

Tseyi Hatsosi [Narrow Canyon], AZ, a gorge cut into Wingate Sandstone Tyende Mesa to the west of Monument Valley: \textit{Chellecito} ‘Little \textit{Chelly}’ {\textless} Navajo \textit{Tséyi’ Hats’ozí} ‘The Narrow-Area Within-Rock (Canyon)’ (attested 1909; Cummings 1910: 9).

Whiskey Creek, AZ, a stream originating in the Chuska Mountains and joining Wheatfields Creek, which flows westward through Canyon de Chelly:  \textit{Río Negro} ‘Black River’, (\textit{Agua de la}) \textit{Ciénega Negra }‘(Water of the) Black Swale’ (on \textit{ciénega}, see St. Michaels, AZ, above; 1859, Bailey 1964a:front endsheet, 54) {\textless} Navajo \textit{Tó Diłhił }(\textit{Líní})\textit{ }‘(The)-Water-Is-Dark-Colored (Flow)’ (Haile 1951: 318; `Whiskey Creek’, 1917, Weber 2004: 420; 1922, Bodo 1998: 165; \textit{Diłhił} also glosses as ‘causes dizziness’, and \textit{Tó Diłhił} therefore means ‘liquor’ as well as ‘dark water’, after water plants).  This is a different place than \textit{Laguna Negra} ‘Black Lake’ and Black Creek some miles to the south (see above).  Other historical Spanish names include \textit{Río Rayada} ‘Striped River’, for the Sonsela Buttes (see below), and \textit{Río de la Pala Negra} ‘River of the Black Shovel/spade‘, \textit{Pala }in this case presumably being short for \textit{Empalizada} ‘Palisade’, for the nearby Palisades, a cliff of columnar basalt.  An alternative Spanish name for Whiskey Creek was \textit{Río Estrella} ‘Star River’; this is not quite a calque (unless of an unrecorded esoteric name) but is a reference to the nearby \textit{Sierra Rayada}, Spanish ‘Striped Range’, the stratified Sonsela Buttes, Navajo \textit{Sǫ' Silá} ‘Stars Lie’ (attested 1906, Weber 2004: 224–225), associated in myth with a primordial star that existed when Father Sky and Mother Earth touched, as well as with Coyote's scattering the left-over stars into the skies (Salabye 2013, Salabye \& Manolescu 2013; cf. Big Starway, Wheelwright \& McAllester 1988).  Either Whiskey Creek’s tributary Coyote Wash or the saddle between the twin buttes (where Coyote placed four or eight stars that he had reserved) was the \textit{Puerta de las Estrellas}, Spanish ‘Gateway of the Stars’ (Van Valkenburgh 1999 [originally, 1941]:101;  Watson 1964: 11–12; Jett 2001: 120–121; McNitt 1964: 47; Linford 2000: 131–132).  The Fransiscan Fathers (1910: 134) and Gregory (1916: 34) loosely translated the name of the buttes as ‘Twin Stars’.

White Clay, AZ, a zone of the Defiance Plateau just to the south of Canyon de Chelly and to the east of its sizable left-bank tributary Monument Canyon, where there are surface occurrences of white rhyolite tuff, which is gathered for wool-whitening, paint-making, and wild-potato-processing:  \textit{Tierra Blanca} ‘White Earth/Land/Ground’ (1864, Madsen 2010:map) {\textless} Navajo \textit{Dleesh}/\textit{Bis Łigaaí} ‘White Clay/Adobe.  The map shows the location to be in the area of today’s Sand Dunes, not far to the southeast of the mouth of Canyon de Chelly, whereas White Clay is actually twenty or more miles to the east (see Jett 2010: 130).  The Sand Dunes (Navajo \textit{Séí Náhoojíshí} ‘At Sand Rolls over and Over’) are orangish in color, not white, and of sand—\textit{séí}—not earth, so it is unlikely that they are being referred to; too, the dunes may not have great age; Jett 2001: 49, 130.)

White House Ruin, a P-III Anasazi (Ancestral Puebloan) cliff-dwelling in Canyon de Chelly, AZ, the exterior wall of one room of which is white-plastered:  \textit{Casa Blanca} ‘White House’ {\textless} Navajo \textit{Kiníí Na’ígai} ‘There-Is-a-White-Strip-across-the-Middle House’ (attested for 1882, Jett 2001: 75; Linford 2000: 47).  According to Bailey (1964a:41), the name \textit{Casa Blanca} was first applied in 1893 by the Bureau of American Ethnology archaeologist Cosmos Mindeleff, following fieldwork a decade previous.  What Mindeleff (1897: 104) in fact stated was that the site had been noted by Lt. James Simpson in 1849 “and subsequently called Casa Blanca”; Simpson himself did not record a name for the site (McNitt 1964: 8).  “Capt. George M. Wheeler called it White House in 1873” (Barnes \& Granger 1960: 319).\footnote{Wheeler did not visit Canyon de Chelly, but his photographer Timothy H. O’Sullivan did, in 1873 (Jurovics et al. 2010).}  This sacred ruin, a home of Holy People, has Upward Reachingway, Blessingway, Nightway, Big Godway, Beautyway, Beadway, Mothway, and Excessway associations (Klah 1942: 111; Matthews 1902: 220–24; 1907: 244–245; Wheelwright 1938: 6–9; Wyman 1970: 427; 1957: 39, 110, 112, 139; Levy 1998: 99; Haile 1978: 61–62, 87; Jett 2001: 75; 2006).

Wide Ruins, AZ, a P-IV Anasazi (Ancestral Puebloan) ruin and an adjacent twentieth-century, trading-post site:  \textit{Pueblo Grande }‘Large Village’ (1859, Bailey 1964a:63; 1860, Van Valkenburgh 1999: 115) {\textless} Navajo \textit{Kinteel} ‘Wide House, i.e., Pueblo’ (Haile 1950: 209) {\textless} a generic (cf. Pueblo Pintado, NM, above; on \textit{pueblo}, see Kinlichee, AZ, above).  Also known in Spanish as \textit{Cuña} ‘Wedge, Quoin’ and in English as Butterfly Ruin, after the shape of its plan (Van Valkenburgh 1999: 115; Van Valkenburgh \& Walker 1945: 94; Linford 2000: 146–147).

\section{Spanish/Navajo Frontier Interaction}\label{jett:sec:5}

Beginning in 1598, Spaniards from Central Mexico (part of New Spain), with their Mexican-Indian allies, conquered and colonized the upper Rio Grande (\textit{Río del Norte} ‘River of the North’) region,\footnote{In pre-American-period times, what is today’s upper and middle Rio Grande was called in Spanish the \textit{Río del Norte}, and the Rio Grande below Chihuahua’s \textit{Río Conchos} was referred to as the \textit{Río Bravo} ‘Wild/Rough River’; the appellation “Rio Grande” seems to have been of Texican coinage.  Today’s San Juan River, major left-bank tributary of the Colorado River, was the \textit{Río Grande }(\textit{de Nabajó }[\textit{del Norte}]) ‘Big (Navajo) River ([of the North])’ (Briggs 1976: 51; Julyan 1998: 316, 333).} gradually expanding their population and their area of occupance during subsequent decades, engaging in farming and livestock-raising.  Naturally, in the process they encountered indigenous groups, including Apacheans, who offered greater or lesser degrees of opposition and/or accommodation (see Forbes 1960; Spicer 1962; Kessell 2002; Acrey 1988; Beck \& Haase 1969).\footnote{Although Spanish mentions of Apacheans predate it, the first documented use of “Navajo” is from 1626, when \textit{Apaches de Nabahú} were mentioned in a report by Father Jerónimo Zárate Salmerón (1966), a Franciscan friar at Jémez Pueblo, \textit{nabahú} corresponding to a Tewa Puebloan word for ‘broad cultivated fields or valley fields’, or to ‘to take from the fields’ (Van Valkenburgh 1937; Spicer 1962: 211).  These people were living in the \textit{Río Chama} watershed.  Proto-Navajos, at least those proximate to Jémez, NM, had previously been referred to as Cocoyes (Wozniak, Kemrer \& Carillo 1992: 5; but see Schaafsma 2002: 221–22).}  In light of the proximity of culturally proto-Navajo people to the Spanish/Mexican settlements of the greater \textit{Río Abajo} (‘Down-River’, from Santa Fé) country of the Middle Rio Grande Valley and in light of the historically-documented interactions between the two societies, it is not surprising that direct borrowings and loan-translations took place.  But we can add much detail as to how acquisition of such borrowings likely occurred and are also able take note of, and account for, the assymetricality of the process.

\subsection{Slaving and Bilingualism}
Both territorial conflict and peaceful trade took place between Hispanos (Navajo: \textit{Naakaii}) and the People (Spanish: \textit{Apaches de Nabajó}), as well as livestock theft and slave-taking on both sides, the last providing particularly notable potential for two-way cultural exchange.  Between A.D. 1694 and 1875, over 1,600 Apachean captives designated “Navajo” were baptized in New Mexico.  Although numbers of conversions between 1720 and 1770 were voluntary, subsequently most were accomplished under duress of captivity.  The trade in Navajo slaves first became important during the interval 1804–1832, and—despite newly-independent Mexico’s abolition of slavery in 1821—the 1820s were a period of particular focus on the \textit{Dine’é} among the (mobile non-Puebloan) American Indian societies targeted by slavers (Bailey 1966: 71–137; Brugge 1985: 23–24, 38).  New Mexicans rationalized their perpetuation of illegal servitude by first forcibly baptizing their Indian captives and then rejecting the idea that such “Christian” individuals should be permitted to rejoin—and risk being tainted by—their pagan peers (Acrey 1988: 144).

The institution of slavery may have diffused from the Spanish to the Navajo; the latter also came to keep slaves of other Amerind ethnicities.  On neither the Navajo nor the \textit{mexicano }side was servitude either absolute or hereditary.  Slaves could marry non-slaves and in some cases became influential members of their adoptive societies; children of slaves were free members of the host societies.  It was only in 1872 that large numbers of Hispano-held Navajos were freed and returned home (Moore 1994: 100), and some Hispano slaves were retained by Navajos into the early twentieth century (Franciscan Fathers 1910: 259).  Many Hispano captives became entirely acculturated to Navajo ways and did not care to be repatriated (Brugge 1985: 144, 163; also, Swadesh 1979).

The principal New Mexican slave-procuring settlements—and recipients of reciprocal raids (Bailey 1964b:23)—were Cebolleta and close-by Cubero (established 1833), plus Ábiquiu on the \textit{Río Chama} (see below) and, to a lesser extent, the missionized pueblo of Jémez and nearby Hispano San Ysidro—all of these being frontier settlements to the west of the Rio Grande Valley proper.  Acquisition of slaves was accomplished both by direct raiding and by \textit{traficantes} trading arms, ammunition, and liquor for captives taken by members of other mobile “tribes” (Brooks 2002).

Cecil H. Brown (1994) has argued logically that bilingualism is the most important factor in encouraging lexical loans.  In 1795, Governor Fernando Chacón observed that the Navajo were “more inclined to speak Spanish than any other heathen [i.e., non-Puebloan] tribe” (Reeve 1971a:104).  Some captives, in particular, did ultimately become significantly bilingual, giving them certain advantages and fostering intercultural transfer (Brugge 1985: 133, 165).  Escaped, released, or exchanged Navajo-held Spanish/Mexican slaves who had learned some Navajo plus some Navajo placenames, carried back to the Hispano settlements onamastic information as well as geographical knowledge of the territories in which they had been held captive and across which they had traveled from and to home.\footnote{To New Mexico Hispanos, a captive was a \textit{cautivo}; a household slave was a \textit{criada/o} ‘servant’; slaves were often referred to by the euphemism \textit{pieza} ‘piece, bit (as an eighth part of a \textit{real}/\textit{peso}/\textit{pedazo de ocho}/Spanish dollar)’.  A livestock-raider was dubbed a \textit{ladrón} ‘thief’; in \textit{Diné Bizaad}, a servant was a \textit{naalté} (Brooks 2002: 374). Although whether or not they were former slaves is unclear, we know a few mid-nineteenth-century Hispano guides to the Navajo Country by name.  One was “old Savidra” (Saavedra), on Lt. Edward Fitzgerald Beale’s 1857 expedition (Stacey 1929: 87).  Another, in 1859, was Capt. Blas Lucero of Albuquerque, an associate of Sandoval, headman of the Cebolleta Navajo band (Correll 1970: 29).  Another was Fernando Aragón of Cubero, NM, a trader to the Indians, slaver, and livestock rustler (Bailey 1964a:40, 101).  The Native governor of Jémez Pueblo, Francisco Hosta, accompanied Col. John M. Washington’s 1849 Navajo expedition, along with Sandoval and the Hispano guide “Carravahal” (Caravajal) of San Ysidro, NM (McNitt 1964: 25).}  One explicit example was reported in 1853 by Lt. A. W. Whipple, who was reconnoitering for an envisioned transcontinental railroad:

\begin{quote}
	... one of our Mexican herders from Covero [Cubero, NM,] understands their [the Navajo’s] language, a vocabulary of it has been obtained from him.  A few years since, while he was playing at Covero spring, he was captured by Navajoes.  For nine months he was a prisoner, and followed the Indians in their wanderings.  He accompanied a party of one thousand warriors through the Moqui [Hopi] country, and afterwards spent much time among their [Navajos’] rancherias [homesteads] in the famous Cañon de Chelly.  (Foreman 1941: 158)
\end{quote}

Similarly, the Cebolletaño C. C. Marino (1954: 15–16) mentioned a local Hispano boy, Juan Ortiz, whom Navajos had captured at age 12 but who had escaped and returned home when he was 21.  Half-Zia, had Spanish Juana Hurtado was a captive of the Navajo from 1690 to 1692.  After being ransomed, she used her intercultural experience “to make trading contacts that allowed her to acquire a substantial amount of property” in Ábiquiu (Ebright \& Hendricks 2006: 30, 33-34).

Some non-Navajo Indian and half-breed servants are known to have fled to the Navajo from difficult masters or from other troublous circumstances in Hispano country, sometimes later returning voluntarily or under duress, again resulting in knowledge-transfer (Brugge 1993b).  Too, the Spaniards recognized the category “\textit{genízaro }(lit., ‘Janissary’; detribalized nomadic Indians reduced to slavery, converted, resettled in Spanish homes or villages and deployed as military auxiliaries”—“coerced cultural mediators,” as the historian James Brooks (2002: 240, 374) termed them and other captives.  Their progeny were free persons and they tended to congregate together; the \textit{Río Chama} settlement of Ábiquiu was populated largely by such individuals (Kessell 2013: 82–83).  \textit{Genízaros} were mostly descendants of abducted Plains females, but—although the term came to be dropped—many Navajo women and children were added to these mixed-origin but acculturated social segments during the Mexican period (Chavez 1979).  A certain number of males among such individuals became guides.

Brooks observed,

\begin{quote}

... residence of Navajo kinspeople in outlying New Mexican and Pueblo villages provided linguistic and cultural facilitators for trading visits.\\
...\\
Navajo boys held captive in New Mexican villages sometimes served as guides for both trading and raiding expeditions into Navajo country.  Cebolletaños had an especially favored captive, raised from boyhood and named Kico [possibly, \textit{Kii Kǫ’} ‘Fire Boy’; mentioned by Marino 1954: 24], who guided their sojourns and forays into those hinterlands, where his knowledge of landscape proved even more valuable than his language skills.  By the nineteenth century, Navajo had become something of a lingua franca in the borderlands, used by New Mexicans and Puebloans alike.  (Brooks 2002: 240–241; see also, Briggs 1976: 19, 154–155; Underhill 1956: 65)
\end{quote}
\subsection{Cebolleta, NM, and The ‘Enemy Navajo’}
The Navajo/Spanish borderland, to the west of the \textit{Río}, was a fluctuating one, in the \textit{Río Puerco} of the East/San Mateo Mountains/Zuni Mountains zone.  Spanish settlements were established and then, occasionally, abandoned owing to Navajo depredations, sometimes to be reestablished, sometimes not.  Too, nonviolent Navajo/Spanish jockeying for land-use rights in the vicinity of settlements took place.  During the first half of the eighteenth century, when Ute raids were stimulating westward and southward emigrations of Navajos from their ancestral \textit{Dinétah}, Colonial authorities invited Navajos to join their cohorts in the Cebolleta area and to be baptized (Reeve 1971a; Acrey 1988 passim).

  The Hispano village known today as Seboyeta, NM, established about 1744 (see above), lies some miles to the west of the present Tohajiileeh (Canoncito) Navajo Reservation, in the drainage of the \textit{Río San José} (\textit{Río Cubero}), a right-bank tributary of the \textit{Puerco}.  The area appears to have been, up to the village’s founding, a staging area for Navajo incursions into the Rio Grande country (Acrey 1988: 98, 100, 108).  Then the tables were turned on the Navajo: “the town ...served as the jumping-off point for all Spanish encroachments against the [Navajo] tribe” (Bailey 1966: 77), and for many decades it was a focus of legal and illegal trading with the Indians.  In 1846, Robinson (1848: 27) estimated Cebolleta’s population to number about 500.  From 1849 to 1851, the U.S. Army maintained a post there in an unsuccessful attempt to halt the illicit trade (Bailey 1964b:24–25).  Thus, a close if not-always-amicable association between local Navajos and Euroamericans long prevailed in the vicinity.

  During this era, Navajos were organized into local bands (Spanish: \textit{parcialidades}), each with a headman who led by consensus (Young 1961: 371; Shepardson 1963: 50–51).  One band—that of the greater Mount Taylor region and involving perhaps 300 to 400 souls in the mid-1800s, living mostly in the watershed of the \textit{Río Puerco} of the East—played a pivotal role in Navajo/Euroamerican relations.  Known to the other Navajos as \textit{’Ana’í Dine’é} ‘Enemy/Alien Navajos’ and to ethnohistory as the Cebolleta band, they seem to have included a Puebloan element (likely, Jémez; possibly, Ácoma and Laguna as well), been exposed to and to some extent accepted Christianity (Wilken 1955: 7, 14-15), learned Spanish as a second language, and frequently cooperated with the European conquerors and occupiers and their descendants—first Spanish, then Mexican, and then American.  They acted as spies, scouts, guides, interpreters, and diplomats during reconnaissance, trading, slaving, livestock-raiding, negotiatory, and punitive military expeditions to or against other Navajo bands.

This cooperative posture vis-à-vis the Euroamericans reflected the group’s vulnerability, being the Navajo band nearest to the Hispano settlements and to hostile Plains peoples and some Apaches (Brugge 1985: 141).  In 1788, the band’s originally uncooperative headman Antonio el Pinto (\textit{Hashke’ Likízhí} ‘Speckled Warrior’) made peace with the Spanish (Carey 2014; Van Valkenburgh 1999: 18).  A later headman, Segundo, also endeavored to maintain peace, offering warriors to help the Spanish attack non-Navajo Apache raiders.  In 1818, an early-nineteenth-century successor known as Joaquín cooperated in the recovery of stock stolen from Jémez Puebloans by other Navajos and also informed the Spanish of an impending Navajo uprising (McNitt 1972: 48–50; Correll 1979: 14; Acrey 1988: 117, 121).  The governor appointed Joaquín as spokesman for all the Navajo and asked him to talk the rest of that nation into abandoning their plans for war.  However, being disrespected as a collaborator and failing to persuade—and sensing that the Navajo would lose in any conflict—Joaquín decided to remove his followers to nearer Jémez Pueblo and then close to Cebolleta and Laguna Pueblo and to ally them with the Spaniards, agreeing to help attack other Navajos and signing a treaty in 1823 (Reeve 1971b:227; Julyan 1998: 60; Acrey 1988: 121, 134).  Joaquín seems, then, to have been the first leader of the then-circa-200-person band to not only participate in a standing peace with the Hispanos but also to cooperate in campaigning against other Navajos, his followers thus becoming the \textit{’Ana’i Dine’é}.  The group’s alliance with the relatively powerful New Mexican people and polity as well as its new greater geographic proximity were also in face of common enemies such as Utes, Comanches, and non-Navajo Apaches, and this entente afforded immunity from attack by the colonists while at the same time facilitating access to Hispano and Puebloan trade goods (Trafzer 1982: 12–13).

  Once having engendered hostile relations with their fellow Navajos, the Cebolleta band became increasingly dependent on its Euroamerican connections; and at least in 1858, the group was rewarded by receiving its portion of the treaty-mandated annuity intended for all Navajos but withheld from every other band owing to ongoing war (Correll 1970: 30).

In 1825, Gov. Antonio Narbona recognized the half-Navajo former political governor of New Mexico (1823-1825), Bartolomé Baca (ca. 1767-1834), as leader of the Cebolleta Band.  However, friction arose between Baca and the Band’s Antonio Cebolla [‘Onion’] Sandoval (Navajo \textit{Tlo’chiin} ‘Onion’; 1807–1859).  Following Navajo raids on his ranch and ceding his lands to the incoming founders of Cubero, NM, in 1833 Baca departed and Sandoval became paramount in the context of the cooperation with the Mexicans (Acrey 1988: 145–46, 150–51, 153).  The principal account of Col. Alexander William Doniphan’s 1846 Mexican-American War military expedition against the Navajo following the bloodless conquest of New Mexico speaks of their guide “Sandoval, a noted chief of one of the Navajo’s cantons, who had a friendly intercourse with the New Mexicans on the frontier.... whose geographical knowledge was extensive and minute....  [T]he New Mexicans have but very limited knowledge of that [Tcheusca] mountain country, never departing far from their settlements, through fear of the Indians” (Hughes 1848: 63–64).

In 1846, the core of Sandoval’s band resided in an intermontane valley to the west of Hispano Cebolleta, NM, maintaining sheep camps to the west of that.  The headman was reportedly rich (a \textit{rico}), boasting some 5,000 sheep and 100 horses (Robinson 1848: 28–29).  Sandoval’s Navajos initially continued to reside near Cebolleta, as well as near Jémez Pueblo in an adjacent drainage, but later removed themselves to the safer area of the present-day Tohajiileeh Reservation, to the east of Cebolleta and likewise in the \textit{Puerco} watershed.  By 1845, like Joaquín before him Sandoval was actively cooperating (although occasionally somewhat treacherously) with the Euroamerican authorities (Correll 1970), not infrequently as a guide.  Sandoval’s forces were sometimes augmented by Navajos from around proximate Cubero, NM, “that old last outpost for raids and counter-raids on the Navajo frontier” (Richard H. Dillon, in Rice 1970: 13).  Sandoval was a vigorous participant in—even initiator of—slaving forays against mainline Navajos (Bailey 1964b:26).

  In some instances, too—e.g., Capt. J. G. Walker’s 1859 expedition to Canyon de Chelly (Bailey 1964a:38)—local Navajo headmen encountered en route were recruited as involuntary guides and no doubt provided to their taskmasters Navajo placenames for translation into Spanish.  The earliest mention I have found of Spaniards recruiting (voluntary) Apachean guides—probably proto-Navajos from the Cebolleta, NM, area—dates to 1599, undertaken by official explorer Vicente de Zaldívar (Reeve 1956: 306).

  One unanswered question is to what extent, if any, calques of Navajo placenames entered Spanish via translation from Puebloan-language calques, since at least in earlier Colonial times Puebloans—especially, from Jémez—would have been the usual native guides utilized (with Sandoval, one Jémez accompanied Col. John M. Washington in 1849; see note 13).  Another query is to what extent, if any, English-ignorant but Spanish-cognizant Navajos gave Anglo-Americans Spanish-language translations of Navajo toponyms without these names ever truly entering Spanish itself.

\section{Toponymic Borrowings from Spanish and Other Tongues into Navajo}\label{jett:sec:6}
Although placename (like lexical) borrowing from Spanish into Navajo appears to have been rare, one exception is \textit{Gaawasóón} (see entry for Cabezon Peak, above), substituting a /w/ for the Spanish voiced bilabial fricative, which Navajo lacks.  Another is \textit{Hwééldi }{\textless} Spanish \textit{Fuerte }‘Fort’ (/hw/ substituting for /f/, which does not exist in Navajo, and\textit{ }/l/ substituting for /r/, which is also absent; attested by The Franciscan Fathers 1910: 132): Fort Sumner, NM, where from 1864 to 1868 the U.S. government kept most Navajos interned on eastern New Mexico’s Bosque Redondo Reservation (Bailey 1964b).  An additional example is \textit{Sokwolah} for Socorro, NM (Van Valkenburgh 1999: 100).

Van Valkenburgh (1999: 23) asserted that Navajo also had adopted the Tewa \textit{Ts’á’mah} ‘Wrestling Place’ as \textit{Tchamah} ‘Chama’, NM (a former Native pueblo).  However, according to Young and Morgan (1980: 742) the Navajo appellation is \textit{Ts’í’mah} ‘Oof!’—still probably inspired by the Tewa \textit{Ts’á’mah} or the Spanish \textit{Chama} (Linford 2000: 191).  (The Tewa toponym has been thought, alternatively, to be \textit{Tzama} ‘Red’, for the color of the river; Pearce 1965: 31; Julyan 1998: 75).  \textit{Rio Sama} and \textit{Sama} appear on a 1602 map, and the placename \textit{Tzooma} is recorded for 1598; \textit{Rio Chama} occurs on 1771 and 1778 maps (Kessell 2013: 39; Eidenbach 2012: 14-17, 45).

Although most Navajo names for individual indigenous pueblos are descriptive, the Western Keres name of Ácoma Pueblo, \textit{’Áakuu’m’e} has entered Navajo as \textit{Haak’oh}.  The Hopi Pueblo Oraibi (\textit{Örayvi} ‘Rock on High’) became \textit{’Oozéí} in Navajo.  The early name of Bluff, UT—Bluff City (founded 1880; van Cott 1990: 44)—came into Navajo as \textit{La Síti}, there being no /b/ before /l/ and no /f/ in Navajo.  The \textit{’ayání} ‘bison’ in the twentieth-century name  Iyanbito, NM, Navajo \textit{’Ayání Bito’}  ‘Bison’s Spring’, may relate to the use of reflexes of \textit{yanasa} ‘bison’, found in a number of Southeastern American Indian languages (see Haas 1978: 33).

The Navajo \textit{Béégashi Bitó} (Cow Springs, AZ) mirrors the Hopi \textit{Wacasva }‘Cow Spring’, and the Navajo\textit{ Bikooh Dotl'izh} ‘Blue Drainage/Canyon’ is comparable to the Hopi \textit{Sakwatupqa} ‘Blue Canyon’, for Blue Canyon, AZ, to the west of Hopi (in English, the color would be classed as gray).

A possible UA calque in Navajo (and Paiute, Hopi, or Navajo calque in Spanish) is the name for Arizona’s enormous, spectacular, and ritually significant Grand Canyon (Spanish: \textit{Gran Cañón}): the esoteric  \textit{Tsékooh Hatsoh} (Navajo, ‘Big-Area Rock Drainage/Canyon’)\footnote{The esoteric sacred or alternative secular name is \textit{Tséłchii Bikooh} `Red Rock's Canyon' (Annerino 2017: 67).} {\textless} \textit{Piapaxa ’Uipi} (Southern Paiute ‘Big-River Canyon’) or {\textless} \textit{Suukotupqa} (Hopi ‘Big Canyon’); however, these descriptive names could have been independently coined.  Another possible instance of borrowing has to do with the imposing and sacred volcanic San Francisco Peaks, just to the north of Flagstaff, AZ: \textit{Dook'o'ooslííd} (Na-Denéan Navajo ‘It Has Never Melted and Run off from Its Summit’); cf. \textit{Nuvatukya'ovi }(UA Hopi ‘Snow-Piled-on-Summit Place’, Humphreys Peak), \textit{Nuvü’ Hatuh} (UA Southern Paiute ‘Snow Sitting’), \textit{Wikagana Pa’dja} (Yuman Havasupai ‘Snowy Mountain’), and \textit{Wik’ Hanbaja} (Yuman Hualapai ‘Snowy Mountain’).  The Peaks are strikingly snow-mantled in winter and spring although today snow-free in summer and earlier autumn (snow likely began earlier and persisted later, especially on high north-facing slopes, during the Little Ice Age, which ended about 1850).

\section{Discussion and Conclusions}
\subsection{Wherefore Asymmetry?}
As has become evident, both direct borrowing and loan-translation of placenames between the Navajo language and New Mexican Traditional Spanish speech were essentially one-way processes.  Only a few instances of Spanish-to-Navajo direct toponymic borrowing or calque-acquisition can be cited, even in the cases of Spanish-established, central places (for example, the Spanish-founded New Mexican towns of Santa Fé, Albuquerque, Bernalillo, Ábiquiu, Cubero, La Jara, and Cuba, all carry Navajo descriptive names), whereas New Mexican Traditional Spanish adopted several Navajo placenames directly plus dozens by translation.

As has been noted, the Navajo language is resistant to borrowing in any form, consistent with general Athabaskan linguistic practice.  I have not investigated globally what Spanish’s general tendency in this regard may be outside of the U.S. Southwest, but my impression is that—despite its status outside Spain as a language of conquerors and perhaps owing to massive Spanish/Native American \textit{mestizaje} ‘miscigenation’—as a tongue it has been more receptive to both direct borrowings (e.g., Nahuatlisms in Mexico) and loan-translations, although not unusually so in the case of the latter.  Still, this seeming asymmetry in linguistic leanings between Navajo and Spanish appears not to account entirely for the striking one-sidedness with respect to these languages in the Four Corners region in terms of quantities of direct onamastic borrowings and of calques.  History seems to supply the remainder of the explanation.

  The proto-Navajo people had had at least a couple of centuries’ occupation of \textit{Dinétah} and also resided in the \textit{Río Chama} drainage and in the vicinities of the Ácoma, NM, and Awatobi, AZ, pueblos, and near Jémez Pueblo and elsewhere, at the time of the arrival of the Spanish conquistadors (Brugge 2006: 46; but see Schaafsma 2002: 233–234), and the \textit{Dine’é} would have had considerably wider geographic knowledge and placename assignment than just the areas of their occupance.  Consequently, culturally-proto-Navajo Athabaskan placenames would have been well and quite widely established by 1598, and the People would have had little or no reason to substitute novel toponyms applied by the Spaniards to places already possessing labels in \textit{Diné} \textit{Bizaad}.  By the same token, the Spaniards would have been offered, in Navajoland and environs, a pre-existing suite of names to adopt, one way or another, if they so chose.

  Whereas Navajos came to raid the Hispano settlements for livestock and slaves, the Spanish and (from 1821 to 1846) their successors the Mexicans seem to have, more often and more systematically, mounted official and unofficial expeditions into the Navajo Country with the same goals—as well as to negotiate peace or to mete out punishment—using, among other guides, interpreters, and auxiliaries, both enslaved Navajos and escaped or freed former Spanish captives of the Navajo.

  Unlike the dispersed Navajo, who had neither central leadership nor central places and who lived in dispersed \textit{rancherías} under informally-elected local-band headmen (Jett 1978), the better-armed and -equipped Hispanos not only possessed towns and administrative organization, including military, but, in the case of certain of the inhabitants of a few frontier villages, became specialists in mutualistic trade and in parasitic and predatory forays, fielding unauthorized expeditions as well as participating in official ones.

There is no record as to the extent to which Navajos’ Hispano captives may have been pressed into service as guides for \textit{Diné} sorties or as informants regarding Spanish toponyms, but we \textit{are} aware that not only did Hispanos who had escaped from servitude among the Navajo provide information upon returning home but also that members of the Cebolleta band of Navajos were routinely recruited as willing guides into the Navajo Country as well as as fellow raiders against other Navajos.  These cooperative operations afforded major occasions for Spanish adoption, directly or in translation, of Navajo names of places passed en route to, and at, targets, as well as of prominent landmarks seen at a distance.  Note, too, that Hispano expeditions, succeeded by Anglo-American ones, regularly penetrated much more deeply into \textit{Diné Bikeyah} than Navajo ones did into Hispano-occupied territory—which in any case was largely linear, along the Puebloan Rio Grande Valley—a zone seemingly free of Navajo-derived toponymic loan-translations.

\subsection{Geographic Distribution of Loans}
The geographic distribution of known apparently-Navajo-derived toponymic calques or quasi-calques in Spanish is not random throughout the Navajo Country.  Of the 30+ positively or tentatively identified here, seven are in the Chinle Wash (Canyon de Chelly) drainage and six in the Black Creek drainage, the latter containing the travel corridor connecting the headwaters of upper Chinle Wash’s right-bank tributaries with Fort Defiance, AZ, the main latter-nineteenth-century Navajo-oriented U.S. base (Frink 1968)—together, the major zone of military presence and operations of the time.  Kinlichee, AZ, in the Pueblo Colorado drainage, is on the main route from the Fort to Canyon de Chelly’s mouth and to Hopiland.  Three calques are within the watershed of the \textit{Río Puerco} of the East, habitat of the Cebolleta band of Navajos and whose right-bank tributary watershed the \textit{Río San José} (Spanish: ‘St. Joseph River’; also, \textit{Río Cubero} ‘Cubero River; \textit{Río de Gallo} ‘Tom Turkey, literally Rooster, River’) is the setting for the Hispano raiding capitals of Seboyeta, NM, and Cubero, NM.  Three of the seven direct-loan toponyms occur in the Rio Puerco, Chinle Wash, and Black Creek areas; in addition, two that represent sections of the Chuska range just to the east of Black Creek Valley and Canyon de Chelly are largely in the Chinle Wash watershed.  Another direct loan is Chaco Canyon, where there are also two calques.  Chuska Valley, lying to the east of, and watered by streams from, the Chuska Mountains that are tributary to Chaco Wash or to the San Juan River, had been a major maize-producing zone since prehistoric times and came to support a sizable population of Navajos, and there are two possible calques there.

  The San Ysidro-Chacolí-Cabezon-Chaco Canyon-Chuska Valley-Narbona (Washington) Pass travel corridor from the capital to the head of Canyon de Chelly (Reeve 1971a:106; Dutton 1886: 175), accounts for two direct loans and four toponymic calques in New Mexico’s San Juan Basin alone.

\subsection{Additional Considerations}
A couple of the postulated calques may be incorrect, owing to ambiguities and/or incomplete information.  Too, being descriptive, there is some chance that one or more of the name sets in the two languages involved independent coinage.  However, this cannot be the case with names like those for Agathla Peak, Blue Canyon, Cabezon Peak, Carrizo Mountains, Dead Mans Wash, Kinlichee, Los Gigantes Buttes, St. Michaels, and Whiskey Creek—names depending on arbitrary Navajo perceptions that, indeed, signal direction of transfer. In a number of other instances, the Navajo names refer to less-than-striking local features that were not also likely to have caught Spaniards' eyes as objects for placename coinage.

  The historical study of Navajo placenames has an advantage over that of toponyms in other Athabaskan languages, owing to the time-depth of literate, record-keeping European colonization in New Mexico as well as the early and massive collecting of ethnographic information.  The attestation of a number of Navajo direct loans and calques as early as 1796—1776 and 1779 in three cases and even 1730 and 1774 in three other instances—provides at least modest support for the notion of the temporal persistence of Navajo placenames.  Further, in the cases of toponyms that allude to mythic events, these attestations establish \textit{ante-quem} dates for the existence of the relevant Navajo ceremonial stories (cf. Brugge 1973).

  An anonymous reviewer of this paper insightfully pointed out that since the Navajo have imposed a virtually wholely-Athabaskan placename network on their relatively-recently-occupied territories in the Southwest, the lack of a substrate of non-Athabaskan placenames in Dené-occupied areas of the North does not necessarily attest to Athabaskans’ having been the first to occupy their present territories or to their having been in place for millennia.  We must keep in mind, however, that Puebloans had abandoned most of today’s Navajo Country by the time that \textit{Diné Bizaad}{}-speakers established themselves there—possibly but not certainly owing, at least in part, to earlier Apachean depradations (Jett 1964; Apacheans seem to be evidenced in southeastern Arizona as early as the 1300s; Seymour 2013: 169).  Other than in the “island” of Hopiland, after A.D. 1300 only in the far west of \textit{Diné Bikéyah} was there even a meager already-present population, of Southern Paiutes and Havasupais (the former became Navajo-acculturated and the latter withdrew).  Thus, the opportunity for Navajo adoption of pre-existing placenames in the areas of their occupance was quite limited, even had these Athabaskans been inclined to adopt.  In the event, they quickly created and (under Puebloan influence\footnote{Note that many elements of the tale of the Hero Twins, formerly supposed to owe its beginning to Puebloan traditions, now appear to have been brought in from the North (Wilson 2015).}) mythically accounted for a new toponymic network.


\vspace{2cm}


\noindent
\textbf{Dedication.}  This paper is dedicated with admiration and affection to James Kari, as well as to the memories of Jim’s University of New Mexico teacher Robert W. Young and U.S. National Park Service Navajo scholar David M. Brugge.

\refheading

\begin{hang}
	Abert, J. W. 1848. Report of the Secretary of War Communicating, in Answer to a Resolution of the Senate, a Report and Map of the Examination of New Mexico, Made by Lieutenant J. W. Abert, of the Topographical Corps.  30\textsuperscript{th} Congress, 1\textsuperscript{st} Session, Executive [Document] 23. Washington.  [facsimile reprint, Lincoln County Heritage Trust, Lincoln, NM]

	Acrey, Bill P.  1988.  \textit{Navajo History to 1846: The Land and the People}.  Shiprock, NM:  Department of Curriculum Materials Development, Central Consolidated School District No. 22.

	Annerino, John. 2017. \textit{Hiking Grand Canyon: A Detailed Guide to More Than 100 Trails}. New York: Skyhorse Publishing.

	Asociación de Academias de la Lengua Española 2010.  \textit{Diccionario de americanismos}.  Lima:  Santillana Ediciones Generales.

	Automobile Club of Southern California. n.d.  \textit{Indian Country Guide Map: Arizona, Colorado, New Mexico,  Utah}.  Los Angeles:  Automobile Club of Southern California.  [periodically updated]

	Bailey, Lynn R.  1964a.  \textit{The Navajo Reconnaissance: A Military Exploration of the Navajo Country in 1859 by Capt. J. G. Walker and Maj. O. L. Shepherd}.  Los Angeles:  Westernlore Press.

	Bailey, Lynn R. 1964b.  \textit{The Long Walk: A History of the Navajo Wars, 1846--1868}.  Great West and Indian Series, 64.  Los Angeles:  Westernlore Press.

	Bailey, Lynn R. 1966.  \textit{Indian Slave Trade in the Southwest: A Study of Slavetaking and the Traffic in Indian Captives}.  Los Angeles:  Westernlore Press.

	Barnes, Will C. \& Byrd H. Granger. 1960.  \textit{Arizona Place Names}.  Tucson:  The University of Arizona Press.

	Beale, E. F. 1858. \textit{ Wagon Road from Fort Defiance to the Colorado River}.  35th Congress, 1st Session, House of Representatives Ex. Doc. 124.  Washington.

	Beck, Warren A. \& Ynez D. Haase. 1969.  \textit{Historical Atlas of New Mexico.}  Norman:  University of Oklahoma Press.

	Beers, Henry Putney. 1979.  \textit{Spanish and Mexican Records of the American Southwest.}  Tucson:  The University of Arizona Press, with the Tucson Corral of the Westerners.

	Begay, Robert L. 2015a.  Kinyaa’áanii Clan History.  \textit{Leading the Way: The Wisdom of the Navajo People} 13(4). 2--4.

	Begay, Robert L. 2015b.  Creation of the Owl and Eagle.  \textit{Leading the Way: The Wisdom of the Navajo People} 13(7). 2--3.

	Bills, Garland D. \& Neddy A. Vigil. 2008.  \textit{The Spanish Language of New Mexico and Southern Colorado: A Linguistic Atlas}.  Albuquerque:  The University of New Mexico Press.

	Bloom, Lansing S. (ed.).  1936.  “Bourke on the Southwest.”  \textit{New Mexico Historical Review} 11(1). 77--122; 11(3). 217--44.

	Bodo, Murray (ed.). 1998. \textit{ Tales of an Endishodi: Father Berard Haile and the Navajos, 1900{}-1961.}  Albuquerque:  University of New Mexico Press.

	Bowden, J. J.  2004.  Ojo del Espiritu Santo Grant.  New Mexico Office of the State Historian; \url{http://admin.newmexicohistory.org/filedetails.php?fileID=24960} (accessed 4 Sept. 2015).

	Bowles, David E., Karim Aziz \& Charles L. Knight. 2000.  Macrobrachium (Decapoda): Caridea: Palaeomonidae) in the Contiguous United States: A Review of the Species and an Assessment of Threats to Their Survival.  \textit{Journal of Crustacean Biology} 20(1). 158--174.

	Briggs, Walter. 1976.  \textit{Without Noise of Arms: The 1776 Domínguez-Escalante Search for a Route from Santa Fe to Monterey}.  Flagstaff, AZ:  Northland Press.

	Bright, William. 2013.  \textit{Native American Placenames of the Southwest: A Handbook for Travelers}.  Norman:  University of Oklahoma Press.

	Brooks, Clinton E. \& Frank D. Reeve (eds.). 1948.  \textit{Forts and Forays. James A. Bennett: A Dragoon in New Mexico, 1850--1856.}  Albuquerque:  The University of New Mexico Press.

	Brooks, James F.  2002.  \textit{Captives and Cousins: Slavery, Kinship, and Community in the Southwest Borderlands}.  Chapel Hill:  University of North Carolina Press, for the Omohundro Institute of Early American History and Culture.

	Brown, Cecil H.  1994.  Lexical Acculturation in Native American Languages.  \textit{Current Anthropology} 35(2). 95--117.

	Brugge, David M.  1965.  \textit{Long Ago in Navajoland.}  Navajoland Publications, 6.  Window Rock, AZ: Navajo Tribal Museum.

	Brugge, David M. 1973. \textit{Navajo Pottery and Ethnohistory}.  Window Rock, AZ:  Navajo Tribal Museum.

	Brugge, David M. 1980.  \textit{A History of the Chaco Navajos}.  Reports of the Chaco Center, 4.  Albuquerque:  Division of Chaco Research, U.S. National Park Service.

	Brugge, David M. 1985. \textit{Navajos in the Catholic Church Records of New Mexico}.  Tsaile, AZ:  Navajo Community College Press.  [repr. Santa Fe, NM:  SAR Press, 2010]

	Brugge, David M. 1986. \textit{Tsegai: An Archaeological Ethnohistory of the Chaco Region}.  Washington:  U.S. National Park Service.

	Brugge, David M. 1993a. An Investigation of AIRFA Concerns Relating to the Fruitland Coal Gas Development Area.  Albuquerque:  Office of Contract Archaeology, University of New Mexico.

	Brugge, David M. 1993b.  Eighteenth-Century Fugitives from New Mexico among the Navajos.  In June-el Piper (comp.), Alexandra Roberts \& Jenevieve Smith (eds.), \textit{Papers from the Third, Fourth, and Sixth Navajo Studies Conferences}, 279--283.  Window Rock, AZ:  Navajo Nation Historic Preservation Department.

	Brugge, David M. 2006.  When Were the Navajo? In Regge N. Wiseman, Thomas C. O’Laughlin \& Cordelia T. Snow (eds.), \textit{Southwestern Interludes: Papers in Honor of Charlotte J. and Theodore R. Frisbie}, 45--52.  Archaeological Society of New Mexico, 32.  Albuquerque.

	Bryan, Kirk.  1925.  Date of Channel-Trenching in the Arid Southwest.  \textit{Science} 62(1607). 338--344.

	Burrill, Meredith F.  1956.  Toponymic Generics II.  \textit{Names} 4(4). 226--240.

	Bye, Robert A. 1981.  Quelites—Ethnoecology of Edible Greens—Past, Present, Future.  \textit{Journal of Ethnobiology} 1(1). 109--123.

	Callaghan, Catherine A. \& Geoffrey Gamble  1996.  Borrowing. In Ives Goddard (ed.), \textit{Handbook of North American Indians, vol. 17.  Languages}, 111--116.  Washington:  Smithsonian Institution.

	Carey, Harold Jr.  2014.  Antonio el Pinto Chief of the Navajos.  Navajo People Culture and History; \url{http://navajopeople.org/blog/antonio-el-pinto-chief-of-the-navajos} (accessed 25 August 2015).

	Chavez, Angelico. 1979.  Genízaros.  In Alfonso Ortiz (ed.), \textit{Handbook of North American Indians, vol. 9.  Southwest}  198--200.  Washington:  Smithsonian Institution.

	Cobos, Rubén. 1983.  \textit{A Dictionary of New Mexico and Southern Colorado Spanish}, 2\textsuperscript{nd} ed,.  Santa Fe:  Museum of New Mexico Press.

	Conrad, Howard Louis. 1890.  \textit{“Uncle Dick” Wootton, the Pioneer Frontiersman of the Rocky Mountain Region}.  Chicago:  W. E. Dibble \& Co.  [repr. New York:  Time-Life Books, 1980]

	Correll, J. Lee. 1970. \textit{Sandoval—Traitor or Patriot?  Navajo Historical Publications, Biographical Series, 1}.  Window Rock, AZ:  Navajo Tribal Museum.

	Correll, J. Lee. 1979.  \textit{Through White Men’s Eyes: A Contribution to Navajo History, A Chronological Record of the Navajo People from Earliest Times to the Treaty of June 1, 1868}. (6 vols.) Window Rock, AZ:  Navajo Heritage Center.

	Cummings, Byron. 1910.  The Ancient Inhabitants of the San Juan Valley.  \textit{Bulletin of the University of Utah} 3(3:2), Archaeological Number 2. 1--45.

	Díaz, Josef, ed. 2014. \textit{The Art and Legacy of Bernardo Miera y Pacheco: New Spain’s Explorer, Cartographer, and Artist}.  Santa Fe:  Museum of New Mexico Press.

	Downer, Alan S. ca. 1988.  Navajo Nation Historic Preservation Plan Pilot Study: Identification of Cultural and Historic\textit{ Properties in Seven Arizona Chapters of the Navajo Nation}.  [Window Rock, AZ]:  Navajo Nation Historic Preservation Department and Navajo Nation Archaeology Department.

	Downer, Alan S. ca. 1989. \textit{Navajo Nation Historic Preservation Plan Pilot Study: Identification of Cultural and Historic Properties in Six New Mexico Chapters of the Navajo Nation}.  [Window Rock, AZ]:  Navajo Nation Historic Preservation Department and Navajo Nation Archaeology Department.

	Downs, James F.  1972.  \textit{The Navajo}.  New York:  Holt, Rinehart and Winston.

	Dutton, Clarence E. 1886.  Mount Taylor and the Zuñi Plateau.  In \textit{49\textsuperscript{th} Congress, 1\textsuperscript{st} Session, House of Representatives, Executive Document, 13}, 105--198.  Washington:  Government Printing Office.  [available at Google Books]

	Dyen, Isidore \& David F. Aberle  1974. \textit{Lexical Reconstruction: The Case of the Proto-Athapaskan Kinship System}.  London:  Cambridge University Press.

	Ebright, Malcolm \& Rick Hendricks. 2006. \textit{The Witches of Abiquiu: The Governor, the Priest, the Genízaro Indians, and the Devil}.  Albuquerque:  University of New Mexico Press.

	Eidenbach, Peter L.  2012.  \textit{An Atlas of Historic New Mexico Maps, 1550--1941}.  Albuquerque:  The University of New Mexico Press.

	El Bienhablao. n.d.  (Website): \url{http://www.elbienhablao.es/significado-cachuli}; accessed 4 Sept. 2015.

	Farmer, Malcolm F.  1953.  The Growth of Navajo Culture.  In  Irwin T. Sanders, Richard B. Woodbury, Frank J. Essene, Thomas P. Field, Joseph R. Schwendeman \& Charles E. Snow (eds.), \textit{Societies around the World.  1, Eskimo, Navajo, Baganda},   199--202.  New York:  The Dryden Press.

	Ferguson, T. J. \& E. Richard Hart. 1985. \textit{A Zuni Atlas}.  The Civilization of the American Indian Series, 172.  Norman:  University of Oklahoma Press.

	Fishler, Stanley A.  1953. \textit{In the Beginning: A Navaho Creation Myth}.  University of Utah Anthropological Papers, 13.  Salt Lake City:  The University of Utah Press.

	Forbes, Jack D. 1960.  \textit{Apache, Navaho, and Spaniard}.  Norman:  University of Oklahoma Press.

	Foreman, Grant (ed.)  1941. \textit{A Pathfinder in the Southwest: The Itinerary of Lieutenant A. W. Whipple During His Explorations for a Railway Route from Fort Smith to Los Angeles in the Years 1853--1854}.  Norman:  University of Oklahoma Press.

	Fort Huachuca, Arizona. n.d.  Huachuca on Maps [Website]. \url{http://huachuca.army.mil/pages/history/maps.html} [1849 Kern map].

	Franciscan Fathers, The. 1910. \textit{An Ethnologic Dictionary of the Navaho Language}.  St. Michaels, AZ:  The Franciscan Fathers.

	Franstead, Dennis. ca. 1979. \textit{An Introduction to the Navajo Oral History of Anasazi Sites in the San Juan Basin Area}.  Albuquerque:  University of New Mexico Chaco Project.

	Frink, Maurice. 1968. \textit{Fort Defiance and the Navajos}.  Boulder, CO:  Pruett Press.

	García de Diego, Vicente. 1955. \textit{Diccionário Etimológico Español e Hispanico}.  Madrid:  Editorial S. A. E. T. A.

	Gelling, Margaret. 1988. \textit{Signposts to the Past: Place-names and the History of England}.  Chichester, West Sussex:  Phillimore.

	Goodman, James M.  1982. \textit{The Navajo Atlas: Environments, Resources, People, and History of the Diné Bikeyah}.  Norman:  University of Oklahoma Press.

	Gregory, Herbert E.  1916. \textit{The Navajo Country: A Geographic and Hydrographic Reconnaissance of Parts of Arizona, New Mexico, and Utah}.  Washington:  Government Printing Office.

	Gregory, Herbert E. 1917.  \textit{Geology of the Navajo Country: A Reconnaissance of Parts of Arizona, New Mexico and Utah}.  Professional Paper, 93.  Washington:  U.S. Geological Survey.

	Haas, Mary R.  1978. \textit{Language, Culture, and History: Essays by Mary R. Haas}.  Anwar S. Dil, comp.  Stanford, CA:  Stanford University Press.

	Haile, Berard. 1938. \textit{\textit{Origin Legend of the Navajo Enemyway}}.  Publications in Anthropology, 17.  New Haven, CT:  Yale University.

	Haile, Berard. 1950. \textit{A Stem Vocabulary of the Navaho Language, Navaho-English}.  St. Michaels, AZ:  St. Michaels Press.

	Haile, Berard. 1951. \textit{A Stem Vocabulary of the Navaho Language, English-Navaho}.  St. Michaels, AZ:  St. Michaels Press.

	Haile, Berard. 1978. \textit{Love-Magic and Butterfly People: The Slim Curly Version of the Ajiłee and Mothway Myths}.  Irvy W. Goossen and Karl W. Luckert, eds.  American Tribal Religions, 2.  Flagstaff:  Museum of Northern Arizona Press.

	Haile, Berard. 1981. \textit{The Upward Moving and Emergence Way: The Gishin Biye’ Version}.  Karl W. Luckert (ed.). Lincoln:  University of Nebraska Press.

	Harrington, John Peabody. 1916.  The Ethnogeography of the Tewa Indians.  \textit{Annual Report, 29},  29--618.  Washington:  Bureau of American Ethnology.

	Haskell, J. Loring. 1987. \textit{Southern Athapaskan Migration, A.D. 200--1750}.  Tsaile, Ariz.:  Navajo Community College Press.

	Hedquist, Saul L., Stewart B. Koyiyumptewa, Peter M. Whiteley, Leigh J. Kuwanwisiwma, Kenneth C. Hill \& T. J. Ferguson. 2014.  Recording Toponyms to Document the Endangered Hopi Language.  \textit{American Anthropologist} 116(2). 324--331.

	Hester, James J.  1962. \textit{Early Navajo Migrations and Acculturations in the Southwest}.  (Papers in Anthropology 6.)  Santa Fe:  Museum of New Mexico.

	Hodge, Frederick Webb. 1937.  \textit{History of Hawikuh, New Mexico: One of the So-called Cities of Cíbola}. Publications of the Frederick Webb Hodge Anniversary Publication Fund, 1.  Los Angeles:  Southwest Museum.

	Hoijer, Harry. 1956. The Chronology of the Athabaskan Languages. \textit{International Journal of American Linguistics} 22(4).219—32.

	Holmes, Barbara E. 1989. \textit{American Indian Land Use of El Malpais}. Albuquerque: Office of Contract Archaeology, University of New Mexico.

	Hughes, John T. 1848. \textit{Doniphan’s Expedition; Containing an Account of the Conquest of New Mexico; General Kearney’s Overland Expedition to California; Doniphan’s Campaign against the Navajos; His Unparalleled March upon Chihuahua and Durango; and the Operations of General Price at Santa Fé}.  Cincinnati:  U. P. James.  [repr. College Station:  Texas A\&M University Press, 1997; Texas A \& M University Military History Series, 56.]

	Iverson, Peter. 2002. \textit{Diné: A History of the Navajos}  Albuquerque:  The University of New Mexico Press.

	Ives, John W., Duane G. Froese, Joel C. Janetski, Fiona Brock \& Christopher Bronk Ramsey.  2014.  A High Resolution Chronology for Steward’s Promentory Culture Collections, Promontory Point, Utah.  \textit{American Antiquity} 79(4). 616--637.

	Jett, Stephen C.  1964.  Pueblo Indian Migrations: An Evaluation of the Possible Physical and Cultural Determinants.  \textit{American Antiquity} 29(3). 281--300.


	Jett, Stephen C.  1965.  Red Rock Country.  \textit{Plateau} 37(3). 80--84.

	Jett, Stephen C. 1970.  An Analysis of Navajo Place-Names.  \textit{Names} 18(3). 175--184.

	Jett, Stephen C.  1978.  The Origins of Navajo Settlement Patterns.  \textit{Annals of the Association of American Geographers} 68(3). 351--362.

	Jett, Stephen C.  1982.  “Ye’iis Lying Down,” a Unique Navajo Sacred Place.  In   David M. Brugge \& Charlotte J. Frisbie (eds.), \textit{Navajo Religion and Culture: Selected Studies. Papers in Honor of Dr. Leland C. Wyman}, 138--149.   Museum of New Mexico, Papers in Anthropology, 17.  Santa Fe:  Museum of New Mexico Press.

	Jett, Stephen C.  1992.  An Introduction to Navajo Sacred Places.  \textit{Journal of Cultural Geography} 13(2). 29--39.

	Jett, Stephen C.  1997.  Place-Naming, Environment, and Perception among the Canyon de Chelly Navajo of Arizona.  \textit{The Professional Geographer} 49(4). 481--493.

	Jett, Stephen C.  2001. \textit{Navajo Placenames and Trails of the Canyon de Chelly System, Arizona}.  American Indian Studies, 12.  New York:  Peter Lang Publishing.

	Jett, Stephen C.  2006.  Reconstructing the Itineraries of Navajo Chantway Stories: A Trial at Canyon de Chelly, Arizona.  In Regge N. Wiseman, Thomas C. O’Laughlin \& Cordelia T. Snow (eds.) \textit{Southwestern Interludes: Papers in Honor of Charlotte J. and Theodore R. Frisbie},  75--86.  Archaeological Society of New Mexico, 32.  Albuquerque.

	Jett, Stephen C.  2011.  Landscape Embedded in Language: The Navajo of Canyon de Chelly, Arizona, and Their Named Places. In David M. Mark, Andrew G. Turk, Niclas Burenhult \& David Stea (eds.), \textit{Landscape in Language: Transdisciplinary Perspectives}, 327--342.  Culture and Language Use: Studies in Anthropological Linguistics, 4.  Gunter Senft, ed.  Philadelphia:  John Benjamins Publishing Company.

	Jett, Stephen C.  2014.  Place Names as the Traditional Navajo’s Title-deeds, Border-alert System, Remote Sensing, Global Positioning System, Memory Bank, and Monitor Screen.  \textit{Journal of Cultural Geography} 31(1). 106--113.

	Julyan, Robert. 1998. \textit{The Place Names of New Mexico}.  Albuquerque:  The University of New Mexico Press.

	Jurovics, Toby, Carol M. Johnson, Glen Willumson \& William F. Stapp. 2010. \textit{Framing the West: The Survey Photographs of Timothy H. O’Sullivan}.  Washington:  Library of Congress and Smithsonian American Art Museum/New Haven:  Yale University Press.

	Kari, James. 1989.  Some Principles of Alaskan Toponymic Knowledge.  In Mary Ritchie Key \& Henry M. Hoenigswald (eds.), \textit{General and Amerindian Ethnolinguistics: In Remembrance of Stanley Newman},  129--151.  Berlin:  Mouton de Gruyter.


	Kari, James. 2011.  A Case Study in Ahtna Athabascan Geographic Knowledge.  In David M. Mark, Andrew G. Turk, Niclas Burenhult \& David Stea (eds.), \textit{Landscape in Language: Transdisciplinary Perspectives},  239--260.  Culture and Language Use: Studies in Anthropological Linguistics, 4.  Gunter Senft, ed.  Philadelphia:  John Benjamins Publishing Company.

	Kari, James \& Ben A. Potter (eds.)  2010. \textit{The Dene-Yeniseian Connection}.  Anthropological Papers of the University of Alaska, N.S. 5(1--2).  Fairbanks, AK:  Department of Anthropology and Alaska Native Language Center.

	Kearns, Robin A. \& Lawrence D. Berg. 2002.  Proclaiming Place: Towards a Geography of Place Name Pronunciation.  \textit{Social \& Cultural Geography} 3(3). 283--302.

	Kelley, Klara Bonsack \& Harris Francis. 1994. \textit{Navajo Sacred Places}.  Bloomington:  Indiana University Press.

	Kelley, Klara Bonsack \& Harris Francis. 2018. \textit{Navajoland Trading Post Encyclopedia}. Window Rock, AZ: Navajo Nation Heritage and Historic Preservation Department.

	Kelly, Lawrence. 1970. \textit{Navajo Roundup: Selected Correspondence of Kit Carson’s Expedition against the Navajo, 1863--1865}.  Boulder, CO:  The Pruett Publishing Company.

	Kessell, John L.  2002. \textit{Spain in the Southwest: A Narrative History of Colonial New Mexico, Arizona, Texas, and California}.  Norman:  University of Oklahoma Press.



	Kessell, John L.  2013. \textit{Miera y Pacheco: A Renaissance Spaniard in Eighteenth-Century New Mexico}.  Norman:  University of Oklahoma Press.

	Klah, Hasteen  1942. \textit{Navajo Creation Myth: The Story of the Emergence}.  Navajo Religion Series, 1.  Santa Fe:  Museum of Navajo Ceremonial Art.

	Kluckhohn, Clyde \& Dorothea Leighton.  1946. \textit{The Navaho}.  Cambridge, MA:  Harvard University Press.  [rev. ed., 1962]

	La Farge, Oliver. 1975. The Navajos. In Alan Ternes (ed.), \textit{Antas, Indians, and Little Dinosaurs}, 240--49. New York: Charles Scribner's Sons.

	Leopold, Luna B.  1951. \textit{Vegetation of Southwestern Watersheds in the Nineteenth Century}.  New York:  American Geographical Society.

	Levy, Jerrold E.  1998. \textit{In the Beginning: The Navajo Genesis}.  Berkeley:  University of California Press.

	Linford, Laurence D.  2000. \textit{Navajo Places: History, Legend, Landscape}.  Salt Lake City:  The University of Utah Press.

	Linford, Laurence D.  2005.  Tony Hillerman’s Navajoland: Hideouts, Haunts, and Havens in the Joe Leaphorn and Jim Chee\textit{ }Mysteries, 2\textsuperscript{nd} ed.  Salt Lake City:  The University of Utah Press.

	Madsen, Steven K.  2010. \textit{Exploring Desert Stone: John N. Macomb’s 1859 Expedition to the Canyonlands of the Colorado}.  Logan, UT:  Utah State University Press.

	Marino, C. C.  1954. The Seboyetanos and the Navahos.  \textit{New Mexico Historical Review} 29(1). 8--27.

	Marmon, Walter G.  1894.  Navajo Agency.  In \textit{Report on Indians Taxed and Not Taxed in the United States (Except Alaska) at the Eleventh Census: 1890},   154--160 + plates.  Washington:  U.S. Census Office.  [repr. New York:  Norman Ross Publishing, 1994]

	Matson, R. G. \& M. P. R. Magne  2013.  North America: Na Dene/Athapaskan Archaeology and Linguistics. In Immanuel Ness (ed.), \textit{The Encyclopedia of Global Human Migration, 1, Prehistory},  1--7.  Hoboken, NJ:  Wiley-Blackwell.

	Matthews, Washington. (ed.)  1897. \textit{Navaho Legends}.  Boston:  Houghton Mifflin.  [repr. Salt Lake City:  The University of Utah Press, 1994]


	Matthews, Washington. (ed.)  1902. \textit{The Night Chant: A Navajo Ceremony}.  Memoirs, 6.  New York:  American Museum of Natural History.  [repr. Salt Lake City:  The University of Utah Press, 1995]

	Matthews, Washington. (ed.)  1907. \textit{Navaho Myths: Prayers and Songs with Texts and Translations}.  Pliny Earle Goddard, ed.  University of California Publications in American Archaeology and Ethnology, 5(2).  Berkeley.

	McNitt, Frank. 1964. \textit{Navaho Expedition: Journal of a Military Reconnaissance from Santa Fe, New Mexico, to the Navajo Country Made in 1849 by Lieutenant James H. Simpson}.  Norman, OK:  University of Oklahoma Press.

	McNitt, Frank. 1972. \textit{Navajo Wars: Military Campaigns, Slave Raids, and Reprisals}.  Albuquerque:  The University of New Mexico Press.

	Mindeleff, Cosmos. 1897. \textit{The Cliff Ruins of Canyon de Chelly}.  Bureau of American Ethnology Annual Report, 16. 73--98.  Washington.

	Mirkowich, Nicholas. 1941.  A Note on Navajo Place Names.  \textit{American Anthropologist} N.S. 43(2). 313--314.

	Moore, William Haas. 1994. \textit{Chiefs, Agents, and Soldiers: Conflict on the Navajo Frontier, 1868--1882}.  Albuquerque:  The University of New Mexico Press.

	Murphy, Lawrence R.  1967. \textit{Indian Agent in New Mexico: The Journal of Special Agent W. F. M. Arny, 1870}.  Southwestern Series, 5.  Santa Fe:  Stagecoach Press.

	Myers, Garth Andrew. 1996.  Naming and Placing the Other: Power and the Urban Landscape.  \textit{Tijdschrift voor Economische en Sociale Geografie} 87. 237--246.

	Newcomb, Franc Johnson. 1970. \textit{Navajo Bird Tales Told by Hosteen Klah Chee}.  Wheaton, IL:  Theosophical Publishing House.

	O'Bryan, Aileen. 1956. \textit{The Dîné: Origin Myths of the Navajo Indians}.  Bulletin,  163.  Washington:  Bureau of American Ethnology.  [new ed., London:  Forgotten Books, 2008]

	Ostermann, Leopold. 2004a.  First Impressions and Councils with Headmen, 1900.  In Howard M. Bahr (ed.), \textit{The Navajos as Seen by the Franciscans, 1898--1921}, 52--71.  Lanham, MD:  The Scarecrow Press.  [orig. pub. 1901]


	Ostermann, Leopold. 2004b.  Neighbors. In Howard M. Bahr (ed.), \textit{The Navajos as Seen by the Franciscans, 1898--1921}, 173--181.  Lanham, MD:  The Scarecrow Press.  [orig. pub. 1902 ]

	Pearce, T. M., ed. 1932.  The English Language in the Southwest.  \textit{New Mexico Historical Review} 7(3). 210--32.

	Pearce, T. M., ed. 1965. \textit{New Mexico Place Names: A Geographical Dictionary}.  Albuquerque:  University of New Mexico Press.

	Perry, Richard J.  1991. \textit{Western Apache Heritage: People of the Mountain Corridor}.  Tucson:  The University of Arizona Press.

	Pine Hill High School students. 1982.  A Condensed History of the Ramah Navajo.  \textit{Tsá’ Ászi’} 5(1). 5--8.

	Reeve, Frank D.  1956.  Early Navajo Geography.  \textit{New Mexico Historical Review} 31(4). 290--309.

	Reeve, Frank D. 1971a.  Navajo Foreign Affairs, 1795--1846.   \textit{New Mexico Historical Review} 46(2). 100--132.  [repr. Tsaile, AZ:  Navajo Community College Press, 1983]

	Reeve, Frank D. 1971b.  Navajo Foreign Affairs, 1795--1846. Part II, 1816--1824.  \textit{New Mexico Historical Review} 46(3). 223--251.  [repr. Tsaile, AZ:  Navajo Community College Press, 1983]

	Reichard, Gladys. 1990. \textit{Navaho Religion: A Study in Symbolism}.  Bollingen Series, 18.  Mythos Series.  Princeton, NJ:  Princeton University Press.  [orig. pub. 1950]

	Rice, Josiah. 1970. \textit{A Canoneer in Navajo Country: Journal of Private Josiah M. Rice, 1851}.  Richard H. Dillon, ed.  Denver:  Old West Publishing Company: Fred A. Rosenstock.

	Robinson, Jacob S. 1848.  \textit{ Journal of the Santa Fe Expedition under Colonel Doniphan}, which left St. Louis in June, 1846.   Portsmouth, NH.  [repr. Santa Barbara, CA:  Narrative Press, 2001]

	Salabye, John E. Jr.  2013.  Reader Question.  There is a place called Sǫ silá in the Wheatfields area. Why is this place important?  \textit{Leading the Way: The Wisdom of the Navajo People} 11(11). 10.

	Salabye, John E. Jr.  \& Kathleen Manolescu. 2013.  Coyote and the Stars.  \textit{Leading the Way: The Wisdom of the Navajo People} 11(12). 20--21.

	Sapir, Edward. 1921. \textit{Language: An Introduction to the Study of Speech}.  New York:  Harcourt, Brace.

	Santamaría, Francisco J.  1959.  \textit{Diccionario de Mejicanismos}.  Mexico City:  Editorial Porrúa.

	Schaafsma, Curtis F.  2002.  \textit{Apaches de Navajo: Seventeenth-Century Navajos in the Chama Valley of New Mexico}.  Salt Lake City: The University of Utah Press.

	Seymour, Deni J. (ed.).  2012.  \textit{From the Land of Ever Winter to the American Southwest: Athabaskan Migrations, Mobility, and Ethnogenesis}.  Salt Lake City:  The University of Utah Press.

	Seymour, Deni J. 2013.  Platform Cache Encampments: Implications for Mobility Strategies and the Earliest Ancestral Apaches.  \textit{Journal of Field Archaeology} 38(2). 161--172.

	Shepardson, Mary. 1963.  \textit{Navajo Ways in Government: A Study in Political Process}.  Memoirs, 96.  [Menasha, WI]:  American Anthropological Association.

	Simoons, Frederick J. 1994.  \textit{Eat Not This Flesh: Food Avoidances from Prehistory to the Present}, 2\textsuperscript{nd} ed.  Madison:  University of Wisconsin Press.

	Sitgreaves, L[orenzo]1962.  \textit{Report on an Expedition down the Zuni and Colorado Rivers}.  Chicago:  Rio Grande Press.  [repr. from 32d Congress, 2d Session, Senate, Executive (Document) 59, Washington:  Robert Armstrong, Public Printer, 1853]

	Spicer, Edward H.  1962.  \textit{Cycles of Conquest:  The Impact of Spain, Mexico, and the United States on the Indians of the Southwest, 1533--1960}.  Tucson:  The University of Arizona Press.

	Stacey, May Humphries. 1929.  \textit{Uncle Sam’s Camels—His Journal, 1857-1858}, ed. Lewis B. Lesley.  Cambridge, MA:  Harvard University Press.

	Steingass, F.  1987.  \textit{A Learner’s English-Arabic Dictionary}.  London:  Oriental University Press.

	Studerus, Lenard. 2001.  \textit{A Thematic Introduction to New Mexico Spanish Nouns: Representative Nested Examples for New Mexico and Southern Colorado}.  Seattle:  Folkecastle Hill Publishing.

	Swadesh, Frances. 1979.  The Structure of Hispanic-Indian Relations in New Mexico.  In Paul Kutsche (ed.), \textit{Survival of Spanish-American Villages}, 53--61.  Colorado Springs:  Research Committee, Colorado College.

	The Santa Fe Site. n.d.  (Website): http://www.thesantafesite.com/articles-database/Matanza-{}-{}-A-New-Mexico-Celebration.html.

	Towner, Ronald H. (ed.). 1996. \textit{The Archaeology of Navajo Origins}.  Salt Lake City:  University of Utah Press.

	Trafzer, Clifford. 1982.  \textit{The Kit Carson Campaign: The Last Great Navajo War}.  Norman:  University of Oklahoma Press.

	Twitchell, Ralph Emerson. 1976.  \textit{The Spanish Archives of New Mexico}, repr. ed., 2 vols.  New York:  Arno Press.  [orig. pub. Cedar Rapids, IA:  Torch Press, 1914]

	Tyler, Daniel. 1984.  \textit{Sources for New Mexico History 1821--1848}.  Santa Fe:  Museum of New Mexico Press.

	Underhill, Ruth. 1956.  \textit{The Navajos}.  Norman:  The University of Oklahoma Press.

	van Cott, John W.  1990. \textit{Utah Place Names: A Comprehensive Guide to the Origins of Geographic Names: A Compilation}.  Salt Lake City:  The University of Utah Press.

	Van Valkenburgh, Richard F.  1937.  “The History of the Navajo Nation,” unpublished manuscript 284-994, Box 9, Richard Fowler Van Valkenburgh Collection, Arizona Historical Society, Tucson.

	Van Valkenburgh, Richard F.  1974.  Navajo Sacred Places.  In \textit{Navajo Indians III}, 9--199.  New York:  Garland Publishing.

	Van Valkenburgh, Richard F.  1999.  Diné Bikéyah (2\textsuperscript{nd} ed.).  Mancos, CO:  Time Traveler Maps.  [1\textsuperscript{st} ed., Window Rock, AZ:  U.S. Indian Service, 1941]

	Van Valkenburgh, Richard F. \& Frank Walker. 1945.  Old Placenames in the Navaho Country.  \textit{The Masterkey} 19(3). 89--94.  Southwest Museum.

	Vogt, Evon Z.  1961.  Navaho. In Edward H. Spicer (ed.),  \textit{Perspectives in American Indian Culture Change}, 278--236.  Chicago:  University of Chicago Press.

	Waters, Frank. 1963.  \textit{Book of the Hopi}.  New York:  The Viking Press.

	Watson, Editha L. 1964.  \textit{Navajo Sacred Places}.  Navajoland Publications, 5.  Window Rock, AZ:  Navajo Tribal Museum.

	Weber, Anselm. 2004a.  The Navajo Trouble of 1905.  In Howard M. Bahr (ed.), \textit{The Navajos as Seen by the Franciscans, 1898--1921}, 221--243.  Lanham, MD:  The Scarecrow Press.  [orig. pub. 1902 ]

	Weber, Anselm. 2004b.  1917: Council at Chin Lee. In Howard M. Bahr (ed.), \textit{The Navajos as Seen by the Franciscans, 1898--1921}, 418--421.  Lanham, MD:  The Scarecrow Press.  [orig. pub. 1902 ]

	Wheelwright, Mary C.  1938.  \textit{Tleji or Yehbechai Myth}.  Santa Fe:  House of Navajo Religion.

	Wheelwright, Mary C. \& David P. McAllester. 1988. \textit{The Myth and Prayers of the Great Star Chant and the Myth of the Coyote Chant}.  Tsaile, AZ:  Navajo Community College Press.

	Wilken, Robert L. 1955.  \textit{Anselm Weber, O.F.M., Missionary to the Navajo 1898-1921}.  Milwaukee, WI:  The Bruce Publishing Company.

	Williams, Edwin B. 1955.  \textit{Spanish and English Dictionary/Diccionario Inglés y Español}.  New York:  Henry Holt and Company.

	Williams, J.S. \& T. Blackhorse. n.d. ``Haash holyé? What's this place called?'' Metaphor, metonymy, and iconicity in Navajo placenaming practices. Unpublished paper (courtesy of David Mark).

	Wilson, Alan \& Gene Dennison. 1995.  \textit{Navajo Placenames: An Observer’s Guide}.  Guilford, CT:  Jeffrey Norton, Publishers.

	Wilson, Joseph. 2015. The Union of Two Worlds: Reconstructing Elements of Proto-Athabaskan Folklore and Religion.  \textit{Folklore} 20. 1--24.

	Winter, Mark. 2011. \textit{The Master Weavers: Celebrating One Hundred Years of Navajo Textile Artists from the Toadlena/Two Grey Hills Weaving Region}.  Newcomb, NM:  The Historic Toadlena Trading Post.

	Wozniak, Frank J., Mead F. Kemrer \& Charles M. Carillo. 1992.  \textit{History and Ethnohistory along the Rio Chama}.  Albuquerque:  U.S. Army Corps of Engineers, Albuquerque District.



	Wyman, Leland C. 1957.  \textit{Beautyway: A Navajo Ceremonial}.  Bollingen Series, 53.  New York:  Pantheon Books.

	Wyman, Leland C. 1962.  \textit{The Windways of the Navahos}.  Colorado Springs:  Taylor Museum of the Colorado Springs Fine Arts Center.

	Wyman, Leland C. 1970.  \textit{Blessingway}.  Tucson:  The University of Arizona Press.

	Wyman, Leland C. 1975.  \textit{The Mountainway of the Navajo}.  Tucson:  The University of Arizona Press.



	Young, Robert W. 1961.  \textit{The Navajo Yearbook, 8.}  Window Rock, AZ:  U.S. Bureau of Indian Affairs, Navajo Agency, Gallup Area Office.

	Young, Robert W. 1983.  Apachean Languages.  In Alfonso Ortiz (ed.), \textit{Handbook of North American Indians, 10.  Southwest},  393--400.  Washington:  Smithsonian Institution.


	Young, Robert W. \& William Morgan. 1980.  \textit{The Navajo Language: A Grammar and Colloquial Dictionary}.  Albuquerque:  The University of New Mexico Press.

	de Zárate Salmerón, Gerónimo. 1966. \textit{Relaciones: An Account of Things Seen and Learned by Father Jerónimo de Zárati Salmerón from the year 1538 to year 1626}.  Alicia Kennedy Milich, tr. and ed.  Albuquerque: Horn \& Wallace.

	Zolbrod, Paul G. 1984.  \textit{Diné Bahane’: The Navajo Creation Story.}  Albuquerque:  The University of New Mexico Press.

\end{hang}

\orcidfooter{Stephen Jett}{scjett@hotmail.com}{}

\label{jett-ch-end}
