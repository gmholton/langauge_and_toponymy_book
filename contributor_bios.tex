\thispagestyle{plain}
{\parindent0pt
\parskip6pt


\textbf{\large Contributors}
\vspace{6pt}

\textbf{Andrea L. Berez-Kroeker} is an Associate Professor in the Department of Linguistics at the University of Hawaiʻi at Mānoa, where she teaches class on language documentation and conservation. She has conducted language research with speakers of Ahtna and Dena'ina in Alaska, and of Kere and Kuman in Papua New Guinea. She is also active in the field of language archiving.

\textbf{Caleb Brucks} holds a Master's degree in Anthropology and Linguistics from the University of Regina. His thesis is on the use of directional  adverbs in Upper Tanana Dene. He conducted fieldwork with the Upper  Tanana language community from 2013–2015.

\textbf{Stephen C. Jett} isProfessor Emeritus of Geography and of Textiles and Clothing, University of California, Davis; served as Geography Chair.Scholarly interests: Navajo history and culture; pre-Columbian transoceanic influences. Author of six books and many articles. Founder and Editor of \textit{Pre-Columbiana: A Journal of Long-Distance Contacts}.

\textbf{David Jason Harris} is a graduate student in linguistics at the University of Alaska Fairbanks.

\textbf{Gary Holton} is Professor of Linguistics and Co-Director of the Biocultural Initiative of the Pacific at the University of Hawaiʻi at Mānoa. His research focuses on language and space and the development of infrastructure for documentary linguistics.

\textbf{Olga Lovick} holds a Ph.D. in Linguistics from the Universität zu Köln.  She is a Professor of Linguistics and Dene Language Studies at the First  Nations University of Canada. She has been working with Upper Tanana since 2006, with a focus on the documentation of traditional stories.

\textbf{Patrick Moore} is Associate Professor of Anthropology at the University of British Columbia. His primary research has been with Dene (Athabaskan) languages of the Yukon, British Columbia and Alberta on topics such as the use of digital technologies, literacy, code-switching, historical and traditional narratives, language revitalization, and ontology.

\textbf{Kenneth Pratt} is an anthropologist and ethnohistorian employed by the Bureau of Indian Affairs and has over 35 years of experience researching Alaska Native land claims. His research interests include the ethnohistory of western and southwestern Alaska, indigenous place names, oral history, intergroup relations and territoriality, and Russian America.

\textbf{Dr. Christine Schreyer} is an Associate Professor of Anthropology at UBC’s Okanagan campus, where she teaches courses in linguistic anthropology. Her research focuses on language documentation and revitalization in Canada (with Tlingit and Secwepemctsín speakers), as well as with Kala speakers in Papua New Guinea. She also conducts research on constructed languages (conlangs) and the fan communities that speak them.

\textbf{Thomas L. Thornton}

\textbf{Edward Vajda}

}
