\chapter[Introduction: James Kari's Contributions to Geolinguistics and Dene Language Studies]{\vspace{-125pt}
{\normalfont\normalsize\begin{flushright}\textit{``Geography is the central theme in Athabaskan culture history.''}\\[2pt]
\citep[443]{kari1996a}\\\end{flushright}\bigskip\bigskip
}
Introduction: James Kari's Contributions to Geolinguistics and Dene Language Studies}

\sethandle{10125/24838}


\def\authorlast{Holton \& Thornton}
\renewcommand{\beginchapter}{\pageref{intro-ch-begin}}
\renewcommand{\finishchapter}{\pageref{intro-ch-end}}
\label{intro-ch-begin}


\thispagestyle{firststyle}

\chapauth{Gary Holton}
\affiliation{University of Hawai‘i at Mānoa}
\chapauth{Thomas Thornton}
\affiliation{University of Alaska Southeast}
\authortoc{Gary Holton \& Thomas Thornton}

\section{Origins of this book}

This book is long overdue. The papers contained here were originally contributed to the Alaska Native Place Names Workshop, held in Anchorage in April 2015.\footnote{\url{https://sites.google.com/a/alaska.edu/alaska-native-place-names-workshop/}} That workshop brought together over one hundred Native speakers, language advocates, scholars, and agency representatives from across Alaska to discuss issues related to Native names, including strategies for developing a statewide database of place names and efforts to promote increased recognition of Native place names. The workshop was made possible thanks to the dedication of many people institutions, including primary sponsors Bristol Bay Native Corporation and the Alaska Native Language Center, but the presence of Jim Kari and his legacy was felt heavily at this event. Though it was convened more than 40 years after Jim arrived in Alaska, this first statewide place names workshop was very much inspired by Jim's tireless efforts to document and promote Alaska Native names.

So in a way this book is even longer overdue. It is well past time to recognize the enormous contribution that Jim has made to the documentation and appreciation of Alaska Native place names. Jim came to Alaska in 1972, while he was finishing a Ph.D. dissertation on Navajo verb prefix morphology. The Alaska legislature had just passed a bill recognizing Native languages and founding the Alaska Native Language Center, which Jim would soon join \citep{krauss1974}. Documentation of Dene languages was fragmentary at best, and the only records of Native place names were those found inaccurately spelled on maps and gazetteers. Now nearly a half a century later we are surrounded by Native names, while attending an event at the Dena'ina Center in Anchorage, visiting Denali, or walking past Troth Yeddha' Park on the University of Alaska Fairbanks campus. The increased visibility of Alaska Native place names is due in no small part to the tireless efforts of Jim Kari. His work has inspired many other, including the contributors to this volume, and continues to set the standard for place name documentation in Alaska and beyond. We review a few of Jim's major contributions below.

\section{Contributions to Dene linguistics}
Although Jim is widely known for his work in the domain of landscape and toponymy, his contributions to Dene (Athabascan) linguistics are no less substantial. Jim is by far the most prolific field worker on Alaska Dene languages; he has done primary field work with all eleven of the Alaska Dene languages, focusing especially Dena’ina and Ahtna. He has assiduously sought out knowledge experts and collaborated with them to document their knowledge of Dene language and culture. Many of the most famous Alaska Native speakers are known today through works which Jim helped to compile and edit. These include now well-known names such as Shem Pete \citealt{kari1987,kari2004a}, Peter Kalifornsky \citep{kari1991a}, Katie John \citep{kari1986a}, Mary Tyone \citep{tyone1996}, and Andrew Balluta \citep{balluta2008}. Jim's later works on Dene texts often include audio recordings so that the reader can hear the original narrative while reading the transcribed and translated text. This corpus of published and unpublished annotated narratives places the Alaska Dene languages among the best documented languages in North America, providing a record of language use in context.

In addition to his work on texts, Jim has made groundbreaking contributions to the field of Dene lexicography. His \textit{Ahtna Athabaskan Dictionary} \nocite{kari1990} was the first comprehensive dictionary of an Alaska Dene language, and it laid out many of the principles now widely accepted in Dene lexicography, particularly with respect to the representation of verb forms. Kari was an early and enthusiastic adopter of Jeff Leer's theory of verb theme categories, which organizes verbs according to their aspectual inflectional and derivational patterns, providing some degree of morphological and semantic predictability \citep{kari1979}. The Ahtna dictionary explicitly categorizes verbs according to theme, showing the patterns of verb stem variation for each derived aspectual category and allowing users to explore semantic connections between abstract roots. Kari's lexicographic work also endorses a templatic approach to verbal morphology, proposing that verbal morphemes fit into predetermined slots in an overarching template rather than being combined according to morphophonological or semantic principles. His verb template for Ahtna includes \hl{XX} distinct verb prefix positions. While the templatic status of the Dene verb continues to be debated, the template has been widely adopted in practical lexicography.

The Ahtna dictionary also broke new ground technologically, being the first Alaska Dene dictionary produced using database software. Working with Bob Hsu's Lexware scripts, Kari was able to maintain lexical database files in a tagged text file format, separating content from formatting and allowing lexical files to be continuously updated and maintained. This same approach was followed in the development of the \textit{Koyukon Athabaskan Dictionary}, for which he also served as editor-in-chief\nocite{jette2000}. Today the use of lexical databases is taken for granted in documentary linguistics, and several well-supported software tools are available. But in the 1980’s when Jim was compiling the Ahtna dictionary most linguists were still using paper index cards to compile dictionaries, and few had considered the application of lexical databases to morphologically complex languages like Dene.

Jim's work is also groundbreaking in terms of its commitment to open access. His numerous publications represent just a fraction of his total contribution to Dene language documentation, but much of his unpublished has been widely circulated and shared. Jim remains the number one contributor to the Alaska Native Language Archive, with nearly one thousand deposits associated with his name.\footnote{see \url{http://www.uaf.edu/anla/about/statistics/}}

\section{Contributions to ethnogeography}
Ethnogeography is field with a long tradition of scholarly contributions from onomastics, linguistics, geography, anthropology, and related fields. Jim’s particularistic and theoretical interests in ethnogeography crossed all of these fields. He was not only careful documenter and translator of place names throughout the Alaska Dene country, the most sizeable linguistic region of Alaska, but also a keen analyst of their semantic and syntactic construction as part of a cognitive orientational and classificatory system. In this regard his publications have been foundational, especially as concerns what might be termed the hydronymic dynamics of place naming and orientation, which order the reckoning of space and direction in Dene language and thought. Kari views Dene place names not as isolated and arbitrary but rather as part of ``systematic multifunctional sign networks that are conducive to being memorized and that facilitate travel and occupancy over large areas'' \citeyearpar[444]{kari1996a}. Place names provide symbolic structure and organization to the landscape, a view which has been widely adopted by other scholars of ethnogeography both within Dene languages and beyond.

Building on the organizational nature of place name networks Jim also recognized the systematic development of hydronymic districts within an indigenous cartography, and related this to the evolution of linguistic and cultural groupings in Dene country. The utility of this kind of analysis has implications for historical and comparative linguistics and the study of prehistory, as Edward Vajda makes clear in his essay on Siberian Yeniseian hydronyms in this volume, noting how Jim’s “pioneering studies of Athapaskan hydronyms (Kari 1996, 2010) offer a potential model for investigating aboriginal place-naming systems in other regions of the world. The present article pays tribute to Jim’s work by applying his methodology.”

In addition Jim has been one of the most attentive scholars to indigenous travelogues and geographic texts, long a topic of interest among ethnogeographers and linguistic anthropologists, such as Melville Jacobs (\citeyear{jacobs1936}; see also citealt{hunn1990}, \citealt{thornton2012}). While geographic names may serve as linguistic artifacts on the land, geographic texts show the meshwork of human lines that knit these points together into a landscape, giving each name additional perspective, context, and meaning within human experience. Indeed such geographic texts constitute an important genre of place, especially among highly mobile peoples such as Dene and Australian Aborigines, where they are sometimes embedded in ancestral “songlines” and “Dreaming tracks.” Jim’s many documented geographic texts appear in a series of manuscripts and edited volumes, including the well-known \textit{Shem Pete’s Alaska} (2004, with Jim Fall and Matt Ganley), and Ahtna Travel Narratives (2010). Within this genre of place, Jim also commented on specific themes, such as the use of directionals, conservatism in naming, and the phenomenon of shared geographic knowledge, as well as subgenres, such as seasonal travel narratives. These lines of enquiry have also proved fruitful and are elaborated upon in essays by Berez, Lovick, Moore, Schreyer, and Thornton in this volume and beyond.

\hl{Add transition sentence or paragraph about relationship with archaeological investigations}

\section{Contributions to prehistory}
[It seems that the essays by Drozda, Pratt, Kaplan (?) and perhaps others could be referenced here as well as his work with Ben Potter and other archaeologists and ethno-historians]

``Athabaskan place name inventories are especially rich sources of information pertaining to Athabaskan prehistory.'' \citep[444]{kari1996a}

Beyond his work with Dene linguistics and ethnogeography, Kari is also well-known for his proposals concerning Dene prehistory. The regularity and structural uniformity of Dene place names has led Kari to make a series of proposals for the deep antiquity of Dene place names and indeed the languages themselves. Most linguists assume a time-depth of approximately 2000 years BP for the initial break-up of Proto-Dene into individual languages \citep{krauss1973a,holman2011}. Dating languages is particularly tricky without corroborating non-linguistic evidence, such as terminology for material culture items with known dates of introduction. Computational methods assuming a constant and uniform rate of change have been largely debunked. Nevertheless, the accepted dates for Dene are generally considered plausible owing to the extreme similarity among the languages; the languages simply haven't had sufficient time to become widely differentiated from each other.

Kari suggests an alternate view of Dene prehistory which maintains that Dene languages are conservative, perhaps deliberately slow. As formulated in the Geolinguistic Conservativism Hypothesis, Dene languages change at a slower rate and thus may be as old as 10,000 BP or more. The similarity in language structures, particularly place name structure, is taken by Kari as evidence of a ``distinctive Athabascan territorial ethos [which] is reflected in the similar, functional, and memorizable place names networks that we find in diverse parts of the large Athabascan language area'' \citep[207]{kari2010b}. Although the evidence for the Geolinguistic Conservativism Hypothesis remains circumstantial (and perhaps even circular), the possibility of a much greater time depth for Dene languages opens up several new avenues of inquiry.

Most notable of these is the possibility of a deep time connection between the Dene languages of North America and the Yeniseian languages of Siberia. Though a connection between these families has been suggested by a number of scholars, only recently have the efforts of Ed Vajda brought the tools of the comparative method to bear on the subject, drawing accurate data sets from both language families. Kari was instrumental in organizing the first Dene-Yeniseian Symposium at the University of Alaska Fairbanks in 2008, bringing together Vajda and a number of prominent scholars to re-evaluate the evidence for a prehistoric connection between Dene and Yeniseian languages \citep{kari-potter2010}. The Dene-Yeniseian Hypothesis has received critical attention from scholars and widespread calls for further research \citep{diamond2011}.

Tracing a long arc of prehistory Kari suggests that Dene place-naming is deliberate, with geopolitical consequences. Place name networks are not just cognitive maps which facilitate memorizability and way-finding, they are deliberate cognitive maps, created as a result of ancient land use agreements and boundary marking. While the evidence supporting a deep time depth for Dene names may be debatable, Jim's work has offered plausible proposals connecting geography, linguistics, and prehistory and providing scholars with many new avenues of research. Kari's proposals have left an indelible mark on the field.

\section{Contributions to policy}
Beyond these fruitful academic lines, Jim has also been an advocate for retaining and utilizing indigenous place names in official cartography, environment and resource management planning, and to help guide archaeological investigations into prehistory. He has worked closely as an advisor to a number of policy making bodies, including the Alaska Historical Commission, the National Park Service, and the U.S. Board on Geographic Names.

Jim's interest in Native place names goes well beyond the academic realm. From the outset Jim has been a tireless advocate for the public use and recognition of Native names. Shortly after the April 2015 workshop the name of Alaska's highest peak was officially changed from Mount McKinley to Denali, restoring a traditional Native place name and ending a 40-year naming dispute. Jim was an early advocate for this change. Writing in the February 1981 issue of \textit{Now in the North}, he noted that the name change ``could create greater public awareness of the importance of preserving Alaska's heritage of aboriginal place names.'' Kari recognized that place names are not merely politically significant but also culturally significant. Alaska Native place names reflect an Indigenous world view that is uniquely Alaskan.

\begin{quote}
``Changing the official name of the mountain from McKinley to Denali has been proposed to the U.S. Board of Geographic Names in Washington D.C. Ohio politicians oppose the change, but native and other Alaskans who support it do not intend it to dishonor former U.S. President McKinley. For the natives, there is a basic difference in cultural values reflected in the place name choice. Athabaskans, in marked contrast to Western cultures, do not name places after people, and it is absolutely unthinkable to many of them that the tallest mountain in their traditional territory should be named after a mere mortal.'' \citep[17]{kari1991}
\end{quote}

The resolution of the Denali controversy is merely the most prominent of a number of efforts at official recognition in which Jim has had a hand. Some of these efforts involved technical hurdles, including convincing the US Board on Geographic Names to recognize special characters required to spell Alaska Native names. One of the early victories in this effort was the name \textit{K'esugi Ridge}, officially adopted in 2002 for the long ridge in Denali State Park east of the Parks Highway. Official policy at the time discouraged the use of apostrophes in names, but the apostrophe in the word \textit{k'esugi} is not a superfluous grammatical feature but rather an essential part of the Dena'ina  writing system. The \textit{k'} symbol represents a completely different sound, and omitting the apostrophe changes the meaning of the word.

An apostrophe can also be found in the name \textit{Troth Yeddha'}, officially adopted in 2013 for the ridge on which the University of Alaska Fairbanks campus is located. This name was a first for its use of the commonly accepted Lower Tanana Athabascan orthography. It was the first official Indigenous place name on a university campus in the United States to be spelled in the proper orthography. Moreover, this name is significant in that both the specific and the generic components of the name are in the Native language. In the name \textit{K'esugi Ridge}, the generic component \textit{ridge} remains in English, whereas in the name \textit{Troth Yeddha'} both the specific \textit{troth} `Indian potato (Hedysarum)' and the generic \textit{yeddha'} `ridge' are Indigenous. That is, the name is not \textit{Troth Yeddha' Ridge} but simply \textit{Troth Yeddha'}.  Though the name was initially opposed by the University of Alaska Board of Regents, Kari's advocacy was instrumental in convincing university officials and local stakeholders to support the proposed name. Again, as with Denali it was the cultural connection that proved most persuasive. The Native name reflects a cultural connection with place.

The success of these official naming proposals has inspired numerous efforts to recognize Native names across Alaska. In 2014 the Gwich'in name \textit{Draanjik River} officially replaced \textit{Black River}. With a length of some 160 miles, this river may not have the same geographical prominence as Denali, but the name change is arguably more significant in that in contrast to Denali, few Alaskans were familiar with the Gwich'in name of the river prior to its adoption. So while the Denali name change puts an official seal on a widely accepted practice, the name Draanjik River promotes wider recognition of a traditional name once known only to locals.

Official Native names are being recognized at an increasing rate, from \textit{Tlax̲satanjín} in Southeast to \textit{Teedriinjik} in the Yukon Flats to \textit{Utqiagvik}, adopted in  2016 to replace the village name Barrow.

\hl{Talk about work with COGNA, advocacy for incorporating indigenous orthographies, etc. –well referenced in Gary’s presentation at the Sharing our Knowledge Conference}

\section{Contributions to scholarship}
Jim's contributions to Dene linguistics, ethnogeography, linguistic prehistory, and toponymy are so vast that it is impossible for researchers not to encounter his work. All who  have ventured to conduct research with Dene languages have been touched in some way by Jim and his work in some way or another. Jim has always been quick to share advice, provide access to recordings and field notes, and assist with research. This remains true whether the request comes from a vetted senior researcher or a novice student or, most notably, from a speaker or language learner. Jim's passion for Dene place names, places, and the people who inhabit those places is legendary, and this passion has inspired dozens of other researchers---the contributors to this volume among them.

The range of chapters in this volume attest to the breadth of Jim's work, cutting across linguistics, anthropology and prehistory. The chapters Caleb Brucks \& Olga Lovick and Andrea Berez-Kroeker are clearly inspired by Jim's work on elite travel narratives, examining the relationship of stories and discourse to place-naming. In a similar vein, chapters by Thomas Thornton and Christine Schreyer reveal how place names reflect  the storied nature of the landscape, as aspect which is also seen in Jim's long-standing commitment to the documentation of narrative. Chapters by Kenneth Pratt, Stephen Jett, and Robert Drozda  reflect Jim's attention to historical materials. The chapter by Harris \& Holton is clearly inspired by Jim's seminal work on Dene place-naming strategies. And the chapter by Ed Vajda reflects Jim's long interest in prehistory, potentially the potential for deep linguistics connections across Bering Strait. These works are notable for their diversity. Touching on historical linguistics, pragmatics, anthropology and geography, they are not easily categorized within a single field. What unites them---in this volume and as a discipline---is the common thread of Jim's relentless passion for Dene languages, place names, landscape, and the original inhabitants of that landscape.  
 
 
\refheading
\bibliographystyle{glossa}
\bibliography{holton}

\label{intro-ch-end}
