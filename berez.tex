\chapter[Directional Reference in Discourse and Narrative]{\vspace{-25pt}Directional Reference in Discourse and Narrative: Comparing indigenous and non-indigenous genres in Ahtna}

\sethandle{10125/24842}

% \usepackage[
% 	doi=false,
% 	backend=biber,
% 	natbib=true,
% 	style=biblatex-sp-unified,
% 	citestyle=sp-authoryear-comp]{biblatex}
%\usepackage[style=ldc.bst]{biblatex}



% Author last name as it appears in the header
\def\authorlast{Berez-Kroeker}

% change author in three references below to the actual author name so that this name is unique and matches the \label commands just below and at the end of the chapter
\renewcommand{\beginchapter}{\pageref{berez-ch-begin}}
\renewcommand{\finishchapter}{\pageref{berez-ch-end}}
\label{berez-ch-begin}



\thispagestyle{firststyle}

\chapauth{Andrea L. Berez-Kroeker}
\affiliation{University of Hawai‘i at Mānoa}

\authortoc{Andrea L. Berez-Kroeker}



\section{Introduction}\hspace{-0.3cm}\footnote{Many thanks to Jim Kari, Marianne Mithun and Sandy Thompson for their comments on this paper, parts of which also benefited greatly from discussion at seminars in the linguistics departments at the University of Melbourne and the University of Sydney. Errors herein are the fault of the author alone. Funding was provided by the University of California Pacific Rim Research Program, the American Philosophical Society, and the Jacobs Research Fund. This paper is meant to honor not only Jim, but also the memory of Markle Pete, whose warm heart and hearth will not be forgotten.}

This paper examines the role of culture in the grammar of motion events in Ahtna, with particular attention to the efficacy of “frog story” experiments in eliciting that grammar. Since the mid-1990s, linguistics has seen a proliferation of literature based on frog story experiments. In this research paradigm, speakers of various languages are shown a textless children’s book of drawings called \textit{Frog, Where are You?} \citep{Meyer1969} and are asked to narrate the events depicted there in their native tongue.\footnote{Frog story research developed out of the desire to make typological comparisons across languages. Because direct translations from a contact language can often distort the native grammar, one solution has been to present speakers of different languages with the same nonlinguistic stimulus in order to elicit spontaneous speech, which is more likely to reveal authentic grammatical constructions.} The story is about a boy and his dog who search through the woods for a pet frog that had escaped from its glass jar during the night. The boy and the dog undergo a series of adventures in which they encounter a host of forest dwelling creatures, including a squirrel, a swarm of angry bees, an owl, and a buck. The characters move from place to place, starting with a fall from the bedroom window and ending climactically with a ride atop the buck’s horns and a tumble from a cliff into a river.

Over the years, narratives resulting from frog story experiments have been used as the basis for inquiry into a wide range of topics in linguistics and psychology. Among these, the study of the expression of direction or path description in language – has garnered much attention. The research, summarized in \citet{Slobin2004}, stems from the work of  \citet{Talmy1985,Talmy1991,Talmy2000}, who presents a typology of the lexicalization of semantic units and the patterns of conflation of those units (i.e., \textsc{motion} \textsc{+} \textsc{manner,} \textsc{motion} \textsc{+} \textsc{path,} \textsc{motion} \textsc{+} \textsc{figure}). \citet{SV2004} is a collection of studies aiming to apply this typology to Warlpiri, Basque, Tzeltal, Icelandic, West Greenlandic, Swedish, Thai, American Sign Language, and Arrernte.

Among these, \citet{Wilkins2004} in particular finds that extralinguistic ethnographic considerations are very closely tied to grammar in his study of frog stories in Arrernte, a language of Central Australia. He shows that even within the genre of frog stories, Arrernte culture is a better predictor than language type of motion event segmentation. According to Slobin’s typology, one could hypothesize that Arrernte speakers devote less attention to the dynamic description of direction than would speakers of other types of languages \citep[see][]{Slobin2004}. Conversely, though, based on the nomadic culture of Desert Aborigines and the prominence of travel in everyday life, one could also hypothesize that Arrernte speakers will pay special attention to routes of motion and should construct elaborately detailed direction descriptions (see \citealt{Wilkins2004} for the argumentation). Wilkins shows that the latter is true: Arrernte speakers segment the direction description into more distinct trajectories in frog stories than English speakers. The difference is both quantitative in terms of path complexity and qualitative in terms of the kinds of linguistic structures used by speakers. As Wilkins writes, “[t]hus, it is the areal ethnographic observations … which here appear to be more predictive of the findings [i.e., than the typological predictions]” (2004: 155). The role of culture on the development of grammar should not be underestimated.

This paper examines the role that cultural considerations play in the grammar of motion events in Ahtna, and finds that frog stories are not the most effective tool for eliciting that grammar. The Ahtna community today consists of eight modern villages (Mentasta, Chistochina, Gakona, Gulkana, Tazlina, Copper Center, Chitina, and Cantwell) in the Copper River and Upper Susitna drainages of south central Alaska. In the 1980s and 1990 Jim Kari was responsible for the lion’s share of documentation and description of the language, and still continues to publish and work with speakers today.

Ahtna society is traditionally semi-nomadic. Hunters and family groups traveled seasonally in pursuit of resources like fish and big game (\citealt{Reckord1979}, \citeyear{Reckord1983a}, \citeyear{Reckord1983b}). Knowledge of the surrounding terrain is not only essential to survival but also plays an important role in ethnic identity and the assertion of the connection of one’s social group (tribe, band, family) to the land, much like \citet{MooreTlen2007} found for Athabascan speakers in the Yukon. Individual Ahtna men – and some women – are often intimately familiar with large swaths of the 35,000 square miles of Ahtna territory and beyond, a feat all the more impressive for having been undertaken on foot or by dogsled.

The importance of geographic knowledge and ‘travel talk’ is reflected in the sheer size of the corpus of Ahtna place names (\citealt{Kari1982}, \citeyear{Kari2008}). The corpus contains over 2,200 names, many of which are documented in a genre of oral literature that Kari terms \textit{elite travel narratives}. These narratives are a kind of “virtual guided tour” in which the speaker discusses, in sequential order, all the meaningful and hence named locations along a given route. A single narrative may cover over one hundred miles of river and/or trail and is often interspersed with personal memories and descriptions of how each site was used seasonally for camping and hunting (for published examples of Ahtna travel narratives see \citealt{Kari1986}; \citealt{KariFall2003}; \citealt{Kari2010}).

In some superficial ways, frog stories and Ahtna travel narratives are similar: animate referents move across the countryside in pursuit of animal(s). But in many ways, particularly cultural ways, they are different. In \textit{Frog, Where Are You?}, referents engage in activities that do not happen everyday: heads get stuck in jars, characters fall out of windows and trees and off cliffs, owls and gophers pop suddenly out of their holes, and characters interact with rather unfriendly bucks and bees. In travel narratives, on the other hand, activities are generally limited to walking, sledding, hunting, and camping.

As is discussed below, speakers telling travel narratives make full use of the grammar of path and location available to them, including adverbial verb prefixes, a class of riverine directionals, and highly systematic toponymy. Interestingly, while all of these (with the exception of possibly toponymy) are also available to frog story narrators, speakers in this genre seem to restrict themselves to only a narrow range.

What role, then, does genre (e.g. \citealt{Mayes2003}) play in an academic study of how a language encodes notions of direction and location in motion events? Can a frog story fully reveal the nature of Ahtna grammar about motion? Or will other concerns, specific to the tasks of telling a frog story or a travel narrative, be more important and ultimately influence where a speaker’s attention lies? The following sections examine two Ahtna travel narratives and an Ahtna frog story with an eye toward answering these questions.

The first travel narrative was recorded in 1980 by Jake Tansy with linguist James Kari. Mr. Tansy describes an overland and riverine hunting route, used exclusively in the summertime, from the mouth of Alaska’s Brushkana River to the Yanert Fork and onward to Valdez Creek. In Ahtna the story is called \textit{Saen} \textit{tah} \textit{xay} \textit{tah} \textit{c’a} \textit{łu’sghideł} ‘we used to travel around in summer and winter’ \citep[59-69]{Kari2010}; henceforth Mr. Tansy’s monologue is referred to as \textit{Saen} \textit{tah} \textit{xay} \textit{tah}  (‘during summer and winter’).

The second narrative was recorded by Adam Sanford in 1986, also with Kari. This is an epic description of yearly hunting routes, often in the extreme mountainous highlands in pursuit of Dall sheep. The entire recording is nearly thirty minutes in length; only the first five minutes are presented here. This narrative is known in Ahtna as \textit{C’uka} \textit{ts’ulaen’i} \textit{gha} \textit{nen’} \textit{ta’stedeł} \textit{dze’} ‘how we went hunting out in the country’ \citep[91-128]{Kari2010}. Henceforth this narrative is referred to as \textit{Ta’stedeł} \textit{dze’} (‘we went hunting thus’). Finally, the narration of \textit{Frog,} \textit{Where} \textit{Are} \textit{You?}---in Ahtna \textit{Naghaay,} \textit{ndaane} \textit{zidaa}, and henceforth \textit{Naghaay} (‘frog’)---was recorded in October 2008 in Tazlina, Alaska by Ahtna speaker Markle Pete with the author.

Section 2 below compares the distribution of the spatially oriented grammatical systems across the two genres. I first look at the use of direction- and location-describing adverbial verb prefixes, a mechanism that is employed by all three speakers. I then look at two areas of the grammar where the stories differ, the use of directionals and toponymy. While all three systems are fully utilized by travel narrators, Mr. Pete makes very limited use of the last two in \textit{Naghaay}.

The unequal use of spatial grammar in the two genres reflects the speakers’ unequal attention to figure and ground. Section 3 examines how the storytelling tasks that the speakers deem important influence the choices they make when structuring discourse. Mr. Tansy and Mr. Sanford foreground the spatial and temporal trajectories of their narratives, while Mr. Pete elects instead to carefully track referents, which is common in frog-story narration worldwide.

Section 4 contains concluding remarks about the role of genre in typology and language documentation. Ultimately it seems Mr. Pete’s concerns may not lie in creating a fully elaborated sense of the fictional landscape in \textit{Frog,} \textit{Where} \textit{Are} \textit{You?}, which in turn affects how well the story reveals Ahtna’s path-describing grammatical systems. As is discussed below, works of oral literature in Ahtna are often a-spatial.

\section{Grammar of Direction and Location in \textit{Saen} \textit{Tah} \textit{Xay} \textit{Tah,} \textit{Ta’stedeł} \textit{Dze’,} and \textit{Naghaay}}

This section describes three mechanisms for elaborating direction and location in Ahtna: adverbial verb prefixes, riverine directionals, and toponymy. It compares the travel narratives to the frog story in terms of each of these linguistic resources.

\subsection{Derivational/Thematic Adverbial Prefixes}

Ahtna, like all Athabascan languages, is a polysynthetic language with templatic verbal morphology. Table~\ref{berez:tab:1} shows a simplified version of the verb template from \citet{Kari1990}. Verbs are stem-final, with eleven prefix zones (further analyzable into up to twenty-eight individual slots) to the left of the stem. Near the far left edge, in position ten, we find the so-called ‘derivational/thematic’ prefixes. Many of the morphemes here are adverbial in function and can describe, among other things, path and location. Kari writes of the morphemes found here, “nearly one hundred morphemes appear in this position \ldots includ[ing] bound postpositions” (1990:40), which provides insight into the source of the spatial nature of these prefixes. The morphemes in this position that were historically free preverbal postpositions are grammaticalizing; they are undergoing phonological fusion to the verb, losing their objects, and becoming less adpositional and more adverbial in function.

\begin{table}[ht]\centering
\caption{Ahtna Verb Template (adapted from \citealt{Kari1990})}
\label{berez:tab:1}
\begin{tabular}{| *{12}{ l | }}
\hline
\rotatebox{90}{Bound postpositional object\ \ } &
\rotatebox{90}{Derivational/Thematic} &
\rotatebox{90}{Iterative} & \rotatebox{90}{Distributive} & \rotatebox{90}{Incorporate} & \rotatebox{90}{Thematic} & \rotatebox{90}{Pronominal} & \rotatebox{90}{Qualifiers} & \rotatebox{90}{Conjugation} & \rotatebox{90}{Subject} & \rotatebox{90}{Classifier} & \rotatebox{90}{Stem}\\

\hline
11 & 10 & 9 & 8 & 7 & 6 & 5 & 4 & 3 & 2 & 1 & \\
\hline
\end{tabular}
\end{table}


In \textit{Saen} \textit{tah} \textit{xay} \textit{tah}, \textit{Ta’stedeł} \textit{dze’}, and \textit{Naghaay}, all three speakers make extensive use of direction-describing adverbial prefixes. Of the seventy-eight motion verbs in the three stories, only four have no such prefix. Furthermore, the prefixes occur with uniform density. In \textit{Saen} \textit{tah} \textit{xay} \textit{tah} twenty-one of twenty-four motion verbs contain at least one direction- or location-describing prefix; the ratio in \textit{Ta’stedeł} \textit{dze’} is twenty-one of twenty-two motion verbs; and in \textit{Naghaay}, thirty-two of thirty-two motion verbs. These distributions do not differ significantly.

In contrast to the highly precise directional system, the adverbial prefixes usually describe simple paths of motion of the subject. In the examples from \textit{Saen} \textit{tah} \textit{xay} \textit{tah} and \textit{Ta’stedeł} \textit{dze’} shown in (1-2), the prefixes encode the simple directions of ‘away’, ‘out’, ‘through’, ‘back’, and ‘up’.

%ex1
\begin{exe}
  \ex   Direction-describing adverbial verb prefixes in \textit{Saen tah xay tah}\footnote{Transcription conventions and abbreviations are found in the Appendix.}\label{berez-ex1}
\begin{xlistn}
\exi{01} \gll Xona,\\
    now\\
\sn \begin{flushright} (0.6) \end{flushright}
\exi{02}  first
\begin{flushright} (0.9) \end{flushright}
\exi{03} \gll   nen’ \textbf{ta}’stghideł de c’a saen ta, \\
    country   1\textsc{pl.sub}.pl.go.\textbf{away}  when  \textsc{foc}    summer  during\\
\sn \begin{flushright} (0.9) \end{flushright}

\exi{04} \gll c’a  Bes Ggeze Na’, \\
    \textsc{foc}  bank   worn     stream.\textsc{pos}\\
\sn \begin{flushright} (0.4) \end{flushright}
\exi{05} \gll  Saas Nelbaay Na’,\\
    sand  grey      stream.\textsc{pos}\\
\exi{06} \gll hwcets’edeł. \\
    1\textsc{pl.sub}.ascend\\

\glt \textit{‘When we first went out in the country during the summer we would go to the base of ‘Worn Bank Stream’ or ‘Sand That is Grey Stream’.’}
\sn \begin{flushright} (1.5) \end{flushright}
\exi{07} \gll       Niłdenta łu,\\
    sometimes  \textsc{evid}\\
\sn \begin{flushright} (0.6) \end{flushright}
\exi{08} \gll  Dghateni    yi ’eł tanidzeh,\\
    stumbling.trail  3s  \textsc{conj}   middle.water\\
\exi{09} \gll      Dghateni \textbf{ts’i}diniłen.\\
    stumbling.trail  water.flows.\textbf{out}\\
\glt \textit{‘Sometimes also to ‘Stumbling Trail’ or the middle one flowing out from ‘Stumbling Trail’.’}

\sn  \ldots

\exi{43} \gll  dets’en,\\
    next\\

\exi{44}    \gll    Nts’ezi Na’ ba’aa,\\
    N.       stream.\textsc{pos}  outside\\

\exi{45}    \gll    dghilaay   \textbf{gha}kudaan de \textbf{ka}nats’edeł n'eł,\\
        mountain   hole.extends.\textbf{through} where \textsc{1pl.sub}.pl.go.\textsc{iter}.\textbf{up}  \textsc{conj} \\

\glt \textit{‘outside of ‘Nts’ezi’s Stream’ we went back up where a tunnel extends through the mountain, (and \ldots)’}

\end{xlistn}

\begin{flushright}
((Jake Tansy, \textit{Saen} \textit{Tah} \textit{Xay} \textit{Tah} \textit{C’a} \textit{Łu’sghideł\\
‘We} \textit{Used} \textit{to} \textit{Travel} \textit{Around} \textit{in} \textit{Summer} \textit{and} \textit{Winter’},\\
00:00:05.580-00:01:20.650. \citealt{Kari2010}:60-61))
\end{flushright}

\end{exe}

%ex2
\begin{exe}
 \ex Direction-describing adverbial verb prefixes in \textit{Ta’stedeł dze’}\label{berez-ex2}
 \begin{xlistn}
\exi{127} \gll {K’a xona} 	yet		hwts’en	xona	\textbf{na}’stetnaesi,\\
		then			there	from.area	then		1\textsc{pl}.travel.nomadically.\textbf{back}\\
\sn \begin{flushright}(2.3)\end{flushright}
\exi{128} \gll			ohh	dahwtnełdak. \\
		oh		3\textsc{s}.be.steep\\
		\textit{‘Then as we moved back from there, oh it (the canyon) was steep.’}
\sn \begin{flushright}(0.6)\end{flushright}
\exi{129} \gll			Niłk’aedze’	dahwtnełdak	xona, \\
		both.sides		3\textsc{s}.be.steep		then\\
\sn \begin{flushright}(0.8)\end{flushright}
\exi{130}	\gll		saanetah	kats’enaes. \\
		barely		1\textsc{pl}.trave.nomadically.up\\
\glt	\textit{	‘It was steep on both sides and then we could barely move up.’}

\end{xlistn}
\begin{flushright}
((Adam Sanford, \textit{C’uka Ts’ul’aen’i gha Nen’ Ta’stedeł dze’}\\
\textit{‘How We Went Hunting Out in the Country’},\\
00:03:56.740-00:04:06.320. Kari 2010:96))
\end{flushright}
\end{exe}

\noindent
The semantic generality of the prefixes allows them to be used in a range of situations, not only describing a journey across the countryside as in (1-2), but also for the up-and-out motion of a squirrel from his den and the downward orientation of a bees’ nest hanging from a tree branch, as in the excerpt from \textit{Naghaay} in (\ref{berez-ex3}).


%ex3
\begin{exe}
\ex Direction-describing adverbial verb prefixes in \textit{Naghaay}\label{berez-ex3}
\begin{xlistn}
\exi{13} \gll	Łic’ae	gilok'ae	naghalts’et.\\
		dog		window		3S.falls.down\\
	\glt	\textit{‘The dog falls down from the window.’}
\sn {[ \ldots ]}
\exi{22}	\gll			Dligi	kaniyaa,\\
		squirrel	3S.go.up.and.out\\
\glt	\textit{	‘A squirrel comes up,’}
\exi{23} \gll			łic’ae	ngga	t’ox 	naggic’eł’i			gha		itsae.\\
		dog		upland	nest		3S.hang.down.REL	at		3S.bark\\
\glt		\textit{‘the dog barks at the nest that is hanging (there).’}
\end{xlistn}
\begin{flushright}
((Markle Pete, \textit{Naghaay Ndaane Zidaa} ‘Frog Where Are You’,\\
oai.paradisec.org.au:ALB01-AHTNB001-060.tiff,\\
ALB01-AHTNB001-061.tiff))
\end{flushright}
\end{exe}

\noindent
Based on the uniform density of spatially oriented prefixes in all three narratives, it would be reasonable to claim that frog story experiments do indeed exemplify how Ahtna speakers use prefixes to describe direction concepts in discourse. However, other data shows that \textit{Naghaay} does not provide us with a complete picture of how richly speakers can describe direction and location in Ahtna. The next section discusses the use of the directional system that is employed extensively in the travel narratives, and only minimally in \textit{Naghaay}.\footnote{It should be made clear that the reason Mr. Pete makes only limited use of the grammar of path and location because he is that he is expertly attending to other concerns, and not because he is in any way unfamiliar with his language.}

\subsection{Directionals}

Like other Athabascan languages, Ahtna has a set of directionals that can be defined as a separate lexical class based on their morphological behavior. Ahtna directionals have a tripartite structure shown in Table~\ref{berez:tab:2}: a stem expressing orientation (a system that is largely, but not completely, riverine; ‘up’ and ‘down’ are included in the paradigm on morphological grounds), an optional prefix expressing relative distance, and an optional suffix that expresses either a point-versus-area distinction or a path toward or away (see \citealt{Kari1985}, \citeyear{Kari1990}; \citealt{Leer1989}; \citealt{MooreTlen2007}).

\begin{table}
\centering
\caption{Tripartite structure of Ahtna directionals \citep[from][]{Kari1990}}
\label{berez:tab:2}
\begin{tabular}{ >{\raggedright\arraybackslash}p{4cm} | >{\raggedright\arraybackslash}p{4cm} | >{\raggedright\arraybackslash}p{4cm} }
\textbf{Prefixes} & \textbf{Stems} & \textbf{Suffixes}\\
\hline
\makecell[c{p{4cm}}]{
\textit{da-} ‘near’\\
\textit{na-} ‘intermediate distance’\\
\textit{’u-} ‘distant’\\
\textit{ts'i-} ‘straight, directly’\\
\textit{ka-} ‘next to’\\
\textit{P+gha-} ‘from P’\\
\textit{n-} ‘neutral’\\
\textit{hw-} ‘area’
}

&

\makecell[l]{
\textit{nae’} ‘upriver, behind’\\
\textit{daa’} ‘downriver’\\
\textit{ngge’} ‘from water, upland’\\
\textit{tsen} ‘toward water, lowland’\\
\textit{naan} ‘across’\\
\textit{tgge’} ‘up vertically’\\
\textit{igge’,} yax ‘down vertically’\\
\textit{’an} ‘away, off’\\
\textit{nse’} ‘ahead’
}
&
\makecell[l]{
\textit{-e} ‘to’\\
\textit{-dze} ‘from’\\
\textit{-t} ‘at a point’\\
\textit{-xu} ‘in a general area’\\ \\ \\ \\
}

\end{tabular}\end{table}


Examples of fully inflected directionals are found in (\ref{berez-ex4}).

\begin{exe}
\ex Examples of fully inflected directionals \label{berez-ex4}

\begin{xlist}


\ex \gll \textit{’u-ngge} \\
distant-upland  \\
\glt ‘distantly upland’


\ex \gll \textit{na-naa} \\
intermediate.distance-across  \\
\glt ‘an intermediate distance across’

\ex \gll \textit{’u-naa-ts’en}\\
distant-across-from \\
\glt ‘from distantly across’

\ex \gll \textit{’u-tsuu-ghe}\\
distant-lowland-general.area \\
\glt ‘in a general area distantly lowland’

\ex \gll ’u-ngga-t\\
distant-upland-at.point \\
\glt ‘at a point distantly upland’

\ex \gll \textit{ka-naa}\\
general.area-across \\
\glt ‘an area across’

\ex \gll \textit{’u-tsii-t}\\
distant-lowland-at.point \\
\glt ‘at a point distantly lowland/downland’

\end{xlist}

\end{exe}

\noindent
The morphological structure of the lexical class of directionals potentially allows speakers to be very precise in describing paths and locations in terms of their relationship to the placement and flow of the local river. In addition, the structure of Ahtna discourse allows even more specificity. Speakers very often use multiple directionals in a single clause to describe changes in trajectory or to pinpoint a destination precisely.\footnote{For more on the use of directionals in Ahtna discourse, see \citet{Berez2011,Berez2014}.} In (\ref{ex:key:5}), Mr. Tansy uses multiple directionals, in addition to adverbial prefixes, to describe a complex path with several trajectory changes (downriver – across – downriver – back).

%ex5
\begin{exe}
\ex Use of directionals in \textit{Saen tah xay tah} \label{ex:key:5}
\begin{xlistn}
\exi{25} \gll 		Niłdenta	łu’,\\
		sometimes	\textsc{evid}\\
\sn \begin{flushright}(0.7)\end{flushright}
\exi{26} \gll 				yet,\\
		there\\
\sn \begin{flushright}(0.8)\end{flushright}
\exi{27}	\gll 			Tl’ahwdicaax 			Na’, \\
		headwaters.be.valuable	stream.\textsc{pos}\\
\sn \begin{flushright}(0.4)\end{flushright}
\exi{28}	\gll 			’udaa’a,\\
		distant.downriver \\
\sn \begin{flushright}(0.5)\end{flushright}
\exi{29}	\gll 			’unaa			daa’a		ts’its’edeł	dze’	dae’, \\
		distant.across	downriver	1PL.go.out			thus\\
\exi{30}	\gll 			Nts’ezi		Na’			hwts’e’, \\
		N.			stream.\textsc{pos}	from.area\\
\sn \begin{flushright}(0.6)\end{flushright}
\exi{30}	\gll 			tes 		ninats’edeł.\\
		pass		1\textsc{pl}.go.back.to.a.point\\
\glt \textit{‘Sometimes then, we come out downstream and across and downstream of ‘Valuable Headwaters Stream’ and we come back to a pass at ‘Nts’ezi’s Stream’.’}
\end{xlistn}

\begin{flushright}
((Jake Tansy, \textit{Saen Tah Xay Tah C’a Łu’sghideł}\\
\textit{‘We Used to Travel Around in Summer and Winter’},\\
00:00:46.710-00:00:57.290. Kari 2010:61))
\end{flushright}
\end{exe}

\textit{Saen} \textit{tah} \textit{xay} \textit{tah} contains twenty-two directionals over 100 lines, and \textit{Ta’stedeł} \textit{dze’} contains twenty-six directionals over 218 lines. \textit{Naghaay}, however, contains only five (over 55 normative sentences). Of the handful of directionals found in \textit{Naghaay}, three occurrences are the use of \textit{ngge’} ‘upland’ to describe the location of the bees’ nest. The nest is not actually upland from a river – in these pages there is no river to be seen – but the term is used idiomatically to mean ‘over there’ or ‘up that-a-way’. Mr. Pete translated the line in (\ref{berez-ex6}) with a small backhand wave gesture:


%% kludge to  prevent page break in next example
\clearpage
%ex6
\begin{exe}
\ex Use of ngge’ ‘upland’ in \textit{Naghaay}\label{berez-ex6}
\begin{xlistn}
\exi{19}\gll Ngge’	t’ox 	naggic’eł’.\\
		upland	nest		3S.hang.down \\
\glt		\textit{‘Up (over there) the nest is hanging.’}
\end{xlistn}
\begin{flushright}
((Markle Pete, Naghaay Ndaane Zidaa ‘Frog Where Are You’,\\
oai.paradisec.org.au:ALB01-AHTNB001-064.tiff))
\end{flushright}
\end{exe}


\noindent
In (\ref{berez-ex7}) the nonriverine directional \textit{’utgge’} ‘distantly up vertically’ is used to describe the owl’s perch.


\begin{exe}
\ex Use of ’utgge’ ‘distantly up vertically’ in \textit{Naghaay}\label{berez-ex7}
\begin{xlistn}
\exi{39}	\gll	Besiini 	’utgge’				dazdaa.\\
		owl			distant.up.vertically	3S.sit\\
	\glt \textit{`The owl is sitting up there (i.e., high up, on a branch).’}
\end{xlistn}

\begin{flushright}
((Markle Pete, \textit{Naghaay Ndaane Zidaa} ‘Frog Where Are You’,\\
oai.paradisec.org.au:ALB01-AHTNB001-064.tiff))
\end{flushright}
\end{exe}

\noindent
The final directional in \textit{Naghaay} is \textit{’unaan} ‘distantly across’. The lexical interpretation of this term is riverine, but Mr. Pete uses it in a nonriverine situation (‘across the grass’) when the boy finally finds the frog behind a log in (\ref{berez-ex8}):

%ex8
\begin{exe}
\ex Use of \textit{’unaan} ‘distantly across’ in \textit{Naghaay}\label{berez-ex8}
\begin{xlistn}
\exi{54} \gll ’Unaan	tl’ogh	ta 		naghaay	c’a		’uka~nasitelyaesi, 	kuts’e’		niłc’ayiłyaał. \\
		distant.across	grass	in		frog			\textsc{foc} 	3\textsc{pl}.look.for.it.\textsc{rel}	to.them		3S.jump \\


\glt \textit{‘The frog they are looking for is jumping across to them on the grass.’}
\end{xlistn}
\begin{flushright}
((Markle Pete, \textit{Naghaay Ndaane Zidaa} ‘Frog Where Are You’,\\
oai.paradisec.org.au:ALB01-AHTNB001-066.tiff))
\end{flushright}
\end{exe}

\noindent
The use of directionals in the two genres differs in both frequency and literalness. This is attributable to the deictic nature of the directional system. In \textit{Saen} \textit{tah} \textit{xay} \textit{tah} and \textit{Ta’stedeł} \textit{dze’}, Mr. Tansy and Mr. Sanford are taking their listeners on a verbal tour through the Alaskan countryside. Because they are describing physical geographic locations in relation to a real river, these speakers use directionals frequently and assign them a literal interpretation.

In \textit{Naghaay}, however, Mr. Pete uses directionals infrequently, and they are either interpreted idiomatically or they are limited to the nonriverine uses of ‘up vertically’ and ‘across’. Although there is a river in the fictitious landscape of \textit{Frog} \textit{Where} \textit{Are} \textit{You,} readers have no real awareness of its spatial relationship to the boy’s trek through the forest until the end of the book. For Mr. Pete to describe the boy’s direction of travel in terms of the flow of the river would make little sense without a mental image of its location. Even in the last few pages where Mr. Pete could have used riverine directionals literally—-specifically the tumble into the water and the boy’s subsequent climb onto dry land—-he chooses not to do so.

We can already see from the use of directionals that \textit{Naghaay} lacks some of the vivid elaboration of direction and location found in the travel narratives. The next section briefly discuss toponymy, another linguistic domain where travel narratives are more richly elaborated for space and location than \textit{Naghaay}. As is discussed, works of oral literature are often a-geographic in Ahtna.

\subsection{Toponymy}

For their length, the travel narratives contain an impressive number of locations referenced by their Ahtna toponyms. Mr. Sanford gives twenty-five tokens of place names over 218 lines in \textit{Ta’stedeł} \textit{dze’}, and Mr. Tansy gives thirty-one in \textit{Saen} \textit{tah} \textit{xay} \textit{tah} over 100 liness. Not surprisingly, Mr. Pete, when describing the fictional scenery of \textit{Naghaay}, gives none.

The reasons for the unequal distribution of toponymy between stories about the Alaskan countryside and the narration of a children’s book may be self-evident, but nonetheless the use of place names is vital to creating a sense of space and location in the travel narratives. While the two genres examined here do not readily lend themselves to meaningful comparisons of toponym distribution, below I highlight a few points about Ahtna toponymy to draw attention to its systematicity. Kari has written extensively on the cognitive and linguistic traits of Ahtna toponymy and its role in Ahtna geographic knowledge (e.g., \citealt{Kari2008}, 2011), and the reader is referred to those sources for more information.

Like other aspects of geographic knowledge, mastery of place names occupies a privileged position in Ahtna culture and identity. Kari stresses that speakers’ attitudes toward the names are consistently cautious and conservative. During his years spent documenting Ahtna toponyms, Kari found that speakers would prefer to leave a feature unnamed rather than guess about a name they were unsure about. Traditionally place names were taught with extreme care and memorized in the sequence in which one would come across them when traveling along a river or trail. Naming is also extremely conservative: new names are rarely coined, never borrowed from non-Athabascan languages, and very frequently shared across language boundaries with neighboring Athabascan groups \citep{Kari2008}.

The names are structurally systematic and follow a limited number of conventions. Nearly a third of the corpus is nominalized verbs, and nearly two-thirds are binomial or trinomial constructions consisting of a specific noun and one or more generic nouns (e.g., lake, hill, river), as in \textit{Kaggos Bene’} ‘swan lake’. Similar names tend to cluster geographically, such that features in the environs of another more prominent feature, for example a hill or a lake, will show a recursive naming pattern based on the name of the prominent feature. \citet{Kari2008} provides an example of clustering in a set of names in the Reindeer Hills area near Denali National Park. \textit{Yidateni} ‘jaw trail’ is the name given to the visually prominent West Reindeer Hill. Also in the region are an array of features physiographically related to, and taking their names from, \textit{Yidateni}: \textit{Yidateni} \textit{Dyii} ‘canyon of jaw trail’, \textit{Yidateni} \textit{Dyii} \textit{Dghilaaya’} ‘mountain of canyon of jaw trail’, \textit{Yidateni} \textit{Caek’e} ‘mouth of jaw trail’, \textit{Yidateni} \textit{Caek’e} \textit{Tes} ‘hill at mouth of jaw trail’, \textit{Yidateni} \textit{Tl’aa} ‘headwaters of jaw trail’, \textit{Yidateni} \textit{Tl’aa} \textit{Bene’} ‘lake at headwaters of jaw trail’ (2008:27). Mr. Sanford’s narrative displays some of this toponymic clustering when he talks about locations near his birthplace at the mouth of the Sanford River. In \textit{Ta’stedeł} \textit{dze’} he names \textit{Ts’itaeł} \textit{Na’} ‘river that flows straight’, \textit{Ts’itaeł} \textit{Na’} \textit{Ngge’} ‘uplands of river that flows straight’, \textit{Ts’itaeł} \textit{Caegge} ‘mouth of river that flows straight’, and \textit{Ts’itaeł} \textit{Tl’aa} ‘headwaters of river that flows straight’.

In \textit{Ta’stedeł} \textit{dze’} and \textit{Saen} \textit{tah} \textit{xay} \textit{tah}, place names function to orient the listener to the appropriate geographic region, and the directionals and adverbial prefixes create a network of paths of motion between them. For speakers and listeners, travel narratives index a shared knowledge of Ahtna territory, but if listeners are not personally familiar with a location, the systematicity of Ahtna toponymy allows them to imagine it. Even if one has never seen the river known as \textit{Ts’itaeł} \textit{Na’}, one understands immediately that \textit{Ts’itaeł} \textit{Na’} \textit{Ngge’} is the name of its uplands, and that \textit{Ts’itaeł} \textit{Caegge} is the name of its mouth. The same is often not true of English place naming conventions. One cannot imagine the physiographic relationship between \textit{Yidateni} and \textit{Yidateni} \textit{Na’} based on their English names alone (West Reindeer Hill and Jack River, respectively).

The absence of place names in \textit{Naghaay}, on the other hand, is typical of works of fiction, known as \textit{yenida’a}, in Ahtna oral literature. Kari observes:

\begin{quote}
``It is quite noticeable that Ahtna \textit{yenida’a} myths with human-animal interaction are \textit{ageographic} \textit{and} \textit{always} \textit{lack} \textit{place} \textit{names} or any local geographic references. For example, the collection of \textit{yenida’a} stories by Jake \citet{Tansy1982} contain[s] no place names and can be considered as pure fiction. On the other hand, the presence of place names in narratives appears to be the mark of Ahtna non-fiction. The clan-origin stories, the pre-contact incidents … when two groups of Russians are killed, as well as much earlier regional war incidents \citep{Kari1986}, are non-fiction, prehistoric events that take place at specific places.'' (\citealt{Kari2008}:28; emphasis original)\footnote{Kari notes that \textit{yenida’a} do contain directionals even though they are lacking place names: “The full nine-point system is used, even when the landscape is left to the imagination” (p.c.).}
\end{quote}

We have seen that the speakers in \textit{Ta’stedeł} \textit{dze’} and \textit{Saen} \textit{tah} \textit{xay} \textit{tah} make full use of all three systems described above, while the speaker in \textit{Naghaay} fully exploits just the adverbial prefixes, makes only limited use of directionals, and does not need to use the toponymic system. \textit{Naghaay} conforms to the a-geographic landscape we expect from Ahtna fiction, but the source of the difference in spatial elaboration between it and the travel narratives goes beyond a simple dichotomy between fiction and non-fiction. Travel narrators and frog-story narrators also have different narrative tasks, which is reflected in their relative attention to figure and ground; that is, to the animate referents in the stories and the landscape across which they travel. The next section discusses how the importance the speakers place on figure and ground is manifested in differing discourse strategies.

\section{Attention to Narrative Tasks}

Mr. Sanford’s and Mr. Tansy’s richly developed sense of place is consistent with the sociocultural function of telling travel narratives, which is to index a speaker’s intimacy with the land and, by extension, the entitlement of the speaker’s social group to the resources found there. Speakers pay a great deal of attention to constructing a sophisticated ground in their stories, and then they move figures across that ground in a predetermined sequence. The figures are not particularly differentiated (generally limited to first person plural), but their sequential progress along the described routes and through the timeline of the story is essential to the task of telling a travel narrative.

Mr. Pete, on the other hand, is not required to fully elaborate the ground in order to tell \textit{Naghaay}. His focus is clearly on the figures. The cast of characters here is varied and unusual – the boy and the dog are joined by highly agentive wild animals – and their interactions with one another are the focus of the story. Details about their path through the forest are unimportant.

The three speakers employ different strategies that reveal what each considers crucial to the task of telling his story. For Mr. Sanford, progression through the physical landscape is important, which is reflected in his use of the deictic postpositional phrase \textit{yihwts’en} ‘from there’ as a discourse connector. Mr. Tansy is attuned to the temporal progression of his story, signaled by his use of \textit{xona} ‘then’ to connect episodes in his story. Finally, Mr. Pete is most concerned with tracking individual referents in \textit{Naghaay}, which he accomplishes via the use of relative clauses.

\subsection{Discourse Use of the Postpositional Phrase \textit{yihwts’en}}

Observe the excerpt in (\ref{berez-ex9}), in which Mr. Sanford repeatedly uses the postpositional phrase \textit{yihwts’en} ‘from there’ (glossed \textit{yi-hw-ts’en} ‘there-area-from’).

%ex9
\begin{exe}
\ex Use of \textit{yihwits’en} ‘from there’ in \textit{Ta’stedeł dze’}\label{berez-ex9}
\begin{xlist}
\exi{30}  	Duu \textbf{yihwts’en},
\exi{31} \		xona ’unggat,
\sn \begin{flushright}(0.4)\end{flushright}
\exi{32}  			Natii Caegge,
\exi{33}  			yedu’ xona,
\sn \begin{flushright}(1.1)\end{flushright}
\exi{34}  			yetdu’ xona nits’edeł.
\glt	\textit{‘\textbf{From there}, then over to ‘Natii Mouth’, then, we would stop there.’}
\sn \begin{flushright}(0.6)\end{flushright}
\exi{35}  			\textbf{Yihwts’en} xona Natii Na’ Ngge’,
\sn \begin{flushright}(2.3)\end{flushright}
\exi{36}  xona ’utgge yii,
\exi{37}   			xungge’ de kudełdiye.
\glt	\textit{‘\textbf{From there}, then in ‘Natii River Uplands’, then above there and the uplands are a short distance.’}
\exi{38} 			About,
\sn \begin{flushright}(3.1)\end{flushright}
\exi{39}   			nduugh miles kulaen.
		\glt	\textit{‘How many miles is it?’}
\exi{40}   			Seven,
\exi{41}   			eight miles,
\exi{42}   			I guess.
\sn \begin{flushright}(1.7)\end{flushright}
\exi{43}   			Yet su xona,
\sn \begin{flushright}(0.3)\end{flushright}
\exi{44}   			debae ka ’stedeł.
		\glt	 \textit{‘There we went for sheep.’}
\sn \begin{flushright}(1.3)\end{flushright}
\exi{45}   			Debae ts’eghaax.
\sn \begin{flushright}(0.7)\end{flushright}
\exi{46}   			Gha yak’a.
		\glt 	\textit{‘We would kill sheep right there.’}
\sn \begin{flushright}(1.7)\end{flushright}
\exi{47} 	Yii kaen’,
\exi{48}   			taade yet ’sneyeł.
		\glt 	\textit{‘We stayed there three days (living) on it.’}
\sn \begin{flushright}(2.0)\end{flushright}
\exi{49}   			Du’ \textbf{yihwts’en},
\sn \begin{flushright}(0.9)\end{flushright}
\exi{50}   			ts’inats’edeł dze’ ’ungge.
	\glt	\textit{‘\textbf{From there}, them we would start out again to uplands.’}
\sn \begin{flushright}(0.9)\end{flushright}
\exi{51} 		’Utggu daagha ngge’,
\sn \begin{flushright}(1.8)\end{flushright}
\exi{52}   			ngga Ts’itaeł Tl'aa ts’e’,
\glt	\textit{‘Up above the treeline upland to ‘Headwaters of River That Flows Straight’,’}
\sn \begin{flushright}(1.9)\end{flushright}
\exi{53}  			\textbf{yihwts’en} ’unggat,
\sn \begin{flushright}(1.7)\end{flushright}
\exi{54}   			Tsaani ’Aeł Na’,
\sn \begin{flushright}(0.7)\end{flushright}
\exi{55}   			yet kets’edeł.
		\glt \textit{‘\textbf{from there} on upland we reached ‘Bear Trap Creek’.’}
\sn \begin{flushright}(1.4)\end{flushright}
\exi{56} 			Yet kanaa,
\sn \begin{flushright}(1.0)\end{flushright}
\exi{57}   			debae una’ c’ilaen,
\exi{58}   			you know.
		\glt	\textit{‘Across from there, there are sheep on that creek, you know.’}
\sn \begin{flushright}(0.9)\end{flushright}
\exi{59}   			Yet cu debae ka łu’stedeł.
	\glt	\textit{	‘There we would hunt again for sheep.’}
\sn \begin{flushright}(0.6)\end{flushright}
\exi{60}   			Debae ts’eghaax,
\exi{61}   			you know.]
		\glt	\textit{‘We would kill sheep, you know.’}
\sn \begin{flushright}(1.0)\end{flushright}
\exi{63}   			Ye naxaełts’eldeli kae,
\sn \begin{flushright}(0.8)\end{flushright}
\exi{64}   			taade nk’e ye ts’eneyeł.
\glt	\textit{‘With what we were packing back, we would camp there three days.’}
\sn \begin{flushright}(1.4)\end{flushright}
\exi{65}   			Duu \textbf{yihwts’en} xona,
\sn \begin{flushright}(0.4)\end{flushright}
\exi{66}   			Natii Na’,
\sn \begin{flushright}(0.8)\end{flushright}
\exi{67}   			Ts’itaeł Na’,
\exi{68}   			kanats’edeł.
\glt \textit{‘\textbf{From there} we would go back to ‘Natii River’ and to ‘River That Flows Straight’.’}

\end{xlist}
\begin{flushright}
((Adam Sanford, \textit{C’uka Ts’ul’aen’i gha Nen’ Ta’stedeł dze’}\\
\textit{‘How We Went Hunting Out in the Country’},\\
00:01:12.440-00:02:17.870. Kari 2010:93-95))
\end{flushright}
\end{exe}

The postpositional phrase \textit{yihwts’en} ‘from there’ here has a discourse function. It is used to mark clauses as belonging to the main storyline, which is a listing of the places on the hunting route Mr. Sanford and his cohort followed. As he names individual locations, he often digresses to give background information. For example, in lines 38-48, he first contemplates the distance to the uplands from the location he has just named, and then mentions that his group would kill sheep and camp there for three days. He then resumes the main storyline of the journey and introduces the next two locations, each with \textit{yihwts’en}, in lines 49-55. He again provides background information about site usage in lines 56-64, and then continues along the path to the next location, which is again introduced with \textit{yihwts’en} in line 65. Each of the twelve occurrences of \textit{yihwts’en} in the entire \textit{Ta’stedeł} \textit{dze’} is used in this way: the discourse use of this postpositional phrase “gets the characters moving” from place to place, after Mr. Sanford has departed from the main events of the story to talk a bit about each location. This discourse use of a spatially-oriented postposition with a deictic demonstrative pronoun as its object highlights the spatial nature of the main storyline, and allows Mr. Sanford to link episodes of the story together against the backdrop of the natural landscape.

\subsection{Discourse Use of the Adverb \textit{xona}}

While Mr. Sanford highlights the spatial ordering of episodes in \textit{Ta’stedeł} \textit{dze’}, in \textit{Saen} \textit{tah} \textit{xay} \textit{tah} Mr. Tansy chooses instead to highlight temporal ordering.\footnote{It is not that Mr. Sanford ignores temporal progression; to the contrary, he uses \textit{xona} ‘then’ frequently as well. In contrast, however, Mr. Tansy exclusively uses \textit{xona}. Furthermore, \textit{yihwts’en} ‘from there’ in \textit{Ta’stedeł} \textit{dze’} and \textit{xona} ‘then’ in \textit{Saen} \textit{tah} \textit{xay} \textit{tah} pattern together in terms of their prosody, suggesting a commonality of function. They tend to occur in intonation unit-initial position, whereas \textit{xona} in \textit{Ta’stedeł} \textit{dze’} tends to occur in the middle of intonation units. Prosodic indications of the discourse use, as opposed to the lexical use, of these items warrants further exploration.} He does this by linking episodes in his narrative with a discourse use of the sequentially-oriented adverb \textit{xona} ‘then.’ This word is clearly related to time; when it is not being used in a episode-tying function, its lexical meaning is ‘now’. Observe the use of \textit{xona} in (\ref{berez-ex10}).

%ex10
\begin{exe}
\ex Use of \textit{xona} ‘then’ in \textit{Saen tah xay tah}\label{berez-ex10}
\begin{xlistn}
\exi{01} 	Xona,
\sn \begin{flushright}(1.0)\end{flushright}
\exi{02} 			first,
\exi{03} 			nen’ ta’stghideł de’ c’a saen ta,
\sn \begin{flushright}(0.9)\end{flushright}
\exi{04} 			c’a Bes Ggeze Na’,
\exi{05} 			Saas Nelbaay Na’,
\exi{06}   			chwcets’edeł.
\glt \textit{‘When we first went out in the country during the summer we would ascend ‘Worn Bank Stream’ or “Sand That is Grey Stream’.’}

\sn {[26 lines about eight locations before reaching ‘Nts’ezi Stream’]}

\exi{32} 	 		Nts’ezi Na’,
\exi{33} 	 		cu yet cu tcenyii kughił’aen’,
\exi{34} 	 		I mean dahtsaa,
\exi{35} 	 		dahtsaa,
\exi{36}   				hwghił’a’.
\glt \textit{‘At ‘Nts'ezi Stream’ was an underground cache, I mean they had a raised cache.’}
\exi{37} 	 		Teye k’a ’udii,
\exi{38} 	 		c’etsen’,
\exi{39}   	 		nkghiłggaasi dahtsaa t’anahghilaes.
\glt \textit{‘The meat they had put (there) was enclosed in the pole cache.’}
\exi{40} 	 		Xona ye łu Nts’ezi Na’,
\exi{41}   	 		ye kae na'sdelgges dze’,
\glt \textit{‘Then we would come back with that (meat) on ‘Nts'ezi Stream’ and,’}
\exi{42} 				dets’en,
\exi{43} 	 		dets’en,
\exi{44} 	 		Nts’ezi Na’ ba’aa,
\exi{45} 	 		dghilaay ghakudaan de kanats’edeł n’eł,
\exi{46}   	 		Bes Ggeze Na’.
\glt \textit{‘outside of ‘Nts’ezi Stream’ we went back up where a tunnel extends through the mountain, and after ‘Nts'ezi Stream’ we would ascend back up through a canyon in the mountains, and at 'Worn Bank Stream’.’}

\sn {[eight lines about area around ‘Worn Bank Stream’ and ‘Sand That is Grey Stream’]}

\exi{55} 			Saas Nelbaay Na’ ngge’,
\exi{56} 			cu ye xona ba’aa,
\exi{57} 			łu- Łuyinanestaani Na’,
\exi{58}   			su hwdedaa’ kanats’edeł.
\glt \textit{‘Upland of ‘Sand That is Grey Stream’ then again out there we would ascend the downstream area of ‘Stream of the One Protruding Into the Glacier’.’}
\sn \begin{flushright}(2.0)\end{flushright}
\exi{59} 			Łuyinanest'aani Na’,
\exi{60} 			yanaasts'en ’uk’atl’adaak’e cu,
\exi{61} 			Ts’es Ce’e de gaa hwnax,
\exi{62} 			gaani,
\exi{63}   			dighiłcaax xu dez’aan.
\glt \textit{‘On the other side of ‘Stream of the One Protruding Into the Glacier’ is also a bluff ‘Big Rock’ that is as large as this [Jake’s] house.’}
\sn \begin{flushright}(0.5)\end{flushright}
\exi{64} 			Ye su xona ’udii,
\sn \begin{flushright}(0.7)\end{flushright}
\exi{65}   			hw’eł hnats’at’iix,
\glt \textit{‘We always used to play there,’}
\exi{66} 			hwghak’aay,
\exi{67}   			hw’eł łu’steltset cu @’snakaey @ts’ghile’ @de @yet.
\glt \textit{‘we would run around on the ridge (of the rock) when we were kids.’}
\sn \begin{flushright}(1.1)\end{flushright}
\exi{68}   			Yak’a k’adii c’edez’aan.
\glt \textit{‘It is still sitting there.’}
\sn \begin{flushright}(1.0)\end{flushright}
\exi{69} 			Xona yet łu’ ye c’a ye łu Łuyinanest'aani Na’,
\sn \begin{flushright}(1.4)\end{flushright}
\exi{70} 			tsen,
\exi{71}   			tsen tene kana’sghideł.
\glt \textit{‘Then there at ‘Stream of the One Protruding Into the Glacier’, we go back up to the lowland trail.’}

\end{xlistn}
\begin{flushright}
((Jake Tansy, \textit{Saen Tah Xay Tah C’a Łu’sghideł}\\
\textit{‘We Used to Travel Around in Summer and Winter’},\\
00:00:05.580-00:02:08.170. Kari 2010:60-63))
\end{flushright}
\end{exe}


Like Mr. Sanford’s use of \textit{yihwts’en} ‘from there’, Mr. Tansy’s use of \textit{xona} ‘then’ marks transitions between the main storyline of the sequences of arrivals at different locations on the one hand, and digressions about site use and personal memories on the other. The story starts with \textit{xona} (perhaps best translated here at ‘first’), then in lines 1-31 Mr. Tansy names ten locations. He digresses in lines 32-39 to talk about a meat cache. He resumes the storyline with \textit{xona} in line 40, where the next event is the return via ‘\textit{Nts’ezi} Stream’ with meat from the cache. The arrival at ‘Stream of the One Protruding Into the Glacier’ is marked with \textit{xona} in lines 55-58, followed by ten lines of personal recollections. Again, the journey is resumed in line 69 with \textit{xona}.

\subsection{Tracking Referents with Relative Clauses}

The discourse strategies of the travel narrators underline the importance they place on ground, as opposed to figure. The digressions consistently provide background information about locations, rather than about the people traveling through them. In fact, the travel narrators give very little information about the characters in these stories, and almost exclusively refer to them with subject prefixes only. Mr. Pete, on the other hand, is far more concerned with figure than with ground in his frog-story narrative. He attends carefully to the task of tracking characters as they interact with each other in the minimally defined landscape of \textit{Naghaay}. He does this most notably by using relative clauses to refer to and delimit referents that have already been introduced. See (\ref{berez-ex11}).

%ex11
\begin{exe}
\ex Use of relative clauses in \textit{Naghaay}\label{berez-ex11}
\begin{xlistn}

\exi{19} \gll 	Ngga	t’ox 	naggic’eł’.\\
upland	nest		3S.hang.down\\
\glt \textit{Up (over there) the nest is hanging.’}

\exi{20} \gll 			Ciił 	c’e’an	ugha	niyaa.\\
boy		den		to.it		3S.come\\
\glt \textit{The boy comes to a den.’}

\exi{21} \gll 			Łic’ae	ngga	t’ox 	\textbf{naggic’eł’i	}		gha		itsae.\\
dog		upland	nest		3s.hang.down.\textsc{rel}	at		3S.bark\\
\glt \textit{'The dog barks at the nest that is hanging (there).'}

\exi{22} \gll 			Dligi 		kaniyaa,\\
squirrel		3S.go.up.and.out\\
\glt \textit{A squirrel comes up,’}

\exi{23} \gll 			łic’ae	ngga	t’ox 	\textbf{naggic’eł’i	}		gha		itsae.\\
dog		upland	nest		3S.hang.down.REL	at		3S.bark\\
\glt \textit{the dog barks at the nest that is hanging (there).’}

\exi{24} \gll 			Dligi	c’a		\textbf{kaghiyaani},\\
squirrel \textsc{FOC} 3S.go.up.\textsc{rel}\\

\exi{25} \gll 			łic’ae	hnał’aen’.\\
dog		3S.see\\
\glt \textit{The squirrel who came out is looking at the dog.’}

\sn	{[...]}

\exi{47} \gll 			Tadedze’	ce’e 	yii 	kenał’aen,\\
driftwood	big		in	3PL.see\\

\exi{48} \gll 			łic’ae	utse’ 		k’e	dayizdaa,\\
dog		his.head		on	3S.sit\\

\exi{49} \gll 			tadedze’	gha		nihnidaetl’,\\
driftwood	to		3PL.arrive\\
\glt	\textit{'they arrive at the driftwood,'}

\exi{50} \gll 			tadedze’	\textbf{nahditaani}	ye kiigha	delts’ii.\\
driftwood	they.found.REL	there by		3PL.sit\\
\glt \textit{they are sitting by the driftwood they found.’}

\sn	{[...]}

\exi{52} \gll 			Tsets 	\textbf{nahditaani} 		k’e	dahdelts’ii.\\
wood	3PL.found.REL		on	3PL.sit\\
\glt	\textit{‘They sit down on the wood they found.’}

\sn	{[...]}

\exi{54} \gll 			’Unaan		tl’ogh	ta 		naghaay	c’a	’uka~\textbf{nasitelyaesi}, 	 kuts’e’		niłc’ayiłyaał.\\
across		grass	in		frog			\textsc{foc} 	3PL.look.for.it.\textsc{rel}	to.them		3S.jump		\\
\glt \textit{‘The frog they are looking for is jumping across to them on the grass.’}

\end{xlistn}
\begin{flushright}
((Markle Pete, \textit{Naghaay Ndaane Zidaa} ‘Frog Where Are You’,\\
oai.paradisec.org.au:ALB01-AHTNB001-060.tiff, ALB01-AHTNB001-061.tiff,\\
ALB01-AHTNB001-065.tiff, ALB01-AHTNB001-066.tiff))
\end{flushright}
\end{exe}



The relative clauses in lines 21 and 23 refer to the bees’ nest, which had been introduced in line 19. The relative clause in 24 refers to the squirrel introduced in line 22. The relative clauses in 50 and 52 refer to the driftwood that had been introduced in line 47, and the relative clause in 54 refers back to the frog, which had been introduced at the beginning of the story. Note that in terms of cognitive activation states of referents (e.g., \citealt{Chafe1994}), such careful tracking may not be strictly necessary in all cases. For instance, the squirrel first appears in line 22, and only one line intercedes between its appearance and the use of a relative clause to refer to it. There is no chance here for confusion with another squirrel, but Mr. Pete packages it carefully just the same. Similarly, the driftwood is introduced in line 47, referred to again in line 49, and then delimited with a relative clause in line 50.

Mr. Pete’s approach to the tasks of narrating \textit{Naghaay} is different from that of the travel narrators. At no point does he depart from the storyline, nor does he use \textit{yihwts’en}, \textit{xona}, or any other such marker to contrast storyline clauses with digression clauses. But where \textit{Naghaay} is lacking in discourse markers and digressions about locations, \textit{Ta’stedeł} \textit{dze’} and \textit{Saen} \textit{tah} \textit{xay} \textit{tah} are plainly lacking in relative clauses and elaborate tracking of characters.\footnote{Except for those that have lexicalized into toponyms, relative clauses occur only once in each travel narrative.} Note that the travel narrators were not asked specifically to avoid stories “about people”—rather, this genre is inherently about landscape, travel, and events over detail tracking of animate referents. The speakers here choose grammatical mechanisms for elaborating figure or ground that are consistent with the tasks they deem necessary for storytelling.

\section{Conclusion}

Typologists have used frog story narration to compare how languages express the notion of direction in motion events, and to make predictions about how a language is likely to behave based on those comparisons. As we have seen, Ahtna frog story narration does give us a glimpse into the resources of the language for expressing the notion of direction: Mr. Pete makes extensive use of the semantically general direction-describing adverbial prefixes. We have also seen that there is much about the grammar of direction that \textit{Naghaay} does not reveal. Had we relied solely on frog stories to tell us about Ahtna encoding of direction and direction in motion events, the descriptive richness and frequent use of the use of directionals and toponymy in the travel narratives would have remained hidden. Indeed, omitting either of these from a discussion of motion events would result in a poor description of Ahtna grammar.

Of course, the goal of typological frog story research is not to develop comprehensive descriptions of the grammar of direction and location for any single language, but to provide semantically unified content for cross-linguistic comparisons. Frog stories are attractive because they provide samples of connected speech, but because they are not the vivid, lived experiences relayed in the travel narratives, they are less likely to reveal what is most natural in discourse. Frog stories are told in a highly contrived setting. The narration of a children’s book is not an indigenous genre of Ahtna discourse in the way that the telling of travel narratives is, a point that was driven home by a female Ahtna consultant who refused to participate because “Ahtna people don’t keep frogs as pets.” We should be careful to include in any description examples from discourse that is more typical of the language community in which the grammatical structures we are investigating arose. In sum, there is nothing linguistic that prevents travel narratives from being focused on people rather than locations and events, but the traditional genre of travel narration comes with an ideology that influences how speakers use the linguistic resources available to them.

Finally, we also need to consider what speakers are actually attending to. During narration of a frog story, it is likely that unless the speaker is savvy enough to understand that a researcher is investigating the particulars of how the language segments direction and manner in motion events, he or she will be attuned to tasks other than providing a good sample for such research. It is far more likely that when asked to narrate the storybook, a speaker will try to do just that: to convey the events in the book in the order in which they happen with attention to whatever factors seem most important. For Mr. Pete, creating richly imagined characters and keeping track of them through the series of unusual events is important. For Mr. Sanford and Mr. Tansy, creating highly elaborated landscapes and providing background information and personal memories is important. If a frog story-narrator does not consider elaborate descriptions of direction to be essential to the storyline, he or she may leave them out in favor of other concerns. Thus we need to cast a wide net when making typological observations and take into account data from a range of sources (e.g., \citealt{ApplebaumBerez2009}). Ultimately it is not essential for frog story narrators to create a fully fleshed out sense of landscape, which can hide aspects of the grammar from us. \\

\bigskip


\begin{center}\textbf{Appendix: Transcription conventions and abbreviations}\end{center}

In the Ahtna examples, each line break indicates one intonation unit (IU, see \citealt{DuBois1992}; Du \citealt{DuBois2006}). The exception to this is examples from \textit{Naghaay}, in which each line corresponds to a normative sentence, which may include up to five intonation units each. Words or morphemes relevant to the discussion are highlighted. Other transcription symbols are as follows:

\begin{table}[!h]
    \begin{tabular}{l l }
 57 & Line number\\
(1.4) & Length of pause in seconds\\
, & Continuative intonation at the end of an intonation unit\\
. & Terminative intonation at the end of an intonation unit\\
@ & Laughter\\    \end{tabular}
\end{table}

\noindent
In the interest of economy, morpheme and word glosses are provided only when relevant to the argument at hand. The following abbreviations are used in this paper: \textsc{1} = ‘first person’, \textsc{3} = ‘third person’, \textsc{area} = ‘areal prefix’, \textsc{conj} = ‘conjunction’, \textsc{evid} = ‘evidential’, \textsc{foc} = ‘focus’, \textsc{iter} = 'iterative', \textsc{pcl} = ‘particle’, \textsc{pl} = ‘plural’, \textsc{pos} = ‘possessive’, \textsc{rel} = ‘relativizer’, \textsc{s} = ‘singular’, \textsc{sub} \textsc{=} \textsc{'subject'.}




\refheading
\bibliographystyle{ldc}
\bibliography{berez}

\orcidfooter{Andrea L. Berez-Kroeker}{aberez@hawaii.edu}{0000-0001-8782-515X}

\label{berez-ch-end}
