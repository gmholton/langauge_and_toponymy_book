Abert, J. W. 1848. Report of the Secretary of War Communicating, in Answer to a Resolution of the Senate, a Report and Map of the Examination of New Mexico, Made by Lieutenant J. W. Abert, of the Topographical Corps.  30\textsuperscript{th} Congress, 1\textsuperscript{st} Session, Executive [Document] 23. Washington.  [facsimile reprint, Lincoln County Heritage Trust, Lincoln, NM]

Acrey, Bill P.  1988.  \textit{Navajo History to 1846: The Land and the People}.  Shiprock, NM:  Department of Curriculum Materials Development, Central Consolidated School District No. 22.

Asociación de Academias de la Lengua Española 2010.  \textit{Diccionario de americanismos}.  Lima:  Santillana Ediciones Generales.

Automobile Club of Southern California. n.d.  \textit{Indian Country Guide Map: Arizona.Colorado.New Mexico. Utah}.  Los Angeles:  Automobile Club of Southern California.  [periodically updated]

Bailey, Lynn R.  1964a.  \textit{The Navajo Reconnaissance: A Military Exploration of the Navajo Country in 1859 by Capt. J. G. Walker and Maj. O. L. Shepherd}.  Los Angeles:  Westernlore Press.

Bailey, Lynn R. 1964b.  \textit{The Long Walk: A History of the Navajo Wars, 1846--1868}.  Great West and Indian Series, 64.  Los Angeles:  Westernlore Press.

Bailey, Lynn R. 1966.  \textit{Indian Slave Trade in the Southwest: A Study of Slavetaking and the Traffic in Indian Captives}.  Los Angeles:  Westernlore Press.

Barnes, Will C. \& Byrd H. Granger. 1960.  \textit{Arizona Place Names}.  Tucson:  The University of Arizona Press.

Beale, E. F. 1858. \textit{ Wagon Road from Fort Defiance to the Colorado River}.  35th Congress, 1st Session, House of Representatives Ex. Doc. 124.  Washington.

Beck, Warren A. \& Ynez D. Haase. 1969.  \textit{Historical Atlas of New Mexico.}  Norman:  University of Oklahoma Press.

Beers, Henry Putney. 1979.  \textit{Spanish and Mexican Records of the American Southwest.}  Tucson:  The University of Arizona Press, with the Tucson Corral of the Westerners.

Begay, Robert L. 2015a.  Kinyaa’áanii Clan History.  \textit{Leading the Way: The Wisdom of the Navajo People} 13(4). 2---4.

Begay, Robert L. 2015b.  Creation of the Owl and Eagle.  \textit{Leading the Way: The Wisdom of the Navajo People} 13(7). 2--3.

Bills, Garland D. \& Neddy A. Vigil. 2008.  \textit{The Spanish Language of New Mexico and Southern Colorado: A Linguistic Atlas}.  Albuquerque:  The University of New Mexico Press.

Bloom, Lansing S. (ed.).  1936.  “Bourke on the Southwest.”  \textit{New Mexico Historical Review} 11(1). 77--122; 11(3). 217--44.

Bodo, Murray (ed.). 1998. \textit{ Tales of an Endishodi: Father Berard Haile and the Navajos, 1900{}-1961.}  Albuquerque:  University of New Mexico Press.

Bowden, J. J.  2004.  Ojo del Espiritu Santo Grant.  New Mexico Office of the State Historian; \url{http://admin.newmexicohistory.org/filedetails.php?fileID=24960} (accessed 4 Sept. 2015).

Bowles, David E., Karim Aziz \& Charles L. Knight. 2000.  Macrobrachium (Decapoda): Caridea: Palaeomonidae) in the Contiguous United States: A Review of the Species and an Assessment of Threats to Their Survival.  \textit{Journal of Crustacean Biology} 20(1). 158--174.

Briggs, Walter. 1976.  \textit{Without Noise of Arms: The 1776 Domínguez-Escalante Search for a Route from Santa Fe to Monterey}.  Flagstaff, AZ:  Northland Press.

Bright, William. 2013.  \textit{Native American Placenames of the Southwest: A Handbook for Travelers}.  Norman:  University of Oklahoma Press.

Brooks, Clinton E. \& Frank D. Reeve (eds.). 1948.  \textit{Forts and Forays. James A. Bennett: A Dragoon in New Mexico, 1850--1856.}  Albuquerque:  The University of New Mexico Press.

Brooks, James F.  2002.  \textit{Captives and Cousins: Slavery, Kinship, and Community in the Southwest Borderlands}.  Chapel Hill:  University of North Carolina Press, for the Omohundro Institute of Early American History and Culture.

Brown, Cecil H.  1994.  Lexical Acculturation in Native American Languages.  \textit{Current Anthropology} 35(2). 95---117.

Brugge, David M.  1965.  \textit{Long Ago in Navajoland.}  Navajoland Publications, 6.  Window Rock, AZ: Navajo Tribal Museum.

Brugge, David M. 1973. \textit{Navajo Pottery and Ethnohistory}.  Window Rock, AZ:  Navajo Tribal Museum.

Brugge, David M. 1980.  \textit{A History of the Chaco Navajos}.  Reports of the Chaco Center, 4.  Albuquerque:  Division of Chaco Research, U.S. National Park Service.

Brugge, David M. 1985. \textit{Navajos in the Catholic Church Records of New Mexico}.  Tsaile, AZ:  Navajo Community College Press.  [repr. Santa Fe, NM:  SAR Press, 2010]

Brugge, David M. 1986. \textit{Tsegai: An Archaeological Ethnohistory of the Chaco Region}.  Washington:  U.S. National Park Service.]

Brugge, David M. 1993a. An Investigation of AIRFA Concerns Relating to the Fruitland Coal Gas Development Area.  Albuquerque:  Office of Contract Archaeology, University of New Mexico.

Brugge, David M. 1993b.  Eighteenth-Century Fugitives from New Mexico among the Navajos.  In June-el Piper (comp.), Alexandra Roberts \& Jenevieve Smith (eds.), \textit{Papers from the Third, Fourth, and Sixth Navajo Studies Conferences}, 279--283.  Window Rock, AZ:  Navajo Nation Historic Preservation Department.

Brugge, David M. 2006.  When Were the Navajo? In Regge N. Wiseman, Thomas C. O’Laughlin \& Cordelia T. Snow (eds.), \textit{Southwestern Interludes: Papers in Honor of Charlotte J. and Theodore R. Frisbie}, 45--52.  Archaeological Society of New Mexico, 32.  Albuquerque.

Bryan, Kirk.  1925.  Date of Channel-Trenching in the Arid Southwest.  \textit{Science} 62(1607). 338--344.

Burrill, Meredith F.  1956.  Toponymic Generics II.  \textit{Names} 4(4). 226--240.

Bye, Robert A. 1981.  Quelites—Ethnoecology of Edible Greens—Past, Present, Future.  \textit{Journal of Ethnobiology} 1(1). 109--123.

Callaghan, Catherine A. \& Geoffrey Gamble  1996.  Borrowing. In Ives Goddard (ed.), \textit{Handbook of North American Indians, vol. 17.  Languages}, 111--116.  Washington:  Smithsonian Institution.

Carey, Harold Jr.  2014.  Antonio el Pinto Chief of the Navajos.  Navajo People Culture and History; \url{http://navajopeople.org/blog/antonio-el-pinto-chief-of-the-navajos} (accessed 25 August 2015).

Chavez, Angelico. 1979.  Genízaros.  \textit{In }Handbook of North American Indians, vol. 9.  Southwest, Alfonso Ortiz, ed.  Pp. 198--200.  Washington:  Smithsonian Institution.

Cobos, Rubén. 1983.  \textit{A Dictionary of New Mexico and Southern Colorado Spanish}, 2\textsuperscript{nd} ed,.  Santa Fe:  Museum of New Mexico Press.

Conrad, Howard Louis. 1890.  \textit{“Uncle Dick” Wootton, the Pioneer Frontiersman of the Rocky Mountain Region}.  Chicago:  W. E. Dibble \& Co.  [repr. New York:  Time-Life Books, 1980]

Correll, J. Lee. 1970. \textit{Sandoval—Traitor or Patriot?  Navajo Historical Publications, Biographical Series, 1}.  Window Rock, AZ:  Navajo Tribal Museum.

Correll, J. Lee. 1979.  Through White Men’s Eyes: A Contribution to Navajo History 1, A Chronological Record of the Navajo People from Earliest Times to the Treaty of June 1, 1868.  Window Rock, AZ:  Navajo Heritage Center.

Cummings, Byron. 1910.  The Ancient Inhabitants of the San Juan Valley.  \textit{Bulletin of the University of Utah} 3(3:2), Archaeological Number 2. 1--45.

Díaz, Josef, ed. 2014. \textit{The Art and Legacy of Bernardo Miera y Pacheco: New Spain’s Explorer, Cartographer, and Artist}.  Santa Fe:  Museum of New Mexico Press.

Downer, Alan S. ca. 1988.  Navajo Nation Historic Preservation Plan Pilot Study: Identification of Cultural and Historic\textit{ Properties in Seven Arizona Chapters of the Navajo Nation}.  [Window Rock, AZ]:  Navajo Nation Historic Preservation Department and Navajo Nation Archaeology Department.

Downer, Alan S. ca. 1989. \textit{Navajo Nation Historic Preservation Plan Pilot Study: Identification of Cultural and Historic Properties in Six New Mexico Chapters of the Navajo Nation}.  [Window Rock, AZ]:  Navajo Nation Historic Preservation Department and Navajo Nation Archaeology Department.

Downs, James F.  1972.  \textit{The Navajo}.  New York:  Holt, Rinehart and Winston.

Dutton, Clarence E.1886.  Mount Taylor and the Zuñi Plateau.  In 49\textsuperscript{th} Congress, 1\textsuperscript{st} Session, House of Representatives, Executive Document, 13, pp. 105--198.  Washington:  Government Printing Office.  [available at Google Books]

Dyen, Isidore \& David F. Aberle  1974. \textit{Lexical Reconstruction: The Case of the Proto-Athapaskan Kinship System}.  London:  Cambridge University Press.

Ebright, Malcolm \& Rick Hendricks. 2006. \textit{The Witches of Abiquiu: The Governor, the Priest, the Genízaro Indians, and the Devil}.  Albuquerque:  University of New Mexico Press.

Eidenbach, Peter L.  2012.  \textit{An Atlas of Historic New Mexico Maps, 1550--1941}.  Albuquerque:  The University of New Mexico Press.

El Bienhablao. n.d.  (Website): \url{http://www.elbienhablao.es/significado-cachuli}; accessed 4 Sept. 2015.

Farmer, Malcolm F.  1953.  The Growth of Navajo Culture.  In  Irwin T. Sanders, Richard B. Woodbury, Frank J. Essene, Thomas P. Field, Joseph R. Schwendeman \& Charles E. Snow (eds.), \textit{Societies around the World.  1, Eskimo, Navajo, Baganda},   199--202.  New York:  The Dryden Press.

Ferguson, T. J. \& E. Richard Hart. 1985. \textit{A Zuni Atlas}.  The Civilization of the American Indian Series, 172.  Norman:  University of Oklahoma Press.

Fishler, Stanley A.  1953. \textit{In the Beginning: A Navaho Creation Myth}.  University of Utah Anthropological Papers, 13.  Salt Lake City:  The University of Utah Press.

Forbes, Jack D.1960.  \textit{Apache, Navaho, and Spaniard}.  Norman:  University of Oklahoma Press.

Foreman, Grant (ed.)  1941. \textit{A Pathfinder in the Southwest: The Itinerary of Lieutenant A. W. Whipple During His Explorations for a Railway Route from Fort Smith to Los Angeles in the Years 1853--1854}.  Norman:  University of Oklahoma Press.

Fort Huachuca, Arizona. n.d.  Huachuca on Maps [Website]. \url{http://huachuca.army.mil/pages/history/maps.html} [1849 Kern map].

Franciscan Fathers, The. 1910. \textit{An Ethnologic Dictionary of the Navaho Language}.  St. Michaels, AZ:  The Franciscan Fathers.

Franstead, Dennis. ca. 1979. \textit{An Introduction to the Navajo Oral History of Anasazi Sites in the San Juan Basin Area}.  Albuquerque:  University of New Mexico Chaco Project.

Frink, Maurice. 1968. \textit{Fort Defiance and the Navajos}.  Boulder, CO:  Pruett Press.

García de Diego, Vicente. 1955. \textit{Diccionário Etimológico Español e Hispanico}.  Madrid:  Editorial S. A. E. T. A.

Gelling, Margaret. 1988. \textit{Signposts to the Past: Place-names and the History of England}.  Chichester, West Sussex:  Phillimore.

Goodman, James M.  1982. \textit{The Navajo Atlas: Environments, Resources, People, and History of the Diné Bikeyah}.  Norman:  University of Oklahoma Press.

Gregory, Herbert E.  1916. \textit{The Navajo Country: A Geographic and Hydrographic Reconnaissance of Parts of Arizona, New Mexico, and Utah}.  Washington:  Government Printing Office.

Gregory, Herbert E. 1917.  \textit{Geology of the Navajo Country: A Reconnaissance of Parts of Arizona, New Mexico and Utah}.  Professional Paper, 93.  Washington:  U.S. Geological Survey.

Haas, Mary R.  1978. \textit{Language, Culture, and History: Essays by Mary R. Haas}.  Anwar S. Dil, comp.  Stanford, CA:  Stanford University Press.

Haile, Berard. 1938. \textit{\textit{Origin Legend of the Navajo Enemyway}}.  Publications in Anthropology, 17.  New Haven, CT:  Yale University.

Haile, Berard. 1950. \textit{A Stem Vocabulary of the Navaho Language, Navaho-English}.  St. Michaels, AZ:  St. Michaels Press.

Haile, Berard. 1951. \textit{A Stem Vocabulary of the Navaho Language, English-Navaho}.  St. Michaels, AZ:  St. Michaels Press.

Haile, Berard. 1978. \textit{Love-Magic and Butterfly People: The Slim Curly Version of the Ajiłee and Mothway Myths}.  Irvy W. Goossen and Karl W. Luckert, eds.  American Tribal Religions, 2.  Flagstaff:  Museum of Northern Arizona Press.

Haile, Berard. 1981. \textit{The Upward Moving and Emergence Way: The Gishin Biye’ Version}.  Karl W. Luckert (ed.). Lincoln:  University of Nebraska Press.

Harrington, John Peabody. 1916.  The Ethnogeography of the Tewa Indians.  Annual Report, 29.  Pp. 29---618.  Washington:  Bureau of American Ethnology.

Haskell, J. Loring. 1987. \textit{Southern Athapaskan Migration, A.D. 200--1750}.  Tsaile, Ariz.:  Navajo Community College Press.

Hedquist, Saul L., Stewart B. Koyiyumptewa, Peter M. Whiteley, Leigh J. Kuwanwisiwma, Kenneth C. Hill \& T. J. Ferguson. 2014.  Recording Toponyms to Document the Endangered Hopi Language.  \textit{American Anthropologist} 116(2). 324--331.

Hester, James J.  1962. \textit{Early Navajo Migrations and Acculturations in the Southwest}  Papers in Anthropology, 6.  Santa Fe:  Museum of New Mexico.

Hodge, Frederick Webb. 1937.  \textit{History of Hawikuh, New Mexico: One of the So-called Cities of Cíbola}. Publications of the Frederick Webb Hodge Anniversary Publication Fund, 1.  Los Angeles:  Southwest Museum.

Hoijer, Harry. 1956. The Chronology of the Athabaskan Languages. \textit{International Journal of American Linguistics} 22(4).219—32.

Holmes, Barbara E. 1989. \textit{American Indian Land Use of El Malpais}. Albuquerque: Office of Contract Archaeology, University of New Mexico.

Hughes, John T. 1848. \textit{Doniphan’s Expedition; Containing an Account of the Conquest of New Mexico; General Kearney’s Overland Expedition to California; Doniphan’s Campaign against the Navajos; His Unparalleled March upon Chihuahua and Durango; and the Operations of General Price at Santa Fé}.  Cincinnati:  U. P. James.  [repr. College Station:  Texas A\&M University Press, 1997; Texas A \& M University Military History Series, 56.]

Iverson, Peter. 2002. \textit{Diné: A History of the Navajos}  Albuquerque:  The University of New Mexico Press.

Ives, John W., Duane G. Froese, Joel C. Janetski, Fiona Brock \& Christopher Bronk Ramsey.  2014.  A High Resolution Chronology for Steward’s Promentory Culture Collections, Promontory Point, Utah.  \textit{American Antiquity} 79(4). 616--637.

Jett, Stephen C.  1964.  Pueblo Indian Migrations: An Evaluation of the Possible Physical and Cultural Determinants.  \textit{American Antiquity} 29(3). 281--300.


Jett, Stephen C.  1965.  Red Rock Country.  \textit{Plateau} 37(3). 80--84.

Jett, Stephen C. 1970.  An Analysis of Navajo Place-Names.  \textit{Names} 18(3). 175--184.

Jett, Stephen C.  1978.  The Origins of Navajo Settlement Patterns.  \textit{Annals of the Association of American Geographers} 68(3). 351--362.

Jett, Stephen C.  1982.  “Ye’iis Lying Down,” a Unique Navajo Sacred Place.  In   David M. Brugge \& Charlotte J. Frisbie (eds.), \textit{Navajo Religion and Culture: Selected Studies. Papers in Honor of Dr. Leland C. Wyman}, 138--149.   Museum of New Mexico, Papers in Anthropology, 17.  Santa Fe:  Museum of New Mexico Press.

Jett, Stephen C.  1992.  An Introduction to Navajo Sacred Places.  \textit{Journal of Cultural Geography} 13(2). 29--39.

Jett, Stephen C.  1997.  Place-Naming, Environment, and Perception among the Canyon de Chelly Navajo of Arizona.  \textit{The Professional Geographer} 49(4). 481--493.

Jett, Stephen C.  2001. \textit{Navajo Placenames and Trails of the Canyon de Chelly System, Arizona}.  American Indian Studies, 12.  New York:  Peter Lang Publishing.

Jett, Stephen C.  2006.  Reconstructing the Itineraries of Navajo Chantway Stories: A Trial at Canyon de Chelly, Arizona.  In Regge N. Wiseman, Thomas C. O’Laughlin \& Cordelia T. Snow (eds.) \textit{Southwestern Interludes: Papers in Honor of Charlotte J. and Theodore R. Frisbie},  75--86.  Archaeological Society of New Mexico, 32.  Albuquerque.

Jett, Stephen C.  2011.  Landscape Embedded in Language: The Navajo of Canyon de Chelly, Arizona, and Their Named Places. In David M. Mark, Andrew G. Turk, Niclas Burenhult \& David Stea (eds.), \textit{Landscape in Language: Transdisciplinary Perspectives}, 327--342.  Culture and Language Use: Studies in Anthropological Linguistics, 4.  Gunter Senft, ed.  Philadelphia:  John Benjamins Publishing Company.

Jett, Stephen C.  2014.  Place Names as the Traditional Navajo’s Title-deeds, Border-alert System, Remote Sensing, Global Positioning System, Memory Bank, and Monitor Screen.  \textit{Journal of Cultural Geography} 31(1). 106--113.

Julyan, Robert. 1998. \textit{The Place Names of New Mexico}.  Albuquerque:  The University of New Mexico Press.

Jurovics, Toby, Carol M. Johnson, Glen Willumson \& William F. Stapp. 2010. \textit{Framing the West: The Survey Photographs of Timothy H. O’Sullivan}.  Washington:  Library of Congress and Smithsonian American Art Museum/New Haven:  Yale University Press.

Kari, James. 1989.  Some Principles of Alaskan Toponymic Knowledge.  In Mary Ritchie Key \& Henry M. Hoenigswald (eds.), \textit{General and Amerindian Ethnolinguistics: In Remembrance of Stanley Newman},  129--151.  Berlin:  Mouton de Gruyter.


Kari, James. 2011.  A Case Study in Ahtna Athabascan Geographic Knowledge.  In David M. Mark, Andrew G. Turk, Niclas Burenhult \& David Stea (eds.), \textit{Landscape in Language: Transdisciplinary Perspectives},  239--260.  Culture and Language Use: Studies in Anthropological Linguistics, 4.  Gunter Senft, ed.  Philadelphia:  John Benjamins Publishing Company.

Kari, James. \& Ben A. Potter (eds.)  2010. \textit{The Dene-Yeniseian Connection}.  Anthropological Papers of the University of Alaska, N.S. 5(1--2).  Fairbanks, AK:  Department of Anthropology and Alaska Native Language Center.

Kearns, Robin A. \& Lawrence D. Berg. 2002.  Proclaiming Place: Towards a Geography of Place Name Pronunciation.  \textit{Social \& Cultural Geography} 3(3). 283--302.

Kelley, Klara Bonsack \& Harris Francis. 1994. \textit{Navajo Sacred Places}.  Bloomington:  Indiana University Press.

Kelly, Lawrence. 1970. \textit{Navajo Roundup: Selected Correspondence of Kit Carson’s Expedition against the Navajo, 1863--1865}.  Boulder, CO:  The Pruett Publishing Company.

Kessell, John L.  2002. \textit{Spain in the Southwest: A Narrative History of Colonial New Mexico, Arizona, Texas, and California}.  Norman:  University of Oklahoma Press.



Kessell, John L.  2013. \textit{Miera y Pacheco: A Renaissance Spaniard in Eighteenth-Century New Mexico}.  Norman:  University of Oklahoma Press.

Klah, Hasteen  1942. \textit{Navajo Creation Myth: The Story of the Emergence}.  Navajo Religion Series, 1.  Santa Fe:  Museum of Navajo Ceremonial Art.

Kluckhohn, Clyde \& Dorothea Leighton  1946. \textit{The Navaho}.  Cambridge, MA:  Harvard University Press.  [rev. ed., 1962]

Leopold, Luna B.  1951. \textit{Vegetation of Southwestern Watersheds in the Nineteenth Century}.  New York:  American Geographical Society.

Levy, Jerrold E.  1998. \textit{In the Beginning: The Navajo Genesis}.  Berkeley:  University of California Press.

Linford, Laurence D.  2000. \textit{Navajo Places: History, Legend, Landscape}.  Salt Lake City:  The University of Utah Press.

Linford, Laurence D.  2005.  Tony Hillerman’s Navajoland: Hideouts, Haunts, and Havens in the Joe Leaphorn and Jim Chee\textit{ }Mysteries, 2\textsuperscript{nd} ed.  Salt Lake City:  The University of Utah Press.

Madsen, Steven K.  2010. \textit{Exploring Desert Stone: John N. Macomb’s 1859 Expedition to the Canyonlands of the Colorado}.  Logan, UT:  Utah State University Press.

Marino, C. C.  1954. \textit{The Seboyetanos and the Navahos}.  New Mexico Historical Review 29(1). 8--27.

Marmon, Walter G.  1894.  Navajo Agency.  In \textit{Report on Indians Taxed and Not Taxed in the United States (Except Alaska) at the Eleventh Census: 1890},   154--160 + plates.  Washington:  U.S. Census Office.  [repr. New York:  Norman Ross Publishing, 1994]

Matson, R. G. \& M. P. R. Magne  2013.  North America: Na Dene/Athapaskan Archaeology and Linguistics. In Immanuel Ness (ed.), \textit{The Encyclopedia of Global Human Migration, 1, Prehistory},  1--7.  Hoboken, NJ:  Wiley-Blackwell.

Matthews, Washington (ed.)  1897. \textit{Navaho Legends}.  Boston:  Houghton Mifflin.  [repr. Salt Lake City:  The University of Utah Press, 1994]


Matthews, Washington (ed.)  1902. \textit{The Night Chant: A Navajo Ceremony}.  Memoirs, 6.  New York:  American Museum of Natural History.  [repr. Salt Lake City:  The University of Utah Press, 1995]

Matthews, Washington (ed.)  1907. \textit{Navaho Myths: Prayers and Songs with Texts and Translations}.  Pliny Earle Goddard, ed.  University of California Publications in American Archaeology and Ethnology, 5(2).  Berkeley.

McNitt, Frank. 1964. \textit{Navaho Expedition: Journal of a Military Reconnaissance from Santa Fe, New Mexico, to the Navajo Country Made in 1849 by Lieutenant James H. Simpson}.  Norman, OK:  University of Oklahoma Press.

McNitt, Frank. 1972. \textit{Navajo Wars: Military Campaigns, Slave Raids, and Reprisals}.  Albuquerque:  The University of New Mexico Press.

Mindeleff, Cosmos. 1897. \textit{The Cliff Ruins of Canyon de Chelly}.  Bureau of American Ethnology Annual Report, 16. 73--98.  Washington.

Mirkowich, Nicholas. 1941.  A Note on Navajo Place Names.  \textit{American Anthropologist} N.S. 43(2). 313--314.

Moore, William Haas. 1994. \textit{Chiefs, Agents, and Soldiers: Conflict on the Navajo Frontier, 1868--1882}.  Albuquerque:  The University of New Mexico Press.

Murphy, Lawrence R.  1967. \textit{Indian Agent in New Mexico: The Journal of Special Agent W. F. M. Arny, 1870}.  Southwestern Series, 5.  Santa Fe:  Stagecoach Press.

Myers, Garth Andrew. 1996.  Naming and Placing the Other: Power and the Urban Landscape.  \textit{Tijdschrift voor Economische en Sociale Geografie} 87. 237--246.

Newcomb, Franc Johnson. 1970. \textit{Navajo Bird Tales Told by Hosteen Klah Chee}.  Wheaton, IL:  Theosophical Publishing House.

O'Bryan, Aileen. 1956. \textit{The Dîné: Origin Myths of the Navajo Indians}.  Bulletin,  163.  Washington:  Bureau of American Ethnology.  [new ed., London:  Forgotten Books, 2008]

Ostermann, Leopold. 2004a.  First Impressions and Councils with Headmen, 1900.  In Howard M. Bahr (ed.), \textit{The Navajos as Seen by the Franciscans, 1898--1921},52--71.  Lanham, MD:  The Scarecrow Press.  [orig. pub. 1901]


Ostermann, Leopold. 2004b.  Neighbors. In Howard M. Bahr (ed.), \textit{The Navajos as Seen by the Franciscans, 1898--1921}, 173--181.  Lanham, MD:  The Scarecrow Press.  [orig. pub. 1902 ]

Pearce, T. M., ed. 1932.  The English Language in the Southwest.  \textit{New Mexico Historical Review} 7(3). 210--32.

Pearce, T. M., ed. 1965. \textit{New Mexico Place Names: A Geographical Dictionary}.  Albuquerque:  University of New Mexico Press.

Perry, Richard J.  1991. \textit{Western Apache Heritage: People of the Mountain Corridor}.  Tucson:  The University of Arizona Press.

Pine Hill High School students. 1982.  A Condensed History of the Ramah Navajo.  \textit{Tsá’ Ászi’} 5(1). 5--8.

Reeve, Frank D.  1956.  Early Navajo Geography.  \textit{New Mexico Historical Review} 31(4). 290--309.

Reeve, Frank D. 1971a.  Navajo Foreign Affairs, 1795--1846.   \textit{New Mexico Historical Review} 46(2). 100--132.  [repr. Tsaile, AZ:  Navajo Community College Press, 1983]

Reeve, Frank D. 1971b.  Navajo Foreign Affairs, 1795--1846. Part II, 1816--1824.  \textit{New Mexico Historical Review} 46(3). 223--251.  [repr. Tsaile, AZ:  Navajo Community College Press, 1983]

Reichard, Gladys. 1990. \textit{Navaho Religion: A Study in Symbolism}.  Bollingen Series, 18.  Mythos Series.  Princeton, NJ:  Princeton University Press.  [orig. pub. 1950]

Rice, Josiah. 1970. \textit{A Canoneer in Navajo Country: Journal of Private Josiah M. Rice, 1851}.  Richard H. Dillon, ed.  Denver:  Old West Publishing Company: Fred A. Rosenstock.

Robinson, Jacob S. 1848.  \textit{ Journal of the Santa Fe Expedition under Colonel Doniphan}, which left St. Louis in June, 1846.   Portsmouth, NH.  [repr. Santa Barbara, CA:  Narrative Press, 2001]

Salabye, John E. Jr.  2013.  Reader Question.  There is a place called Sǫ silá in the Wheatfields area. Why is this place important?  \textit{Leading the Way: The Wisdom of the Navajo People} 11(11). 10.

Salabye, John E. Jr.  \& Kathleen Manolescu. 2013.  Coyote and the Stars.  \textit{Leading the Way: The Wisdom of the Navajo People} 11(12). 20--21.

Sapir, Edward. 1921. \textit{Language: An Introduction to the Study of Speech}.  New York:  Harcourt, Brace.

Santamaría, Francisco J.  1959.  \textit{Diccionario de Mejicanismos}.  Mexico City:  Editorial Porrúa.

Schaafsma, Curtis F.  2002.  \textit{Apaches de Navajo: Seventeenth-Century Navajos in the Chama Valley of New Mexico}.  Salt Lake City: The University of Utah Press.

Seymour, Deni J. (ed.).  2012.  \textit{From the Land of Ever Winter to the American Southwest: Athabaskan Migrations, Mobility, and Ethnogenesis}.  Salt Lake City:  The University of Utah Press.

Seymour, Deni J. 2013.  Platform Cache Encampments: Implications for Mobility Strategies and the Earliest Ancestral Apaches.  \textit{Journal of Field Archaeology} 38(2). 161---172.

Shepardson, Mary. 1963.  \textit{Navajo Ways in Government: A Study in Political Process}.  Memoirs, 96.  [Menasha, WI]:  American Anthropological Association.

Simoons, Frederick J. 1994.  \textit{Eat Not This Flesh: Food Avoidances from Prehistory to the Present}, 2\textsuperscript{nd} ed.  Madison:  University of Wisconsin Press.

Sitgreaves, L[orenzo]1962.  \textit{Report on an Expedition down the Zuni and Colorado Rivers}.  Chicago:  Rio Grande Press.  [repr. from 32d Congress, 2d Session, Senate, Executive (Document) 59, Washington:  Robert Armstrong, Public Printer, 1853]

Spicer, Edward H.  1962.  \textit{Cycles of Conquest:  The Impact of Spain, Mexico, and the United States on the Indians of the Southwest, 1533--1960}.  Tucson:  The University of Arizona Press.

Stacey, May Humphries. 1929.  \textit{Uncle Sam’s Camels—His Journal, 1857-1858}, ed. Lewis B. Lesley.  Cambridge, MA:  Harvard University Press.

Steingass, F.  1987.  \textit{A Learner’s English-Arabic Dictionary}.  London:  Oriental University Press.

Studerus, Lenard. 2001.  \textit{A Thematic Introduction to New Mexico Spanish Nouns: Representative Nested Examples for New Mexico and Southern Colorado}.  Seattle:  Folkecastle Hill Publishing.

Swadesh, Frances. 1979.  The Structure of Hispanic-Indian Relations in New Mexico.  In Paul Kutsche (ed.), \textit{Survival of Spanish-American Villages}, 53--61.  Colorado Springs:  Research Committee, Colorado College.

The Santa Fe Site. n.d.  (Website): http://www.thesantafesite.com/articles-database/Matanza-{}-{}-A-New-Mexico-Celebration.html.

Towner, Ronald H. (ed.). 1996. \textit{The Archaeology of Navajo Origins}.  Salt Lake City:  University of Utah Press.

Trafzer, Clifford. 1982.  \textit{The Kit Carson Campaign: The Last Great Navajo War}.  Norman:  University of Oklahoma Press.

Twitchell, Ralph Emerson. 1976.  \textit{The Spanish Archives of New Mexico}, repr. ed., 2 vols.  New York:  Arno Press.  [orig. pub. Cedar Rapids, IA:  Torch Press, 1914]

Tyler, Daniel. 1984.  \textit{Sources for New Mexico History 1821--1848}.  Santa Fe:  Museum of New Mexico Press.

Underhill, Ruth. 1956.  \textit{The Navajos}.  Norman:  The University of Oklahoma Press.

van Cott, John W.1990. \textit{Utah Place Names: A Comprehensive Guide to the Origins of Geographic Names: A Compilation}.  Salt Lake City:  The University of Utah Press.

Van Valkenburgh, Richard F.  1937.  “The History of the Navajo Nation,” unpublished manuscript 284-994, Box 9, Richard Fowler Van Valkenburgh Collection, Arizona Historical Society, Tucson.

Van Valkenburgh, Richard F.  1974.  Navajo Sacred Places.  In \textit{Navajo Indians III}, 9--199.  New York:  Garland Publishing.

Van Valkenburgh, Richard F.  1999.  Diné Bikéyah (2\textsuperscript{nd} ed.).  Mancos, CO:  Time Traveler Maps.  [1\textsuperscript{st} ed., Window Rock, AZ:  U.S. Indian Service, 1941]

Van Valkenburgh, Richard F. \& Frank Walker. 1945.  Old Placenames in the Navaho Country.  \textit{The Masterkey} 19(3). 89---94.  Southwest Museum.

Vogt, Evon Z.  1961.  Navaho. In Edward H. Spicer (ed.),  \textit{Perspectives in American Indian Culture Change}, 278---236.  Chicago:  University of Chicago Press.

Waters, Frank. 1963.  \textit{Book of the Hopi}.  New York:  The Viking Press.

Watson, Editha L. 1964.  \textit{Navajo Sacred Places}.  Navajoland Publications, 5.  Window Rock, AZ:  Navajo Tribal Museum.

Weber, Anselm. 2004a.  The Navajo Trouble of 1905.  In Howard M. Bahr (ed.), \textit{The Navajos as Seen by the Franciscans, 1898--1921}, 221---243.  Lanham, MD:  The Scarecrow Press.  [orig. pub. 1902 ]

Weber, Anselm. 2004b.  1917: Council at Chin Lee. In Alfonso Ortiz (ed.), \textit{The Navajos as Seen by the Franciscans, 1898--1921}, 418---421.  Lanham, MD:  The Scarecrow Press.  [orig. pub. 1902 ]

Wheelwright, Mary C.  1938.  \textit{Tleji or Yehbechai Myth}.  Santa Fe:  House of Navajo Religion.

Wheelwright, Mary C. \& David P. McAllester. 1988. \textit{The Myth and Prayers of the Great Star Chant and the Myth of the Coyote Chant}.  Tsaile, AZ:  Navajo Community College Press.

Wilken, Robert L. 1955.  \textit{Anselm Weber, O.F.M., Missionary to the Navajo 1898-1921}.  Milwaukee, WI:  The Bruce Publishing Company.

Williams, Edwin B. 1955.  \textit{Spanish and English Dictionary/Diccionario Inglés y Español}.  New York:  Henry Holt and Company.

Wilson, Alan \& Gene Dennison. 1995.  \textit{Navajo Placenames: An Observer’s Guide}.  Guilford, CT:  Jeffrey Norton, Publishers.

Wilson, Joseph. 2015. The Union of Two Worlds: Reconstructing Elements of Proto-Athabaskan Folklore and Religion.  \textit{Folklore} 20. 1--24.

Winter, Mark. 2011. \textit{The Master Weavers: Celebrating One Hundred Years of Navajo Textile Artists from the Toadlena/Two Grey Hills Weaving Region}.  Newcomb, NM:  The Historic Toadlena Trading Post.

Wozniak, Frank J., Mead F. Kemrer \& Charles M. Carillo. 1992.  \textit{History and Ethnohistory along the Rio Chama}.  Albuquerque:  U.S. Army Corps of Engineers, Albuquerque District.



Wyman, Leland C. 1957.  \textit{Beautyway: A Navajo Ceremonial}.  Bollingen Series, 53.  New York:  Pantheon Books.

Wyman, Leland C. 1962.  \textit{The Windways of the Navahos}.  Colorado Springs:  Taylor Museum of the Colorado Springs Fine Arts Center.

Wyman, Leland C. 1970.  \textit{Blessingway}.  Tucson:  The University of Arizona Press.

Wyman, Leland C. 1975.  \textit{The Mountainway of the Navajo}.  Tucson:  The University of Arizona Press.



Young, Robert W. 1961.  \textit{The Navajo Yearbook, 8.}  Window Rock, AZ:  U.S. Bureau of Indian Affairs, Navajo Agency, Gallup Area Office.

Young, Robert W. 1983.  Apachean Languages.  In Alfonso Ortiz (ed.), \textit{Handbook of North American Indians, 10.  Southwest},  393---400.  Washington:  Smithsonian Institution.


Young, Robert W. \& William Morgan. 1980.  \textit{The Navajo Language: A Grammar and Colloquial Dictionary}.  Albuquerque:  The University of New Mexico Press.

de Zárate Salmerón, Gerónimo. 1966. \textit{Relaciones: An Account of Things Seen and Learned by Father Jerónimo de Zárati Salmerón from the year 1538 to year 1626}.  Alicia Kennedy Milich, tr. and ed.  Albuquerque: Horn \& Wallace.

Zolbrod, Paul G. 1984.  \textit{Diné Bahane’: The Navajo Creation Story.}  Albuquerque:  The University of New Mexico Press.
