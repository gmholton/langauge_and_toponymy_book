\chapter[Crossing the Etolin Strait and the Challenges Presented by a Single Place Name]{\vspace{-25pt}Crossing the Etolin Strait and the Challenges Presented by a Single Place Name}

\sethandle{10125/24848}
\def\authorlast{Drozda}
\renewcommand{\beginchapter}{\pageref{drozda-ch-begin}}
\renewcommand{\finishchapter}{\pageref{drozda-ch-end}}
\label{drozda-ch-begin}
\thispagestyle{firststyle}

\chapauth{Robert Drozda}
\affiliation{BIA-ANCSA}

\authortoc{Robert Drozda}




\section{Introduction}

Nunivak Island (Nuniwar) in the Bering Sea is separated from mainland Alaska and Nelson Island (Qaluyaaq)\footnote{While technically an island, geographically Nelson Island is considered part of mainland Alaska (cf. Pratt 2009).}  by the hazardous waters and strong currents of Etolin Strait  (Akularer/Akuluraq in Cup’ig/Yup’ik).\footnote{Official maps label the strait separating Nunivak from the mainland, Etolin Strait. It was named by the Russian mariner Khromchenko for Captain A. K. Etolin who first encountered it in 1821. Etolin earlier named it “Cook Strait” for Captain James Cook (Orth 1967: 320) and the latter name can still be found on some early maps. The native name on both sides of the strait is Akularer, with minor spelling variances reflecting differences in pronunciation between Cup’ig and Yup’ik.} Prior to the introduction of air travel in Western Alaska, Nunivak remained isolated from the mainland for all but a few months of the year. Unlike much of Alaska where frozen waterways facilitate travel, winter travel across Etolin strait was impossible because the shifting currents never allow it to completely freeze. Summer months too were fraught with hazards associated with small boat travel on the open ocean. Still, as expert ocean travelers some Nuniwarmiut were known to paddle their kayaks across the strait to Nelson Island and from there to points north and south.

The direct distance between Nunivak and Nelson Islands’ two closest points of land is less than 30 kilometers. Each point includes a historical occupation site, Englulrarmiut on the Nunivak side and Aternermiut at Nelson Island \hl{(figure 1)}. Both names are representative of the Cup’ig dialect historically spoken only at Nunivak Island. Aternermiut, the focus of this paper, is the only known habitation site outside of Nunivak Island with a Cup’ig name, a name nearly forgotten by Nuniwarmiut and also known to some mainland Central Yup’ik speakers who directly attribute it to the Nunivak people and their distinct language. The site is extensive with over 40 structural features and 35 graves reported. It is documented as a part of a larger Qaluyaarmiut settlement but has also functioned as a staging area and launch place for Nuniwarmiut traveling to and from their home island. A fragment of the bygone era of non-motorized indigenous travel is preserved in the place name Aternermiut.


\section{Akularer, Etolin Strait and the Inaccessibility of Nunivak Island}

Maps of the Bering Sea are deceptive with respect to the extent of Nunivak Island’s inaccessibility. For example Nunivak lies much closer to the mainland than do the Pribilof Islands or St. Lawrence Island, each of which experienced sustained Western contact much earlier than did Nunivak. But it is just this closeness, combined with its location at the fluctuating southern boundary of winter sea ice extent and the strength of tidal currents through the strait that makes access to the island so difficult. \hl{(photo-shifting pan ice of ES)}

The inaccessibility of Nunivak particularly with respect to the hazards of near-shore sea travel is well documented (Drozda 2010: 5-6; Griffin 2004: 116, Lantis XXXX, NOAA 2013; Pratt 2009: 99; VanStone 1957: 97). Most of the island is challenging to approach by sea and its waters remained largely uncharted well into contemporary times; today potential marine obstacles such as reefs and submerged rocks are still not fully surveyed. The uncertain navigation deterred mariners and was a major factor in delaying the effects of Western contact on the Nuniwarmiut (Lantis 1946: 161; VanStone 1957: 97),  who Lantis estimated in 1939-40 were “about fifty years behind Nome, Unalakleet, or Bethel in acculturation” (Lantis 1960:vi).

Etolin Strait also presented a formidable obstacle to both traditional (indigenous) travel and early contact, particularly of missionaries. John Kilbuck who established the Moravian Mission at Bethel in 1884 made but one journey to Nunivak staying less than one day in1 897,\hl{add endnote on JKs observations?} and the Jesuits established their church at Tununak (only 55 km from the Nunivak village of Mekoryuk) on Nelson Island in 1889, yet Christianity in the form of the Swedish Evangelical Covenant Church did not arrive at Nunivak until 1939 (\hl{reference}).


\section{Crossing the Strait}

According to the earliest references in the Alaska Coast Pilot, “tidal currents are so strong {[}in Etolin Strait{]} that the middle portion does not freeze over in winter. Navigation is difficult from mid-December to mid-May and usually is suspended from early January to late March” (NOAA 2013: 430).\footnote{This statement has seen little modification from its first printing in the 1908 Coast Pilot, where it reads, “It is stated that the tidal currents in Etolin Strait are so strong that the middle portion does not freeze over in winter.” (1908:40, 2013: 430).}   This is common knowledge among the Nuniwarmiut and is represented in their oral history. Nunivak elder Joe David (2005; Drozda 2010: 6) recounted a story involving a non-native shipwreck survivor who attempted to walk from Nunivak to the mainland. Feeling stranded the man gazed across the strait at Nelson Island, which is readily seen from eastern Nunivak on clear days; noting its nearness he disregarded warnings of unstable ice, attempted the crossing and drowned. Nelson Island elders speak of the hazard as well. Phillip Moses of Toksook Bay, stated, “it does not freeze, but the ice goes back and forth and there’s no way through it.” And John Alirkar said, “The current is evidently extremely strong in the ocean {[}Etolin Strait{]}” (CEC 2011: 157).

Nuniwarmiut are expert boatmen and prior to the introduction of motorized craft were known to make the crossing by kayak (Fienup-Riordan 1996: 156).  Nunivak elder Peter Smith stated with a characteristic simplicity, “I am an ocean man” (Smith 1986) \hl{Smith photo with kayaks}, and Kay Hendrickson and George Williams, Sr., both reported crossing the strait by kayak on more than one occasion (references). Author and skin boat builder Skip Snaith who has made an ethnographic study of the Nunivak kayak, reported Nuniwarmiut “retained large fleets of active kayaks into the {[}19{]}50s, and there was isolated use beyond that time.” He described the strait as “highly exposed and tide swept, and reported that “even today with aluminum boats and 150 h.p. outboards locals think long and hard before such an attempt {[}at crossing{]}” (Snaith 1999). Williams also noted that a kayaker would gauge the tide in order to make the crossing as quickly as possible, estimated at three to five hours on a calm day (Williams 1999). Travel between Nunivak and the mainland was almost exclusively a Nuniwarmiut venture (see \hl{Pratt and Lantis} in Pratt 2009).

In an apparent error Orth (1967:16) described a winter crossing of the strait by members of the Jarvis party (a.k.a. Overland Relief Expedition) in December 1897. Orth states the men traveled “(f)rom Nunivak Island… by dog teams across the delta and lake country to Andreafski, on the Yukon (River).” Other sources reveal the party actually departed from Nelson Island: “Captain Jarvis and his party left Cape Vancouver December 16, 1897, starting on a journey of eighteen hundred miles across the frozen waste.”(Bagley 1916: 416) Orth’s statement: “On December 16, 1897, he and three companions were landed on Nunivak Island by the revenue cutter Bear” is an error and “Nunivak” should be replaced with “Nelson” Island.


\section{Aternermiut – a Nuniwarmiut Staging Area at Nelson Island}

\begin{quote}
	“As this land was called Nelson Island, there was a saying, in those days, that 	there was a story that a settlement existed called Aternermiut. I personally saw 	this settlement. I know this settlement, as I used to travel by kayak between 	Nuniwar and Nelson Island. This settlement was abandoned when I started 	traveling by kayak. But before I traveled with a kayak, this settlement was 	inhabited” (Williams 1991b).\footnote{George Williams was born ca. 1922, so it’s fair to assume the site was abandoned, at least by Nunivakers by the mid or late 1930s.}
\end{quote}


The abandoned Aternermiut site was investigated by BIA ANCSA archeologists with Nelson Island Yup’iks\footnote{The plural form of Yup’ik is Yupiit, likewise the plural of Cup’ig is Cupiit and many researchers use those forms as well. However, since I write in English here I attach the plural “s.”}  in the summer of 1984. Based on interviews, investigators identified the site as a “spring and summer camp/village” and reported the name originated “from the Nunivak Island dialect and is associated with ‘going back to Nunivak Island’ (USBIA 1988).” Paul Agimuk an elder of the Nelson Island village of Tununak stated, “…Nunivak Island people called this Aternermiut” (Agimuk 1984). Nelson Islanders used the term as well, although the site is situated within a larger site complex referred to collectively as Up’nerkillermiut. There appears to be an inconsistency in documented site names and all names for the given sites may not be known to the nearby residents and users of the sites, including those elders who resided there as youths. As patterns of land use and site use change, the names can change too with a tendency from specific to general. Comparisons of the recorded names especially over the last 35 years reveal some names generalized over broader areas or forgotten altogether.

\hl{my interpretation is the site had two names, one used by Nunivakers and the other, Up’nerkillermiut, used by Nelson Islanders. As the site fell into disuse and was abandoned completely by the Nelson Islanders the name eventually came to refer to the entire complex of three sites. }

Nunivak Island Cup’ig place names are well documented (Drozda 1998), however, mainland names were not included in the structured recording process. Still, Aternermiut occurs on several Nunivak oral history recordings. George Williams, Sr., recalled Aternermiut in a historical narrative he told involving outsiders arriving at the island. Williams (1991) speculated\footnote{Williams clearly stated he did not witness these events, but was repeating them as part of the oral tradition.}  a group of foreigners known as Qaviayarmiut\footnote{The identity of the Qaviayarmiut remains in question, some Nunivakers believe they were from St. Lawrence Island (J. Williams 1986; Amos and Amos 2003: ), others claim they were Kawerak Inupiat, and others say it’s a generic term for any Eskimo north of Yup’ik territory, however the term may have a different meaning among the Nuniwarmiut.}  left Aternermiut and arrived at Qavlumiut or Taprarmiut\footnote{These two east coast Nunivak habitation sites are near each other.}  on Nunivak; eventually, he said, they split up and one group went to Amiigtulirmiut and another to Amkumiut\footnote{Amkumiut is not a place name. Reed provided the translation, “the ones who live over there,” while Amos offered it as a variant name for Taprarmiut (G. Williams 1991). Pratt 2009: 241-242 recorded a possible variant with the Nuniwarmiut subgroup name of Agkumiut, meaning “people of the east coast, in general.”}  \hl{(Figure ??, map) flesh out that footnote ix with elaboration on Y/C base am-/ag- from dictionaries.} Aternermiut figures prominently in at least two better-known Nuniwarmiut traditional tales (in English titled “The Dog Husband” and “The Giant Shrew,” although in published versions (cf. Lantis, Fienup-Riordan) the site has remained unnamed (cf. Williams 1991b).

Recent place name work at Nelson Island (cf, CEC 2012; Rearden and Fienup-Riordan.) reveals a point of land associated with the village of Up’nerkillermiut” was named Aterneq, this being the base word of Aternermiut.  There is no specific statement that Aterneq is a Cup’ig name in origin. Martina John (b. 1936) of Toksook Bay recalled:

\begin{quote}
    	“They say many kayakers from Nunivak used to arrive here (a place named 	Umkuuk) on Nelson Island in the past. They’d arrive with kayaks. And when they 	started to use boats, they’d continually come up with boats. … They always 	traveled up {[}to the Kuskokwim river area{]}. And when they returned home, they’d 	arrive here. We’d go up on top of Umkuuk (rd: near or part of Umkumiut; why is 	this a dual ending?) and search the {[}ocean{]}. And then after a while, across there 	beyond Cingigyaq (rd: described in ELOKA as “cape”, but more likely sandbars 	extending off the cape, there is a channel, Kuiguyurraq, that cuts across it) we’d 	see a sail, and sometimes there would be two. And when we climbed down, we’d 	tell them that we had seen boats. Then they’d head our way and arrive, and we’d 	see that they were people from Nunivak Island. Back in those days, since they 	weren’t educated in schools, their children would speak in their dialect, and they 	were fun to listen to” (Rearden 2011: 62-65).
\end{quote}

John spoke of Nelson Island men also going to the Kuskokwim River region for trade, but makes no mention of them traveling to Nunivak Island by boat. It is evident that Aternermiut was a place for Nuniwarmiut to stage and wait or gather in preparation for crossing the strait.


\section{Misplots and Errors}

Physically the site complex is well documented (Okada et al. 1982; U.S. BIA1988), yet until recently its location was (and in some cases still is) erroneously marked on maps in records and publications at the federal (USBIA), state (SHPO), regional (Calista Corporation and AVCP), local (Rearden and Riordan) and now global cyber level (ELOKA). Ironically the exact location of the site was ascertained by the author after consulting outdated Google Earth imagery in which site features (predominantly house depressions and food cache pits) can be clearly delineated and precisely matched to archeological ground survey maps. While these site maps appear very accurate, the site is plotted on base maps in at least three different locations.

Published map names often create confusion in the process of documenting traditional names and accurately placing them on maps. For example, USGS maps do not include the name Aternermiut (or variants) nor does the name occur in the Dictionary of Alaska Place Names (Orth 1967). However, the base (Aterner) is present in a nearby name with the variant spelling “Atrnak Point,” but, and here is the confusing part, apparently the USGS has transposed the names Atrnak Point (Aterner) and Uluruk Point (Ulurruk) on official maps, such that the Atrnak name is plotted about 5.6 km southeast of the site positively identified as Aternermiut. The actual location of Aterner(q?)/Aternermiut is at the point of land named Uluruk Point on the USGS map (USGS 195?). Errors of this sort are not uncommon on official maps of the region.


\section{Linguistic Considerations}

As previously mentioned the site complex that includes Aternermiut consists of three individually named sites and is also known by a collective name, variantly Up’nerkarmiut (Okada et al. 1982) and Up’nerkillermiut (U.S. BIA1988). While the three sites each have a Yup’ik name, Aternermiut is a Cup’ig variant (for which there is no Yup’ik equivalent) remembered by some Nelson Island people (reference). In addition to its Cup’ig name Aternermiut is also known by Yup’ik speaking Nelson Islanders as Up’nerkillermiut.

Aternermiut, is reported by Native language speakers on both sides as Nunivak in origin (Agimuk 1984; CEC 2013; Rearden 2011; USBIA 1988; ELOKA). Linguistically the place name remains somewhat of a puzzle with a number of different English translations provided over the years. Translations have been made by Yup’ik and Cup’ig translators on both sides of the strait, the etymology remains uncertain but rooted in a proto-Yupik language.

Translations of the base, Aterner include, “result of going down,” (Williams 1991a), “a place to step down” (Williams 1991b), “a place to prepare to go down” (BIA ANCSA 1984 {[}I seem to have lost the precise reference{]}), “one drifting out to sea, referring to kayaks leaving Nelson Island” (Rearden (2011:14-15). Another description or loose translation is; "to float away" (\hl{see NI trans - I seem to have lost the precise reference, maybe in Reardens text}).

Neither the Yup’ik Eskimo Dictionary (Jacobson 2012) nor the Cup’ig Eskimo Dictionary (Amos and Amos 2003) include the name; however both include an entry for Ater-, respectively: “to get down from something; to go down” and “to get down from a high place.” The name is derived from proto-Eskimo at(ə)-, meaning “down” and at(ə)r-, “go down (to shore)”(Fortescue et al. 2010:   ); in the case of Aterner, the post base  –ner, “result of” would allow for the implied meaning of “prepare to go down (cross to Nunivak).” With the –miut ending, the place name translates, “settlement of a place to prepare to go down (cross to Nunivak Island).”

Dissatisfied with the various translations I wondered, rather than referring to descending a hill or bluff, if “prepare to go down” or “place to step down” or “result of going down” might imply instead to “go down” (cross) to Nunivak Island. I put the question to Cup’ig speaker/dictionary compiler Howard Amos and he replied, “Yes, that is a good assumption, to prepare to go down to Nuniwar. I've heard older people using that term at moments of beginning their trek to Nuniwar [from Nelson Island]” (Amos 2013).


\begin{sidewaystable}[h]
    \centering\small
    \begin{tabular}{p{2cm} | p{3cm} | p{3cm} | p{6cm}}
BASE & Yup’ik (YED2012) & Cup’ig (CED2003) & note \\
\hline
ater-&to get down from something; to go down&to get down from a high place& \\
at’er & --- & to go down to river or coast & \\
atercete-, aterceta’arte- &to fish with a driftnet& --- & \\
aternir-&to blow from shore out to sea \# of wind&& \\
at &&& \\
Atn(e)q (Fortescue et al.)&&&PY ‘cape’ and CAY at(n)(e)q Cape Darby (place name) (CoED2010:53) \\
aterneq (NUN)&?&&Aterner in CED orthography; pb –neq(Y), -ner(C) <thing that results from V-ing> (YED) \\
aterte-&to drift with current&to float away; to drift away&PY at(ə)rtə, “drift(out to sea)” \\
atrar-, atr(ar)- (in NUN)&to go down; to descend&to go down to riverbank or coast&  \\
atrartuq (at’ertur in NUN) &he is going down&at’ertur he is going down& \\

&Reed (ANLC)&Amos &Other \\

Aternermiut&village/residents of Aterner <result of going down>&1) a place to step down. (Williams 1991b)
2) to prepare to go down [to Nunivak]&1) a place to prepare to go down (US BIA 1984).
2) to float away (ref?).
3) one drifting out to sea, referring to kayaks leaving Nelson Island” (Alice Rearden 2011: 14-15). \\
Atternermuit& --- & --- &Calista B-0799-C; AA-9734. Apl. Source: Tununak\\


\multicolumn{4}{l}{Related Nunivak Place Names} \\ \hline
Atrew’ig && a place to descend [MA]&NPNP III, 8.11 \\
Aterwig&place to go down, especially towards water, the beach[IR]&&Presumably 8.11; 86NUNpn5:3, 4,7,8; 86NUN25:10 \\
At'erwig&&NPNP I & II \\

    \end{tabular}
    \caption{Caption}
    \label{tab:my_label}
\end{sidewaystable}

\section{Conclusion}
Here, what might otherwise be a simple discussion or record of a single name provides an example of the complexity associated with toponymic research. Like many named places, Aternermiut may be clearly delineated in space, but like all place names its linguistic isolation or “singling out” removes it from a greater context or, “constellation of place names,” to borrow a term from Ray (1971:1). \hl{expand}

\refheading

\begin{hang}

\hl{Amos and Amos 2003}

Bagley, Clarence. 1916. \textit{History of Seattle from the Earliest Settlement to the Present Time, Volume 3.} The S. J. Clarke Publishing Company, Chicago.

CEC 2011

Griffin, Dennis. 2001. Contributions to the Ethnobotany of the Cup’it Eskimo, Nunivak Island, Alaska. \textit{Journal of Ethnobiology} 21(2). 91-127

Jacobson, Steven A. 2012. \textit{Yup'ik Eskimo Dictionary}. Fairbanks: Alaska Native Language Center.

Lantis 1960

Orth 1967

\hl{Pratt 2009}

On-site interview at Aternermiut 84BAY004 (Paul Agimuk) \& 033(Nicholaus Kailukiak).

Williams 1991

\end{hang}

\orcidfooter{Robert Drozda}{}{}

\label{drozda-ch-end}
